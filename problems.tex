\documentclass{article}
\usepackage[utf8]{inputenc}
\usepackage[russian]{babel}
\usepackage{hyperref}
\usepackage{amssymb}
\usepackage[a4paper]{geometry}
\begin{document}
{
	\thispagestyle{empty}
	\vspace*{\fill}
	\centering 

	\Large Сборник заданий для семинарских занятий \\
	по курсу \\
	<<Разработка мобильных приложений>>\\
	\large
	\vspace{40pt}
	\vspace*{\fill}
	\newpage
}

\tableofcontents
\newpage
\section{Общие сведения}

Сборник содержит задания для семинарских занятий по курсу <<Разработка мобильных приложений>>. 

Задачник расчитан на 48 часов семинарских занятий.

Структурно курс делится на две части:

\begin{enumerate}
	\item изучение языка Kotlin,\\
	\item изучение разработки на платформе Android.
\end{enumerate}

В практических и лабораторных работах предполагается использование языка Kotlin в средах
IntelliJ IDEA и Android Studio. При этом необходимо соблюдать 
\href{https://kotlinlang.org/docs/reference/coding-conventions.html}{Coding Conventions} (требования к стилю кода). 

Перед сдачей работы добейтесь, чтобы среда не выдавала предупреждений 
при запуске подпункта \textbf{Inspect code} пункта меню \textbf{Analyze}.

Задачи в большей степени расчитаны на освоение возможностей языка, а не
на алгоритмические сложности, потому
осуществляйте написание кода в соответствии с заданием, а не с целью, чтобы 
он просто работал.

Обратите внимание на то, что во всех заданиях необходимо проверять корректность входных данных, программа не должна <<падать>>
ни в каких ситуациях.

\section{Язык Kotlin}

\subsection{Практическая работа <<Основы языка Kotlin>> (2 часа)}

Цель первой работы -- обеспечить подготовку к выполнению заданий, 
нацеленных на освоение особенностей языка Kotlin. В первой работе
необходимо написать несколько небольших программ, чтобы убедиться в понимании
базовых конструкций языка (функция main, ветвления, циклы) и базовые типы (целые, вещественные, логический, символьный, массивы, строки), а также обеспечить успешную настройку среды разработки.

При выполнении заданий обращайте внимание на использование специфических особенностей языка везде, где это возможно: 
if и when могут быть как операторами, так и частью выражений; фигурные скобки во многих случаях можно опускать;
точки с запятой почти никогда не используются, корректно выбирайте, как помечать переменные: ключевым словом var или val.

При выполнении работ обеспечивайте \textbf{оптимальность} предлагаемой программы как по скорости, так и по памяти. В
случае противоречия между двумя критериями, выбирайте алгоритм, который обеспечивает лучшее быстродействие.

В частности, это обозначает, что нельзя использовать дополнительные строки (в заданиях, кроме как для ввода, строки не нужны),
следует избегать сложных структур (списков, множеств). В целом необходимо выполнить задание теми способами, которые использовались в 
курсе <<Основы алгоритмизации и программирования>>, но на новом языке.

\textbf{Задание №1} %for+if+ввод+вывод+строки: 15 минут


Для данного неотрицательного целого числа (в пределах Int)
найдите указанный результат. Осуществите проверку 
корректности ввода. Оформите программу и устраните все warning.

\begin{enumerate}
\item сумма четных цифр
\item сумма нечетных цифр
\item произведение четных цифр
\item произведение нечетных цифр
\item максимальную четную цифру
\item минимальную четную цифру
\item максимальную нечетную цифру
\item минимальную нечетную цифру

\item сумма цифр, кратных трем
\item сумма цифр, некратных трем
\item произведение цифр, кратных трем
\item произведение цифр, некратных трем
\item максимальную цифру, кратную трем
\item минимальную цифру, кратную трем
\item максимальную цифру, некратную трем
\item минимальную цифру, некратную трем

\item сумма цифр, стоящих на четных позициях в числе (если нумеровать цифры с конца): для числа 1234 ответ 4
\item произведение цифр, стоящих на четных позициях в числе (если нумеровать цифры с конца): для числа 1234 ответ 3
\item максимальная цифра среди стоящих на четных позициях в числе (если нумеровать цифры с конца): для числа 1234 ответ 3
\item минимальная цифра среди стоящих на четных позициях в числе (если нумеровать цифры с конца): для числа 1234 ответ 1

\item сумма цифр, стоящих на позициях в числе, номера которых кратны трем (если нумеровать цифры с конца): для числа 1234 ответ 2
\item произведение цифр, стоящих на позициях в числе, номера которых кратны трем (если нумеровать цифры с конца): для числа 1234 ответ 2
\item максимальная цифра среди стоящих на позициях в числе, номера которых кратны трем (если нумеровать цифры с конца): для числа 1234 ответ 2
\item минимальная цифра среди стоящих на позициях в числе, номера которых кратны трем (если нумеровать цифры с конца): для числа 1234 ответ 2

\item максимальная цифра среди стоящих на позициях в числе, номера которых кратны четырем (если нумеровать цифры с конца): для числа 1234 ответ 1 

\end{enumerate}

\textbf{Задание №2} % массивы и строки, 30 минут

\begin{enumerate}
\item С клавиатуры вводится описание массива из $10$ элементов в виде:

\textit{номер}:\textit{значение}

однако, порядок указания элементов может быть любой. Выведите все элементы
массива в порядке возрастания номеров.

\item В строке указано несколько неотрицательных целых чисел, разделенных пробелами (по одному
пробелу между числами). Какие цифры присутствуют в каждом числе?

\item В строке указано несколько неотрицательных целых чисел, разделенных пробелами (по одному
пробелу между числами). Какие цифры присутствуют хотя бы в двух числах?

\item В строке указано несколько неотрицательных целых чисел, разделенных пробелами (по одному
пробелу между числами). Какие цифры присутствуют ровно в одном числе?

\item В строке указано несколько неотрицательных целых чисел, разделенных пробелами (по одному
пробелу между числами). Какие цифры присутствуют ровно в двух числах?

\item В строке указано несколько неотрицательных целых чисел, разделенных пробелами (по одному
пробелу между числами). Какие цифры отсутствуют ровно в двух числах?

\item В строке указано несколько неотрицательных целых чисел, разделенных пробелами (по одному
пробелу между числами). Какие цифры отсутствуют ровно в одном числе?



\item В строке указано несколько слов, разделенных пробелами (по одному
пробелу между словами). Какие символы присутствуют в каждом слове? 
Предполагается, что все символы в строке имеют код, не больший, чем 127.

\item В строке указано несколько слов, разделенных пробелами (по одному
пробелу между словами). Какие символы присутствуют хотя бы в двух словах?
Предполагается, что все символы в строке имеют код, не больший, чем 127.


\item В строке указано несколько слов, разделенных пробелами (по одному
пробелу между словами). Какие символы присутствуют ровно в одном слове?
Предполагается, что все символы в строке имеют код, не больший, чем 127.


\item В строке указано несколько слов, разделенных пробелами (по одному
пробелу между словами). Какие символы отсутствуют ровно в одном слове?
Предполагается, что все символы в строке имеют код, не больший, чем 127.

\item В строке указано несколько слов, разделенных пробелами (по одному
пробелу между словами). Какие символы отсутствуют ровно в двух словах?
Предполагается, что все символы в строке имеют код, не больший, чем 127.



\item В строке указано несколько неотрицательных целых чисел, разделенных пробелами (по одному
пробелу между числами). Какое количество чисел удовлетворяет условию отсутствия
повторяющихся цифр? 

\item В строке указано несколько неотрицательных целых чисел, разделенных пробелами (по одному
пробелу между числами). Какое количество чисел удовлетворяет условию наличия
повторяющихся цифр? 

\item В строке указано несколько слов, разделенных пробелами (по одному
пробелу между словами). Какое количество слов удовлетворяет условию отсутствия
повторяющихся символов? 
Предполагается, что все символы в строке имеют код, не больший, чем 127.


\item В строке указано несколько слов, разделенных пробелами (по одному
пробелу между словами). Какое количество слов удовлетворяет условию наличия
повторяющихся символов? 
Предполагается, что все символы в строке имеют код, не больший, чем 127.


\item В строке указано несколько неотрицательных целых чисел, разделенных пробелами (по одному
пробелу между числами). В каком количестве чисел присутствуют все цифры от
0 до 9?

\item Имеется некоторая последовательность цифр от $0$ до $9$. 
С клавиатуры вводится $9$ строк следующего вида:

\textit{цифра}->\textit{цифра}

Каждая строка обозначает, что после цифры, стоящей до стрелки, в последовательности стоит цифра, стоящая после стрелки. 

Выведите исходную последовательность.

\item В строке указано несколько неотрицательных целых чисел, разделенных пробелами (по одному
пробелу между числами). Какие цифры присутствуют в каждом числе дважды?

\item В строке указано несколько неотрицательных целых чисел, разделенных пробелами (по одному
пробелу между числами). Какие цифры присутствуют хотя бы в одном числе дважды?

\item В строке указано несколько неотрицательных целых чисел, разделенных пробелами (по одному
пробелу между числами). Какие цифры присутствуют ровно в одном числе дважды?

\item С клавиатуры вводится несколько строк, последняя строка -- пустая (пустая строка -- признак окончания ввода и дальше игнорируется). Выведите символы, что присутствуют в каждой строке. Предполагается, что коды всех символов в строке не превышают 127.

\item С клавиатуры вводится несколько строк, последняя строка -- пустая (пустая строка -- признак окончания ввода и дальше игнорируется). Выведите символы, что присутствуют ровно в одной строке. 
Предполагается, что коды всех символов в строке не превышают 127.

\item С клавиатуры вводится несколько строк, последняя строка -- пустая (пустая строка -- признак окончания ввода и дальше игнорируется). Выведите символы, что присутствуют ровно в двух строках.
Предполагается, что коды всех символов в строке не превышают 127.


\item С клавиатуры вводится несколько строк, последняя строка -- пустая (пустая строка -- признак окончания ввода и дальше игнорируется). Выведите символы, что отсутствуют ровно в двух строках.
Предполагается, что коды всех символов в строке не превышают 127.



\end{enumerate}

\textbf{Задание №3}

Обратите внимание, что в этом задании требуется оптимальное решение 
(продвинутые функции дадут менее оптимальные решения из-за копирования). Можно предполагать,
что все символам соответствует одно значение типа char (однако,
если вы сделаете корректное решение (разумеется, вне пары), то это будет плюсом).

\begin{enumerate}
\item Найдите первый символ в первом максимально длинном слове с нечетным числом символов в строке (в строке указываются только слова, разделенные одним или несколькими пробелами). 
\item Найдите последний символ в первом максимально длинном слове с нечетным числом символов в строке (в строке указываются только слова, разделенные одним или несколькими пробелами). 
\item Найдите первый символ в последнем максимально длинном слове с нечетным числом символов в строке (в строке указываются только слова, разделенные одним или несколькими пробелами). 
\item Найдите последний символ в последнем максимально длинном слове с нечетным числом симвлов в строке (в строке указываются только слова, разделенные одним или несколькими пробелами). 
\item Найдите первый символ в первом самом коротком слове в строке с нечетным числом символов (в строке указываются только слова, разделенные одним или несколькими пробелами). 
\item Найдите последний символ в первом самом коротком слове в строке с нечетным числом символов (в строке указываются только слова, разделенные одним или несколькими пробелами). 
\item Найдите первый символ в последнем самом коротком слове в строке с нечетным числом символов (в строке указываются только слова, разделенные одним или несколькими пробелами). 
\item Найдите последний символ в последнем самом коротком слове в строке с нечетным числом символов (в строке указываются только слова, разделенные одним или несколькими пробелами). 

\item Найдите первый символ в первом максимально длинном слове с четным числом символов в строке (в строке указываются только слова, разделенные одним или несколькими пробелами). 
\item Найдите последний символ в первом максимально длинном слове с четным числом символов в строке (в строке указываются только слова, разделенные одним или несколькими пробелами). 
\item Найдите первый символ в последнем максимально длинном слове с четным числом символов в строке (в строке указываются только слова, разделенные одним или несколькими пробелами). 
\item Найдите последний символ в последнем максимально длинном слове с четным числом симвлов в строке (в строке указываются только слова, разделенные одним или несколькими пробелами). 
\item Найдите первый символ в первом самом коротком слове в строке с четным числом символов (в строке указываются только слова, разделенные одним или несколькими пробелами). 
\item Найдите последний символ в первом самом коротком слове в строке с четным числом символов (в строке указываются только слова, разделенные одним или несколькими пробелами). 
\item Найдите первый символ в последнем самом коротком слове в строке с четным числом символов (в строке указываются только слова, разделенные одним или несколькими пробелами). 
\item Найдите последний символ в последнем самом коротком слове в строке с четным числом символов (в строке указываются только слова, разделенные одним или несколькими пробелами). 

\item Найдите второй символ в первом максимально длинном слове с четным числом символов в строке (в строке указываются только слова, разделенные одним или несколькими пробелами). 
\item Найдите предпоследний символ в первом максимально длинном слове с четным числом символов в строке (в строке указываются только слова, разделенные одним или несколькими пробелами). 
\item Найдите второй символ в последнем максимально длинном слове с четным числом символов в строке (в строке указываются только слова, разделенные одним или несколькими пробелами). 
\item Найдите предпоследний символ в последнем максимально длинном слове с четным числом симвлов в строке (в строке указываются только слова, разделенные одним или несколькими пробелами). 
\item Найдите второй символ в первом самом коротком слове в строке с четным числом символов (в строке указываются только слова, разделенные одним или несколькими пробелами). 
\item Найдите предпоследний символ в первом самом коротком слове в строке с четным числом символов (в строке указываются только слова, разделенные одним или несколькими пробелами). 
\item Найдите второй символ в последнем самом коротком слове в строке с четным числом символов (в строке указываются только слова, разделенные одним или несколькими пробелами). 
\item Найдите предпоследний символ в последнем самом коротком слове в строке с четным числом символов (в строке указываются только слова, разделенные одним или несколькими пробелами). 

\item Найдите первый символ в первом максимально длинном слове с числом символов,кратным трем, в строке (в строке указываются только слова, разделенные одним или несколькими пробелами). 



\end{enumerate}


\subsection{Семинар <<Условные конструкции>> (2 часа)}

Во второй работе предполагается освоение оператора when.

При выполнении заданий обращайте внимание на использование специфических особенностей языка везде, где это возможно: 
if и when могут быть как операторами, так и частью выражений; фигурные скобки во многих случаях можно опускать;
точки с запятой почти никогда не используются, корректно выбирайте, как помечать переменные: ключевым словом var или val.

При выполнении работ обеспечивайте \textbf{оптимальность} предлагаемой программы как по скорости, так и по памяти. В
случае противоречия между двумя критериями, выбирайте алгоритм, который обеспечивает лучшее быстродействие.

В частности, это обозначает, что нельзя использовать дополнительные строки (в заданиях, кроме как для ввода, строки не нужны),
следует избегать сложных структур (списков, множеств). 

\textit {В задачах предполагается, что не используются специальные функции, предназначенные для работы с датами.}

\begin{enumerate}
\item Даны год, месяц и число (в виде чисел) -- выведите год, месяц и число для следующего дня (месяц надо выводить в виде названия, например <<1 января 2000 года>>).
\item Даны год, месяц и число (в виде чисел) -- выведите год, месяц и число для предыдущего дня (месяц надо выводить в виде названия, например <<1 января 2000 года>>).
\item Дано число от 1 до 1000 -- выведите его на русском языке, например <<семьсот семьдесят семь>>.
\item Дано число от 1 до 1000 -- выведите его на английском языке, например <<seven hundred seventy-seven>>.

\item Даны год, месяц и число (в виде чисел) -- выведите год, месяц и число для дня, который будет через неделю (месяц надо выводить в виде названия, например <<1 января 2000 года>>).
\item Даны год, месяц и число (в виде чисел) -- выведите год, месяц и число для дня, который был неделю назад (месяц надо выводить в виде названия, например <<1 января 2000 года>>).

\item Даны две даты одного года (то есть 5 чисел -- год, месяц, число, месяц, число) -- выведите фразу <<Между указанными датами прошло \textit{столько-то (словами)} дней>>.

\item По году, месяцу и числу выведите знак зодиака.

\item Даны год, месяц и число (в виде чисел) -- выведите год, месяц и число для того же дня следующего месяца, а если его не существует, то первого дня месяца, что идет после следующего (месяц надо выводить в виде названия, например <<1 января 2000 года>>).
\item Даны год, месяц и число (в виде чисел) -- выведите год, месяц и число для того же дня предыдущего месяца, а если его не существует, то последнего дня месяца, что идет до предыдущего (месяц надо выводить в виде названия, например <<1 января 2000 года>>).

\item Даны две даты одного года (то есть 5 чисел -- год, месяц, число, месяц, число) -- выведите фразу <<\textit{Number (for exampe, seventy-seven)} years have passed between the dates indicated>> на английском языке. 



\end{enumerate}




\subsection{Семинар <<Массивы в Kotlin>> (2 часа)}

В третьей работе предполагается освоение массивов языка Kotlin.

При выполнении заданий обращайте внимание на использование специфических особенностей языка везде, где это возможно: 
if и when могут быть как операторами, так и частью выражений; фигурные скобки во многих случаях можно опускать;
точки с запятой почти никогда не используются, корректно выбирайте, как помечать переменные: ключевым словом var или val.

При выполнении работ обеспечивайте \textbf{оптимальность} предлагаемой программы как по скорости, так и по памяти. В
случае противоречия между двумя критериями, выбирайте алгоритм, который обеспечивает лучшее быстродействие.

В частности, это обозначает, что нельзя использовать дополнительные строки (в заданиях, кроме как для ввода, строки не нужны),
следует избегать сложных структур (списков, множеств).

\textbf{Во всех заданиях предполагается использование массивов (не списков, не множеств и других подобных структур), которые оптимальны по скорости для больших по объему исходных данных}

\begin{enumerate}
\item С клавиатуры вводится описание массива из $10$ элементов в виде:

\textit{номер}:\textit{значение}

однако, порядок указания элементов может быть любой. Выведите все элементы
массива в порядке возрастания номеров.

\item В строке указано несколько неотрицательных целых чисел, разделенных пробелами (по одному
пробелу между числами). Какие цифры присутствуют в каждом числе?

\item В строке указано несколько неотрицательных целых чисел, разделенных пробелами (по одному
пробелу между числами). Какие цифры присутствуют хотя бы в двух числах?

\item В строке указано несколько неотрицательных целых чисел, разделенных пробелами (по одному
пробелу между числами). Какие цифры присутствуют ровно в одном числе?

\item В строке указано несколько неотрицательных целых чисел, разделенных пробелами (по одному
пробелу между числами). Какие цифры присутствуют ровно в двух числах?

\item В строке указано несколько неотрицательных целых чисел, разделенных пробелами (по одному
пробелу между числами). Какие цифры отсутствуют ровно в двух числах?

\item В строке указано несколько неотрицательных целых чисел, разделенных пробелами (по одному
пробелу между числами). Какие цифры отсутствуют ровно в одном числе?



\item В строке указано несколько слов, разделенных пробелами (по одному
пробелу между словами). Какие символы присутствуют в каждом слове? 
Предполагается, что все символы в строке имеют код, не больший, чем 127.

\item В строке указано несколько слов, разделенных пробелами (по одному
пробелу между словами). Какие символы присутствуют хотя бы в двух словах?
Предполагается, что все символы в строке имеют код, не больший, чем 127.


\item В строке указано несколько слов, разделенных пробелами (по одному
пробелу между словами). Какие символы присутствуют ровно в одном слове?
Предполагается, что все символы в строке имеют код, не больший, чем 127.


\item В строке указано несколько слов, разделенных пробелами (по одному
пробелу между словами). Какие символы отсутствуют ровно в одном слове?
Предполагается, что все символы в строке имеют код, не больший, чем 127.

\item В строке указано несколько слов, разделенных пробелами (по одному
пробелу между словами). Какие символы отсутствуют ровно в двух словах?
Предполагается, что все символы в строке имеют код, не больший, чем 127.



\item В строке указано несколько неотрицательных целых чисел, разделенных пробелами (по одному
пробелу между числами). Какое количество чисел удовлетворяет условию отсутствия
повторяющихся цифр? 

\item В строке указано несколько неотрицательных целых чисел, разделенных пробелами (по одному
пробелу между числами). Какое количество чисел удовлетворяет условию наличия
повторяющихся цифр? 

\item В строке указано несколько слов, разделенных пробелами (по одному
пробелу между словами). Какое количество слов удовлетворяет условию отсутствия
повторяющихся символов? 
Предполагается, что все символы в строке имеют код, не больший, чем 127.


\item В строке указано несколько слов, разделенных пробелами (по одному
пробелу между словами). Какое количество слов удовлетворяет условию наличия
повторяющихся символов? 
Предполагается, что все символы в строке имеют код, не больший, чем 127.


\item В строке указано несколько неотрицательных целых чисел, разделенных пробелами (по одному
пробелу между числами). В каком количестве чисел присутствуют все цифры от
0 до 9?

\item Имеется некоторая последовательность цифр от $0$ до $9$. 
С клавиатуры вводится $9$ строк следующего вида:

\textit{цифра}->\textit{цифра}

Каждая строка обозначает, что после цифры, стоящей до стрелки, в последовательности стоит цифра, стоящая после стрелки. 

Выведите исходную последовательность.

\item В строке указано несколько неотрицательных целых чисел, разделенных пробелами (по одному
пробелу между числами). Какие цифры присутствуют в каждом числе дважды?

\item В строке указано несколько неотрицательных целых чисел, разделенных пробелами (по одному
пробелу между числами). Какие цифры присутствуют хотя бы в одном числе дважды?

\item В строке указано несколько неотрицательных целых чисел, разделенных пробелами (по одному
пробелу между числами). Какие цифры присутствуют ровно в одном числе дважды?

\item С клавиатуры вводится несколько строк, последняя строка -- пустая (пустая строка -- признак окончания ввода и дальше игнорируется). Выведите символы, что присутствуют в каждой строке. Предполагается, что коды всех символов в строке не превышают 127.

\item С клавиатуры вводится несколько строк, последняя строка -- пустая (пустая строка -- признак окончания ввода и дальше игнорируется). Выведите символы, что присутствуют ровно в одной строке. 
Предполагается, что коды всех символов в строке не превышают 127.

\item С клавиатуры вводится несколько строк, последняя строка -- пустая (пустая строка -- признак окончания ввода и дальше игнорируется). Выведите символы, что присутствуют ровно в двух строках.
Предполагается, что коды всех символов в строке не превышают 127.


\item С клавиатуры вводится несколько строк, последняя строка -- пустая (пустая строка -- признак окончания ввода и дальше игнорируется). Выведите символы, что отсутствуют ровно в двух строках.
Предполагается, что коды всех символов в строке не превышают 127.



\end{enumerate}

\subsection{Семинар <<Строки в Kotlin>> (2 часа)}

В последней ознакомительной работе предполагается освоение работы со строками языка Kotlin (без использования специальных функций).

Обратите внимание, что в этом задании требуется оптимальное решение 
(продвинутые функции дадут менее оптимальные решения из-за копирования). Можно предполагать,
что все символам соответствует одно значение типа char (однако,
если вы сделаете корректное решение (разумеется, вне пары), то это будет плюсом).



\begin{enumerate}
\item Найдите первый символ в первом максимально длинном слове с нечетным числом символов в строке (в строке указываются только слова, разделенные одним или несколькими пробелами). 
\item Найдите последний символ в первом максимально длинном слове с нечетным числом символов в строке (в строке указываются только слова, разделенные одним или несколькими пробелами). 
\item Найдите первый символ в последнем максимально длинном слове с нечетным числом символов в строке (в строке указываются только слова, разделенные одним или несколькими пробелами). 
\item Найдите последний символ в последнем максимально длинном слове с нечетным числом симвлов в строке (в строке указываются только слова, разделенные одним или несколькими пробелами). 
\item Найдите первый символ в первом самом коротком слове в строке с нечетным числом символов (в строке указываются только слова, разделенные одним или несколькими пробелами). 
\item Найдите последний символ в первом самом коротком слове в строке с нечетным числом символов (в строке указываются только слова, разделенные одним или несколькими пробелами). 
\item Найдите первый символ в последнем самом коротком слове в строке с нечетным числом символов (в строке указываются только слова, разделенные одним или несколькими пробелами). 
\item Найдите последний символ в последнем самом коротком слове в строке с нечетным числом символов (в строке указываются только слова, разделенные одним или несколькими пробелами). 

\item Найдите первый символ в первом максимально длинном слове с четным числом символов в строке (в строке указываются только слова, разделенные одним или несколькими пробелами). 
\item Найдите последний символ в первом максимально длинном слове с четным числом символов в строке (в строке указываются только слова, разделенные одним или несколькими пробелами). 
\item Найдите первый символ в последнем максимально длинном слове с четным числом символов в строке (в строке указываются только слова, разделенные одним или несколькими пробелами). 
\item Найдите последний символ в последнем максимально длинном слове с четным числом симвлов в строке (в строке указываются только слова, разделенные одним или несколькими пробелами). 
\item Найдите первый символ в первом самом коротком слове в строке с четным числом символов (в строке указываются только слова, разделенные одним или несколькими пробелами). 
\item Найдите последний символ в первом самом коротком слове в строке с четным числом символов (в строке указываются только слова, разделенные одним или несколькими пробелами). 
\item Найдите первый символ в последнем самом коротком слове в строке с четным числом символов (в строке указываются только слова, разделенные одним или несколькими пробелами). 
\item Найдите последний символ в последнем самом коротком слове в строке с четным числом символов (в строке указываются только слова, разделенные одним или несколькими пробелами). 

\item Найдите второй символ в первом максимально длинном слове с четным числом символов в строке (в строке указываются только слова, разделенные одним или несколькими пробелами). 
\item Найдите предпоследний символ в первом максимально длинном слове с четным числом символов в строке (в строке указываются только слова, разделенные одним или несколькими пробелами). 
\item Найдите второй символ в последнем максимально длинном слове с четным числом символов в строке (в строке указываются только слова, разделенные одним или несколькими пробелами). 
\item Найдите предпоследний символ в последнем максимально длинном слове с четным числом симвлов в строке (в строке указываются только слова, разделенные одним или несколькими пробелами). 
\item Найдите второй символ в первом самом коротком слове в строке с четным числом символов (в строке указываются только слова, разделенные одним или несколькими пробелами). 
\item Найдите предпоследний символ в первом самом коротком слове в строке с четным числом символов (в строке указываются только слова, разделенные одним или несколькими пробелами). 
\item Найдите второй символ в последнем самом коротком слове в строке с четным числом символов (в строке указываются только слова, разделенные одним или несколькими пробелами). 
\item Найдите предпоследний символ в последнем самом коротком слове в строке с четным числом символов (в строке указываются только слова, разделенные одним или несколькими пробелами). 

\item Найдите первый символ в первом максимально длинном слове с числом символов,кратным трем, в строке (в строке указываются только слова, разделенные одним или несколькими пробелами). 
\end{enumerate}



\subsection{Практическая работа <<функции Kotlin>> (2 часа)}

\textbf{Задание №1} %15 минут

Измените программу, сделанную в задании №1 практической работы №1: основной алгоритм вынесите в функцию. 
Опишите функцию несколькими способами: 
\begin{enumerate}
	\item как обычную функцию;
	\item как tailrec-функцию.
\end{enumerate}

Обратите внимание на то, что основной алгоритм не должен содержать ввод-вывод и работу с любыми глобальными самописными объектами.

\textbf{Задание №2} %5 минут

В программах, сделанных в задании №1, вынесите проверяемое условие (то есть то условие, что сформулировано в задании)
в отдельную single-expression функцию. 

\textbf{Задание №3} %5 минут

Функции, созданные в задании №1, модифицируйте таким образом, чтобы условие, по которому происходит отбор, можно было передавать как
аргумент (один из аргументов функции должен быть функционального типа со значением по умолчанию -- условием, что указано было в вашем варианте).

\textbf{Задание №4} %15 минут 

В задании №3 первой практической работы реализуйте следующее: выделите основной алгоритм в отдельную функцию (без ввода и вывода, использования
глобальных переменных), куда в качестве аргумента передавайте функцию, имеющую смысл -- способ сравнения двух чисел; 
изменяя данную функцию, пользователь вашей функции (то есть другой программист) должен иметь возможность получить
информацию либо о самых длинных, либо о самых коротких словах. Аналогичным образом передавайте в вашу функцию функцию, которая будет
определять условие отбора слов.

\textbf{Задание №5} %10 минут

Создайте функцию, которая реализует алгоритм второго задания первой практической работы, в которую все числа, слова или пары (в зависимости от варианта)
передаются в аргументах функции. Например: f(123,25,222); f("dfd","dd","ddd"); d (Pair(2,3),Pair(3,5),Pair(4,1)). В этом задании вывод может осуществлять
во вновь создаваемой функции.

\textbf{Задание №6} 



\begin{enumerate}
	\item Создайте функцию, которая по данным функциям с параметром типа Int и результатами типа Int возвращает новую функцию -- сумму данных 
		(количество исходных функций -- любое).
	\item Создайте функцию, которая по данным функциям с параметром типа Int и результатами типа Int возвращает новую функцию -- произведение данных
		(количество исходных функций -- любое).
	\item Создайте функцию, которая по данным функциям с параметром типа Int и результатами типа Int возвращает новую функцию -- максимум данных
		(количество исходных функций -- любое).
  	\item Создайте функцию, которая по данным функциям с параметром типа Int и результатами типа Int возвращает новую функцию -- минимум данных
		(количество исходных функций -- любое).
	\item Создайте функцию, которая по данной функции $f:Int->Int$ и числу $n$ возвращает функцию $f(f(f(...f(x)...)$, где $f$ вызывается $n$ раз.
  	\item Создайте функцию, которая по данным функциям без параметров и результатам типа String возвращает новую функцию без параметров,
		что возвращает конкатенацию данных (количество исходных функций -- любое).
	\item Создайте функцию, которая по данным функциям с параметром типа Int и результатами типа Int возвращает новую функцию с аргументом $x$ типа Int,
		которая возвращает номер первой функции, имеющей максимальное значение, при подстановке в качестве аргумента $x$.
		(количество исходных функций -- любое).
	\item Создайте функцию, которая по данным функциям с параметром типа Int и результатами типа Int возвращает новую функцию с аргументом $x$ типа Int,
		которая возвращает номер первой функции, имеющей минимальное значение, при подстановке в качестве аргумента $x$.
		(количество исходных функций -- любое).
	\item Создайте функцию, которая по данным функциям с параметром типа Int и результатами типа Int возвращает новую функцию с аргументом $x$ типа Int,
		которая возвращает номер последней функции, имеющей максимальное значение, при подстановке в качестве аргумента $x$.
		(количество исходных функций -- любое).
	\item Создайте функцию, которая по данным функциям с параметром типа Int и результатами типа Int возвращает новую функцию с аргументом $x$ типа Int,
		которая возвращает номер последней функции, имеющей минимальное значение, при подстановке в качестве аргумента $x$.
		(количество исходных функций -- любое).
	\item Создайте функцию, которая по данным двум функциям с параметром типа Int и результатами типа Int? возвращает новую функцию -- сумму данных.
		Если результат хотя бы одной из суммируемых функций -- null, то и результат возвращаемой функции -- null.
	\item Создайте функцию, которая по данным двум функциям с параметром типа Int и результатами типа Int? возвращает новую функцию -- произведение данных.
	Если результат хотя бы одной из умножаемых функций -- null, то и результат возвращаемой функции -- null.
	\item Создайте функцию, которая по данным двум функциям с параметром типа Int и результатами типа Int? возвращает новую функцию -- максимум данных.
		Если результат хотя бы одной из исходных функций -- null, то и результат возвращаемой функции -- null.
\item Создайте функцию, которая по данным двум функциям с параметром типа Int и результатами типа Int? возвращает новую функцию -- минимум данных.
			Если результат хотя бы одной из исходных функций -- null, то и результат возвращаемой функции -- null.
		\item Создайте функцию, которая по двум данным функциям f(x) и g(x) возвращает функцию f(g(x)), параметры всех упомянутых функций
			имеют тип Int, результат -- Int?. Если функция g для данного x дает результат null, то результирующая функция так же
			равна null.
	\item Создайте функцию, которая по данной функции с параметром типа Int и результатом типа Int, а также целому числу $n$
		возвращает новую функцию, которая по массиву из $n$ элементов типа Int возвращает массив результатов применения функции $f$ 
		к каждому элементу данного массива.
	\item Создайте функцию, которая по данному массиву целых чисел возвращает функцию, которая при каждом вызове последовательно
		возвращает элементы массива, а когда элементы кончатся -- null.
	\item Создайте функцию, которая по данной функции, имеющей аргумент типа Int и результат типа Int, возвращает функцию, 
		которая при каждом вызове последовательно возвращает результаты применения функции-аргумента к числам $1$, $2$, $3$, \dots.
	\item Создайте функцию, которая по данному массиву целых чисел возвращает функцию, которая при каждом вызове последовательно
		возвращает элементы массива в обратном порядке, а когда элементы кончатся -- null.
	\item Создайте функцию, которая по данной строке возвращает функцию, которая при каждом вызове последовательно
		возвращает символы строки, а когда символы кончатся -- null.
	\item Создайте функцию, которая по данной строке возвращает функцию, которая при каждом вызове последовательно
		возвращает символы строки в обратном порядке, а когда символы кончатся -- null.
	\item Создайте функцию, которая по данным функциям с параметром типа Float и результатами типа Float возвращает новую функцию -- среднее 
		арифметическое данных
		(количество исходных функций -- любое).
	\item Создайте функцию, которая по данным функциям с параметром типа Float и результатами типа Float возвращает новую функцию -- среднее 
		квадратическое данных
		(количество исходных функций -- любое).
	\item Создайте функцию, которая по данным функциям с параметром типа Float и результатами типа Float возвращает новую функцию -- среднее 
		геометрическое данных
		(количество исходных функций -- любое).
	\item Создайте функцию, которая по данной функции $f:Float->Float$ и числу $x$ возвращает функцию, которая 
		при каждом вызове последовательно возвращает $f(x)$, $f(f(x))$, $f(f(f(x)))$, $\dots$.

\end{enumerate}

\subsection{Практическая работа <<Функции высшего порядка>> (4 часа)}
% TODO доделать количество вариантов до 25 
В данной работе требуется написать не самую оптимальную реализацию, а реализацию, 
которая наиболее полноценно использует функции над
коллекциями, использующие функциональный подход, и строковые функции. В работе запрещено использовать mutable коллекции и var переменные.

Реализация должна состоять из одной строки с точечными вызовами, включая ввод и вывод, использовать рекурсию запрещено.

Примечание: данный способ реализации программы нужен исключительно в учебных целях, в дальнейшем разбивайте подобные решения на небольшие функции,
которые удобно повторно использовать.

\textbf{Задания №№1-3} Реализуйте задания №№1-3 первой практической работы.

\textbf{Задание №4} С клавиатуры вводится несколько целых значений через пробел. Найдите (без учета тех чисел, где соответствующей цифры нет):

\begin{enumerate} %fold
%	\item Побитовое И последней цифры всех чисел
%	\item Побитовое ИЛИ последней цифры  всех чисел
%	\item Побитовое исключающее ИЛИ последней цифры всех чисел
	\item Побитовое И предпоследней цифры всех чисел
	\item Побитовое ИЛИ предпоследней цифры  всех чисел
	\item Побитовое исключающее ИЛИ предпоследней цифры всех чисел
	\item Побитовый штрих Шеффера последней цифры всех чисел (операции выполняются слева направо)
	\item Побитовый штрих Шеффера предпоследней цифры всех чисел (операции выполняются слева направо)
	\item Побитовую стрелку Пирса последней цифры всех чисел (операции выполняются слева направо)
	\item Побитовую стрелку Пирса предпоследней цифры всех чисел (операции выполняются слева направо)
	\item Побитовый штрих Шеффера последней цифры всех чисел (операции выполняются справа налево)
	\item Побитовый штрих Шеффера предпоследней цифры всех чисел (операции выполняются справа налево)
	\item Побитовую стрелку Пирса последней цифры всех чисел (операции выполняются справа налево)
	\item Побитовую стрелку Пирса предпоследней цифры всех чисел (операции выполняются справа налево)

	\item Побитовое И первой цифры всех чисел
	\item Побитовое ИЛИ первой цифры  всех чисел
	\item Побитовое исключающее ИЛИ первой цифры всех чисел
	\item Побитовое И второй цифры всех чисел
	\item Побитовое ИЛИ второй цифры  всех чисел
	\item Побитовое исключающее ИЛИ второй цифры всех чисел
	\item Побитовый штрих Шеффера первой цифры всех чисел (операции выполняются слева направо)
	\item Побитовый штрих Шеффера второй цифры всех чисел (операции выполняются слева направо)
	\item Побитовую стрелку Пирса первой цифры всех чисел (операции выполняются слева направо)
	\item Побитовую стрелку Пирса второй цифры всех чисел (операции выполняются слева направо)
	\item Побитовый штрих Шеффера первой цифры всех чисел (операции выполняются справа налево)
	\item Побитовый штрих Шеффера второй цифры всех чисел (операции выполняются справа налево)
	\item Побитовую стрелку Пирса первой цифры всех чисел (операции выполняются справа налево)
	\item Побитовую стрелку Пирса второй цифры всех чисел (операции выполняются справа налево)


\end{enumerate}

\textbf{Задание №5}

\begin{enumerate}
	\item С клавиатуры вводится информация о студентах: фамилия, имя, оценки. Выведите на экран информацию о трех лучших студентах по среднему баллу. В случае,
		если у нескольких студентов средний балл совпадает, то выведите большее число студентов (пока не будут выведены все студенты или 
		не будут полностью исчерпаны студенты с тремя лучшими баллами). Вывод надо осуществлять в порядке убывания среднего балла, а 
		для одинаковых средних баллов -- в алфавитном порядке по фамилии и имени.
	\item С клавиатуры вводится информация о студентах: фамилия, имя, оценки. Выведите на экран информацию о трех лучших студентах по максимальному баллу. В случае,
		если у нескольких студентов средний балл совпадает, то выведите большее число студентов (пока не будут выведены все студенты или 
		не будут полностью исчерпаны студенты с тремя лучшими баллами). Вывод надо осуществлять в порядке убывания максимального балла, а 
		для одинаковых максимальных баллов -- в алфавитном порядке по фамилии и имени.
	\item С клавиатуры вводится информация о студентах: фамилия, имя, оценки. Выведите на экран информацию о трех лучших студентах по минимальному баллу. В случае,
		если у нескольких студентов средний балл совпадает, то выведите большее число студентов (пока не будут выведены все студенты или 
		не будут полностью исчерпаны студенты с тремя лучшими баллами). Вывод надо осуществлять в порядке убывания минимального балла, а 
		для одинаковых минимальных баллов -- в алфавитном порядке по фамилии и имени.
	\item С клавиатуры вводится информация о студентах: фамилия, имя, оценки. Выведите на экран информацию о трех худших студентах по среднему баллу. В случае,
		если у нескольких студентов средний балл совпадает, то выведите большее число студентов (пока не будут выведены все студенты или 
		не будут полностью исчерпаны студенты с тремя худшими баллами). Вывод надо осуществлять в порядке возрастания среднего балла, а 
		для одниковых средних баллов -- в алфавитном порядке по фамилии и имени.
	\item С клавиатуры вводится информация о студентах: фамилия, имя, оценки. Выведите на экран информацию о трех худших студентах по максимальному баллу. В случае,
		если у нескольких студентов максимальный балл совпадает, то выведите большее число студентов (пока не будут выведены все студенты или 
		не будут полностью исчерпаны студенты с тремя худшими баллами). Вывод надо осуществлять в порядке возрастания максимального балла, а 
		для одинаковых максимальных баллов -- в алфавитном порядке по фамилии и имени.
	\item С клавиатуры вводится информация о студентах: фамилия, имя, оценки. Выведите на экран информацию о трех худших студентах по минимальному баллу. В случае,
		если у нескольких студентов минимальный балл совпадает, то выведите большее число студентов (пока не будут выведены все студенты или 
		не будут полностью исчерпаны студенты с тремя минимальными баллами). Вывод надо осуществлять в порядке возрастания минимального балла, а 
		для одинаковых минимальных баллов -- в алфавитном порядке по фамилии и имени.

	\item С клавиатуры вводится информация об абитуриентах: фамилия, имя, а далее названия предметов и оценки ЕГЭ по ним. Выведите на экран информацию о трех лучших абитуриентах по максимальному баллу за сумму трех ЕГЭ. В случае,
		если у нескольких абитуриентов средний балл совпадает, то выведите большее число абитуриентов (пока не будут выведены все абитуриенты или 
		не будут полностью исчерпаны абитуриентами с тремя лучшими баллами). Вывод надо осуществлять в порядке убывания максимальной суммы
		баллов за три ЕГЭ, а 
		для одинаковых сумм баллов -- в алфавитном порядке по фамилии и имени.
	\item С клавиатуры вводится информация об абитуриентах: фамилия, имя, а далее названия предметов и оценки ЕГЭ по ним. Выведите на экран информацию о трех худших абитуриентах по максимальному баллу за сумму трех ЕГЭ. В случае,
		если у нескольких абитуриентов средний балл совпадает, то выведите большее число абитуриентов (пока не будут выведены все абитуриенты или 
		не будут полностью исчерпаны абитуриентами с тремя худшими баллами). Вывод надо осуществлять в порядке возрастания максимальной суммы
		баллов за три ЕГЭ, а 
		для одинаковых сумм баллов -- в алфавитном порядке по фамилии и имени.
\end{enumerate}

\textbf{Задание №6}

\begin{enumerate}
	\item По номеру числа Фибоначчи найдите число Фибоначчи  (не используйте факты, которые вы не можете доказать самостоятельно)
	\item По числу Фибоначчи найдите его номер (не используйте факты, которые вы не можете доказать самостоятельно)
	\item По натуральному числу найдите его факториал
	\item По факториалу найдите исходное число
        \item По данному числу найдите простое число с таким номером (если простые числа нумеровать в порядке возрастания)
	\item По простому числу определите его номер в последовательности всех простых чисел, расположенных по возрастанию
	\item По данному числу найдите все его простые делители
	\item По числу $n$ найдите $n$-ое совершенное число (не используйте факты, которые вы не можете доказать самостоятельно)
	\item По совершенному числу найдите его номер в последовательности всех совершенных чисел, расположенных в порядке возрастания
                   (не используйте факты, которые вы не можете доказать самостоятельно)
	\item По натуральному числу найдите его двойной факториал
	\item По двойному факториалу найдите исходное число
	\item Рассмотрим все тройки натуральных чисел, удовлетворяющих уравнению $a^2+b^2=c^2$. Для данного $n$ найдите такую 
		тройку чисел $a$, $b$, $c$, что $a^2+b^2=c^2$, чтобы $a+b+c$ было меньше $n$ и наиболее близко к $n$.
	\item Рассмотрим все тройки натуральных чисел, удовлетворяющих уравнению $a^2+b^2=c^2$. Для данного $n$ найдите такую 
		тройку чисел $a$, $b$, $c$, что $a^2+b^2=c^2$, чтобы $a+b+c$ было больше $n$ и наиболее близко к $n$.
	\item По данному натуральному числу $n$ найдите наименьшее простое число, большее $n$
	\item По данному натуральному числу $n$ найдите наибольшее простое число, меньшее $n$
	\item По данному натуральному числу $n$ найдите наименьший факториал, больший $n$
	\item По данному натуральному числу $n$ найдите наибольший факториал, меньший $n$
	\item По данному натуральному числу $n$ найдите наименьший двойной факториал, больший $n$
	\item По данному натуральному числу $n$ найдите наибольший двойной факториал, меньший $n$
	\item По данному натуральному числу $n$ найдите наименьшее число Фибоначчи, большее $n$ (не используйте факты, которые вы не можете доказать самостоятельно)

	\item По данному натуральному числу $n$ найдите наибольшее число Фибоначчи, меньшее $n$ (не используйте факты, которые вы не можете доказать самостоятельно)

	\item По данному натуральному числу $n$ найдите наименьшее совершенное число, большее $n$ (не используйте факты, которые вы не можете доказать самостоятельно)

	\item По данному натуральному числу $n$ найдите наибольшее совершенное число, меньшее $n$ (не используйте факты, которые вы не можете доказать самостоятельно)
	\item Для данного натурального числа $n$ найдите такое простое число $p$, что входит в разложение на простые множители числа $n$ наибольшее число
		раз.
	\item Для данного натурального числа $n$ найдите такое простое число $p$, что входит в разложение на простые множители числа $n$ наименьшее число
		раз.
\end{enumerate}


\subsection{Семинар <<Поддержка ООП в Kotlin>> (2 часа)} 
% TODO неплохо бы добавить UNIT-тестирование


% взять из основ программирования
Напишите программу осуществляющую ввод информации о сущностях, описанных в вашем варианте задания и вывод на экран
некоторых из них. Количество вводимых сущностей не ограничено; обязательно использовать
ООП, инкапсуляцию, наследование и полиморфизм. Проверять корректность входных данных и делать проверку того,
что хватает памяти не обязательно. 

\begin{enumerate}
\item Товары Интернет-магазина -- книги и диски. Все товары определяются ценой, книги имеют название,
автора, количество страниц; диски -- название, количество треков.
 Выведите на экран все товары со стоимостью меньше 100 рублей.
\item Преподаватели определяются ФИО. Для тех, кто имеют диссертацию дополнительно вводится ее название;
для остальных -- стаж работы. Вывести всех преподавателей, у которых ФИО начинается на букву <<А>>.
\item Телефоны определяются названием модели. Проводные телефоны дополнительно определяются типом номеронабирателя
(диск или кнопки); а беспроводные -- дальностью действия радиосигнала.
Вывести все телефоны, название которых начинается на <<А>>.
\item Покатушки определяются названием и расстоянием. Однодневные катушки дополнительно определяются
плановым временем поездки (в часах). Многодневные катушки определяются количеством дней и категорией сложности
похода (от 1 до 6). Вывести все покатушки длиной более 100 км.
\item Музыкальная композиция определяется названием и композитором. Дополнительно для песни указывается
автор стихов. Выведите информацию о всех композициях, у которых композитор начинается на букву <<А>>.
\item Олимпиада определяется названием. Если олимпиада участвует в программе приема в ВУЗы дополнительно
указывается уровень олимпиады (1--3), если олимпиада -- этап всероссийской, то указывается название этапа
(школьная, окружная, региональная, всероссийская), в остальных случаях -- размер призового фонда.
Выведите все олимпиады, название которых начинается на букву <<А>>.
\item Проездной билет определяется стоимостью. Билет на количество поездок определяется 
количеством поездок. Билет на неограниченное количество поездок
определяется сроком действия (1 день, 5 дней, 10 дней, 15 дней, месяц, три месяца, 6 месяцев, год).
Выведите информацию о билетах, стоимостью меньше 300 рублей.
\item Информация о студенте определяется ФИО. Для студентов, не имеющих автомата, указывается балл,
полученный на экзамене (2--5); для студентов, имеющих автомат указывается основание (олимпиада или контрольные работы).
В случае, если контрольная работа -- то также указывается средний балл за к/р.
Выведите всю информацию о студентах с фамилией, начинающейся на буквы от А до К.
\item Сотовый телефон определяется названием. Для смартфонов указывается операционная система. А для других телефонов
-- наличие браузера. Выведите информацию  о телефонах, название которых содержит слово <<Nokia>>.

\item Куртка определяется названием модели, наличием капюшона. 
Для мембранных курток указывается степень водонепроницаемости 
(число в мм рт. ст.), для остальных -- наличием пропитки.
Выведите информацию обо всех куртках, имеющих капюшон.

\item Жесткий диск определяется названием и емкостью. Внешние жесткие диски определяются дополнительно
наличием системы, смягчающей последствия падения. Внутренние жесткие диски -- размером (2.5/3.5 дюйма).
Выведите информацию о дисках, емкостью больше 200 Гб.

\item Велосипед определяется названием модели. Горному велосипеду соответствует количество скоростей, BMX -- тип
конструкции (фривил, кассетная, фрикостер). Выведите информацию обо всех велосипедах,
содержащих в названии <<Norco>>.

\item Электронная книга определяется названием и размером экрана. Для EInk-дисплея указывается поколение
(pearl, vizplex); для LCD -- количество поддерживаемых цветов. Выведите информацию о всех книгах с размером экрана
не менее  7 дюймов.

\item GPS определяется названием, диагональю экрана. Для переносных GPS указывается наличие велосипедного
крепления; для автомобильных -- поддержка отображения пробок и наличие радар-детектора.
Выведите информацию обо всех GPS с размером экрана менее 7 дюймов.

\item Пылесос определяется названием модели. Для обычного пылесоса указывается мощность, для пылесоса-робота --
размер убираемого помещения и количество виртуальных стен. Выведите информацию обо всех пылесосах,
содержащих в названии слово Indesit.


\item  Туры определяются названием. Для пляжного тура указывается тип пляжа (галечный, песок); для экскурсионного --
количеством экскурсий. Выведите информацию обо всех турах, содержащих слово Египет.
\item Язык программирования определяется названием. Алгоритмические языки определяются поддержкой ООП
(отсутствует, на классах, прототипная), остальные языки -- типом (функциональный, логический, стиль ). 
Выведите информацию обо всех языках, название которых начинается с буквы <<А>>.
\item Контрагенты определяются названием. Индивидуальные предприниматели дополнительно определяются
наличием счета в банка, а юридические лица -- формой организации (ООО, ОАО. ЗАО).
Выведите информацию обо всех контрагентах, название которых начинается с буквы <<А>>.
\item Счет в банке определяется номером. Для текущего счета указывается плата за обслуживание, 
для сберегательного счета -- проценты годовых и наличие капитализации.
Выведите информацию обо всех счетах, номер которого начинается с 408178...
\item Автомобильная дорога определяется названием и километражом. Бесплатная дорога определяется статусом автомагистрали
(автомагистраль или нет), а платная -- стоимостью за километр для обычных пользователей. Выведите информацию о дорогах,
длина которых менее 100 км.
\item Офисное здание определяется адресом. В случае наличия стоянки указывается количество машиномест и стоимость
аренды за месяц. Выведите информацию о зданиях, в адресе которых присуствует слово Тверская.
\item Товары Интернет-магазина -- GPS-навигаторы и карты. 
Все товары определяются ценой и названием, GPS-навигаторы имеют назначение (ручной, автомобильный) и признак 
возможности загрузки карт;
карты -- размером (в Мб).  Выведите информацию о всех товарах со стоимостью менее 4000 рублей.
\item Товары Интернет-магазина --  чаи и кофе.
Все товары определяются ценой, названием и весом, кофе -- типом (растворимый, молотый, в зернах), чаи -- типом
(черный, зеленый). Выведите информацию о всех товарах с весом менее 150 г.
\item Объекты продаваемые в коттеджном поселке: участки (определяются площадью, стоимостью, наличием подряда), дома 
(определяются этажностью, площадью и стоимостью). Выведите все объекты со стоимостью меньше 1000000 рублей.
\item Вопросам теста соответствует формулировка и количество баллов за правильный ответ. 
Вопросам с вариантами правильных ответов соответствует 4 варианта ответа и номер правильного ответа;
остальным вопросам -- формулировка правильного ответа. Выведите все вопросы, оцениваемые в 10 баллов и выше.
\item Слова определяются собственно словом. Для существительных указывается род, для глаголов --
спряжение. Выведите информацию обо всех словах, начинающихся на букву <<А>>.
\item Операционная система определяется названием. Для операционной системы на базе Linux указывается
название менеджера пакетов; для остальных -- стоимость лицензии. Вывести все операционные системы, у которых название
начинается на букву <<A>>.
\item Рюкзаки определяются названием модели и емкостью. Для городских рюкзаков указывается наличие <<вентилируемой
спины>> для походных -- количество отделений и наличие крепления для трекинговых палок. Вывести информацию обо
всех рюкзаках, в названии которых присутствует слово <<Trek>>. 
\item Автостоянка определяется названием, количеством машиномест. Для крытой автостоянки указывается 
количество этажей. Для открытой стоянки -- наличие охраны. Вывести информацию обо всех автостоянках с количеством мест больше 20.
\item Партия определяется названием. Для тех партий, что финансируются из бюджета указывается размер
ассигнований, а для остальных -- количество депутатов в каких-либо представительных органах власти. Вывести
информацию обо всех партиях с названиями, начинающимися на буквы от <<А>> до <<К>>.
\end{enumerate}


\subsection{Семинар <<Перегрузка операторов в Kotlin>> (2 часа)} 

\begin{enumerate}
\item Реализуйте класс комплексного числа с реализацией операций +,-,*,/
\item Реализуйте класс обыкновенной дроби (состоящей из целой части, числителя и знаменателя) с реализацией операций +,-,*,/
\item Реализуйте класс 3-х мерного вектора с реализацией операторов +,-,* (векторное произведение), \% (скалярное произведение)
\item Реализуйте класс матрицы $2\times 2$ с реализацие операторов +,-,*, /.
\end{enumerate}


\subsection{Семинар <<Делегирование в Kotlin>> (2 часа)} 

Реализуйте несколько классов, предоставляющих возможность реализации паттерна Decorator посредством делегирования языка Kotlin. Обратите внимание на то, что представленные задания невозможно решить с помощью наследования без дублирования кода или изобретения сложных механизмов (кроме языков с множественным наследованием).

\begin{enumerate}
\item Осуществить ввод информацию о преподавателях и вывод информации о тех из них, у кого
средний балл учащихся выше 4. Преподаватели характеризуются фамилией, именем, отчеством,
полом, средним балом учащихся. Если преподаватель имеет категорию, то он также характеризуется датой последнего подтверждения категории и номером приказа об этом. Если преподаватель
имеет кандидатскую степень, то он также характеризуется темой диссертации, научным направлением и датой защиты.
\item Осуществить ввод информации о студентах и вывод информации о тех из них, у кого средний
балл выше 4. Студент характеризуется фамилией, именем, отчеством, полом, средним баллом.
Если студент – платник, то также номером договора и признак отсутствия задолженности по
оплате. Если студент – военнообязанный, то также номером приписного свидетельства или военного билета, названием военкомата, признаком того, что он проходил воинскую службу.
\item Осуществить ввод информации об остановочных пунктах пригородного железнодорожного транспорта. Вывести информацию о тех, где пассажиропоток выше 1000 человек/сутки. Остановочный
пункт характеризуется названием, пассажиропотоком, расстоянием от вокзала. Если остановочный пункт имеет турникетный комплекс, то он также характеризуется количеством турникетных
павильонов, стоимостью билета «на выход». Если остановочный пункт расположен вблизи станции метро, то он характеризуется названием ближайшей станции метро и расстоянием до нее.
\item Осуществите ввод информации о ВУЗах и вывод информации о тех из них, где учится больше
2000 человек. ВУЗ характеризуется названием, количеством студентом, адресом, номером лицензии. Если ВУЗ имеет аккредитацию, то он дополнительно характеризуется номером аккредитации и датой окончания аккредитации. Если ВУЗ выдает документы международного образца
то он дополнительно характеризуется стоимостью выдачи такого документа и регионом, где он
признается (например, весь мир, ЕС, США).
\item Осуществите ввод информации о планшетах и вывод информации о тех из них, что стоят меньше
15000 рублей. планшет характеризуется названием, производителем, стоимостью. Если планшет
поддерживает использование SIM-карты то дополнительно указывается поддержка 3G, поддержка LTE. Если планшет поддерживает Bluetooth, то указывается какой именно стандарт Bluetooth
поддерживается, возможно ли подключить Bluetooth клавиатуру.
\item Осуществить ввод информации о смартфонах и вывод информации о тех из них, что стоят меньше
10000 рублей. Смартфон определяется названием модели, названием производителя, ценой. Если
смартфон поддерживает WiFi, то дополнительно указывается стандарт WiFi и максимальная
стоимость. Если смартфон поддерживает геопозиционирование, то указывается поддерживает
ли он GPS, поддерживает ли он глонасс и количество каналов.
\item Осуществить ввод информации о телевизорах и вывод о тех из них, что стоят меньше 10000 рублей. Телевизор определяется названием марки, названием производителя, размером диагонали.
Если телевизор поддерживает Smart TV, то дополнительно указывается название стандрта Smart
TV, предварительно установленные программы. Если телевизор поддерживает многоканальный
звук то дополнительно указываетс наличие оптического выхода, наличие поканальных выходов,
\item Осуществить ввод информации о накопителях и вывод о тех из них, у которых емкость меньше 100 Гб. Накопитель определяется названием модели, названием производителя, емкостью в
гигабайтах. Если накопитель имеет интерфейс USB, то дополнительно указывается номер стандарта USB и скорость передачи информации. Если накопитель – жесткий диск, то дополнительно
указывается количество поверхностей и размер (физический).
\item Осуществить ввод информации о компьютерах и вывести информацию о компьютерах со стоимостью меньше 20000 рублей. Компьютер определяется названием производителя, названием
модели, стоимостью. Если компьютер является моноблоком, то он дополнительно определяется
размером диагонали дисплея. Если компьютер поддерживается WiFi, то дополнительно указывается стандарт WiFi и максимальная поддерживаемая скорость.
\item Осуществить ввод информации о телеканалах и вывести информацию о тех из них, аудитория
которых превышает 1000000. Телеканал определяется названием и размером аудитории. Если
телеканал является эфирным, то дополнительно указывается диапазон (МВ, ДМВ) и частота.
Если телеканал является государственным, то дополнительно указывается размер его финансирования.
\item Осуществить ввод информации о реках и выведите на экран те из них, что имеют длину более
20 км. Река определяется названием и длиной. Если река является судоходной, то дополнительно
указывается ширина и глубина основной части реки. Если река является водоснабжающей, то
указывается суточный водозабор и название водопроводной станции.
\item Осуществить ввод информации о памятниках и вывести информацию о тех из них, у которых
рейтинг выше 2.5. памятник определяется названием, адресом и рейтингом. Если памятник посвящен одному человеку, то указывается ФИО этого человека и годы его жизни. Если памятник
связан с военными действиями, то указывается название войны и годы войны.
\item Осуществить ввод информации о ПИФах и вывести информацию о тех, из них, что основаны до
2008 года. ПИФ определяется названием управляющей компании, названием ПИФа и годом основания. Для интервального ПИФа дополнительно указывается начало и окончание следующего
интервала (год, месяц, число). Для ПИФа акций указывается акции какого эшелона покупаются
(первого, второго, другие).
\item Осуществить ввод информации о вкладах и вывести информацию о тех, у которых доходность
больше 9 процентов. Вклад определяется названием, банком и процентом. Для вкладов, где разрешены взносы, дополнительно указывается за сколько дней до окончания можно делать взносы
и минимальный размер взноса. Для вкладов, где разрешено снятие, дополнительно указывается
через сколько дней после начала вклада разрешено снятие и размер неснижаемого остатка.
\item Осуществить ввод информации о маршрутах электричек и вывести те из них, длина которых
превышает 50 км. Маршрут определяется названиями отправного и конечного пункта и длиной
маршрута. Если маршрут является экспрессным, то дополнительно указывается стоимость проезда и время в пути. Если маршрут является межобластным, то дополнительно указываются
названия областей.
\item Осуществить ввод информации о холодильниках и вывести информацию о тех из них, высота которых превышает 1.5 метра. Холодильник определяется высотой, шириной, глубиной, названиями
модели и производителя. Для холодильников с морозильником дополнительно указывается высота, глубина и ширина морозильной камеры. Для холодильников имеющих зону свежести также
указывается высота, глубина и ширина зоны свежести.
\item Осуществить ввод информации о наушниках и вывести информацию о тех из них, которые стоят
больше 500 рублей. Наушник определяется названием производителя, названием модели, наличием системы активного шумоподавления, ценой. Для беспроводных наушников дополнительно
указывается радиус действия и флаг того, что сигнал цифровой. Для вставных наушников дополнительно указывается количество пар сменных амбушюров в комплекте и наличие регулятора
громкости.
\item Осуществить ввод информации об электронных книгах и вывести информацию о тех из них,
которые стоят больше 3000 рублей. Электронная книга определяется названием модели и названием производителя, стоимостью. Если электронная книга имеет дисплей на основе электронных чернил, то дополнительно указывается цветной дисплей или черно-белый, а также наличие
встроенной подсветки. Если электронная книга имеет сенсорный интерфейс, то дополнительно
указывается его тип (емкостный, резистивный, индуктивный, инфракрасный).
\item Осуществить ввод информации об устройствах GPS и вывести информацию о тех из них, что
стоят больше 5000 рублей. Устройство определяется названиями моделей и производителя, стоимостью и флагом поддержки Глонасс. Если устройство туристическое, то дополнительно указывается флаг возможности нахождения под водой, наличие магнитного компаса и тип батареи
(обычная или специальная). Если устройство имеет цветной дисплей, то указывается количество
поддерживаемых цветов.
\item Осуществить ввод информации о туристических палатках и вывести информацию о тех из них,
у которых водостойкость дна больше 7000. Палатка определяется названием производителя и
названием модели, водостойкостью тента и дна. Если палатка имеет внутреннюю палатку, то
указывается указывается ширина и высота внутренней палатки. Если палатка многоместная, то
указывается количество мест и количество входов.
\item Осуществить ввод информации об обогревателях и вывести информацию о тех из них, что стоят
больше 1500 рублей. Обогреватель определяется названиями производителя и модели, стоимостью, мощностью Если обогреватель инфракрасный, то дополнительно указывается тип (галогенный, карбоновый или кварцевый) и флаг наличия автоматического вращения. Если обогреватель
режим программирование то указываются флаги возможности отсрочки старта и регулировки
температуры.

\end{enumerate}


\subsection{Практическая работа <<Абстрактное программирование в Kotlin>>}
% TODO усложнить, подумать об in- и out- параметрах, а также where

\textbf{Задание №1}

\begin{enumerate}
	\item Создайте функцию, которая по данным функциям с параметром любого типа и результатами типа Int возвращает новую функцию -- сумму данных 
		(количество исходных функций -- любое).
	\item Создайте функцию, которая по данным функциям с параметром любого типа и результатами типа Int возвращает новую функцию -- произведение данных
		(количество исходных функций -- любое).
	\item Создайте функцию, которая по данным функциям с параметром любого типа и результатами типа Int возвращает новую функцию -- максимум данных
		(количество исходных функций -- любое).
  	\item Создайте функцию, которая по данным функциям с параметром любого типа и результатами типа Int возвращает новую функцию -- минимум данных
		(количество исходных функций -- любое).
	\item Создайте функцию, которая по данной функции $f:T->T$ и числу $n$ возвращает функцию $f(f(f(...f(x)...)$, где $f$ вызывается $n$ раз. Здесь $T$ -- любой тип.
  	\item Создайте функцию, которая по данным функциям с единственным параметром типа $T$ и результатами типа String возвращает 
		новую функцию с параметром типа $T$, что возвращает конкатенацию данных (количество исходных функций -- любое).
	\item Создайте функцию, которая по данным функциям с параметром типа $T$ и результатами типа Int возвращает новую функцию с аргументом $x$ типа $T$,
		которая возвращает номер первой функции, имеющей максимальное значение, при подстановке в качестве аргумента $x$.
		(количество исходных функций -- любое). Здесь $T$ -- любой тип.
	\item Создайте функцию, которая по данным функциям с параметром типа $T$ и результатами типа Int возвращает новую функцию с аргументом $x$ типа $T$,
		которая возвращает номер первой функции, имеющей минимальное значение, при подстановке в качестве аргумента $x$.
		(количество исходных функций -- любое). Здесь $T$ -- любой тип.

	\item Создайте функцию, которая по данным функциям с параметром типа $T$ и результатами типа Int возвращает новую функцию с аргументом $x$ типа $T$,
		которая возвращает номер последней функции, имеющей максимальное значение, при подстановке в качестве аргумента $x$.
		(количество исходных функций -- любое). Здесь $T$ -- любой тип.





	\item Создайте функцию, которая по данным функциям с параметром типа $T$ и результатами типа Int возвращает новую функцию с аргументом $x$ типа $T$,
		которая возвращает номер последней функции, имеющей минимальное значение, при подстановке в качестве аргумента $x$.
		(количество исходных функций -- любое). Здесь $T$ -- любой тип.

	\item Создайте функцию, которая по данным двум функциям с параметром типа $T$ и результатами типа Int? возвращает новую функцию -- сумму данных.
		Если результат хотя бы одной из суммируемых функций -- null, то и результат возвращаемой функции -- null. Здесь $T$ -- любой тип.






	\item Создайте функцию, которая по данным двум функциям с параметром типа $T$ и результатами типа Int? возвращает новую функцию -- произведение данных.
	Если результат хотя бы одной из умножаемых функций -- null, то и результат возвращаемой функции -- null. Здесь $T$ -- любой тип.
	\item Создайте функцию, которая по данным двум функциям с параметром типа $T$ и результатами типа Int? возвращает новую функцию -- максимум данных.
	Если результат хотя бы одной из исходных функций -- null, то и результат возвращаемой функции -- null. Здесь $T$ -- любой тип.
	\item Создайте функцию, которая по данным двум функциям с параметром типа $T$ и результатами типа Int? возвращает новую функцию -- минимум данных.
	Если результат хотя бы одной из исходных функций -- null, то и результат возвращаемой функции -- null. Здесь $T$ -- любой тип.

	\item Создайте функцию, которая по двум данным функциям f(x) и g(x) возвращает функцию f(g(x)), параметры всех упомянутых функций
			имеют тип $T$, результат -- $T?$. Если функция g для данного x дает результат null, то результирующая функция так же
			равна null. Здесь $T$ -- любой тип.

	\item Создайте функцию, которая по данной функции с параметром типа $T$ и результатом типа Int, а также целому числу $n$
		возвращает новую функцию, которая по массиву из $n$ элементов типа $T$ возвращает массив результатов применения функции $f$ 
		к каждому элементу данного массива. Здесь $T$ -- любой тип.

	\item Создайте функцию, которая по данному массиву значений типа $T$ возвращает функцию, которая при каждом вызове последовательно
		возвращает элементы массива, а когда элементы кончатся -- null. Здесь $T$ -- любой тип.

	\item Создайте функцию, которая по данной функции, имеющей аргумент типа Int и результат произвольного типа, возвращает функцию, 
		которая при каждом вызове последовательно возвращает результаты применения функции-аргумента к числам $1$, $2$, $3$, \dots.
	\item Создайте функцию, которая по данному массиву значений произвольного типа возвращает функцию, которая при каждом вызове последовательно
		возвращает элементы массива в обратном порядке, а когда элементы кончатся -- null.
%	\item Создайте функцию, которая по данной строке возвращает функцию, которая при каждом вызове последовательно
%		возвращает символы строки, а когда символы кончатся -- null.
%	\item Создайте функцию, которая по данной строке возвращает функцию, которая при каждом вызове последовательно
%		возвращает символы строки в обратном порядке, а когда символы кончатся -- null.
%	\item Создайте функцию, которая по данным функциям с параметром типа Float и результатами типа Float возвращает новую функцию -- среднее 
%		арифметическое данных
%		(количество исходных функций -- любое).
%	\item Создайте функцию, которая по данным функциям с параметром типа Float и результатами типа Float возвращает новую функцию -- среднее 
%		квадратическое данных
%		(количество исходных функций -- любое).
%	\item Создайте функцию, которая по данным функциям с параметром типа Float и результатами типа Float возвращает новую функцию -- среднее 
%		геометрическое данных
%		(количество исходных функций -- любое).
%	\item Создайте функцию, которая по данной функции $f:Float->Float$ и числу $x$ возвращает функцию, которая 
%		при каждом вызове последовательно возвращает $f(x)$, $f(f(x))$, $f(f(f(x)))$, $\dots$.







	
\end{enumerate}

\textbf{Задание №2}


Обозначим $n$ -- номер варианта.

Функция добавления элемента в список выбирается учащимся исходя из значения \verb|n % 4+1|

\begin{enumerate}
\item \verb|fun push (el: T): Bool|
вставляет элемент в начало списка;
\item \verb|fun add (el: T): Bool|
вставляет элемент в конец списка;
\item \verb|fun insert (el: T,n: Int): Bool|
вставляет элемент на позицию n (нумерация идет с единицы) с начала списка;
\item \verb|fun insert (els:Array<T>,count: Int,n: Int): Bool|
вставляет count элементов массива els начиная с позиции n
\end{enumerate}

Процедура удаления элемента из списка выбирается учащимся исходя из значения \verb|n/4%4+1|:

\begin{enumerate}
	\item \verb|fun delete(): Bool| удаляет элемент из начала списка
	\item \verb|fun delete(): Bool| удаляет элемент с конца списка
	\item \verb|fun delete (n: Int): Bool| удаляет элемент с позиции n
	\item \verb|fun delete (count: Int,n: Int): Bool| удаляет count элементов начиная с позиции n
\end{enumerate}

Процедура печати элементов списка выбирается учащимся исходя из значения \verb|n mod 5+1|:

\begin{enumerate}
	\item \verb|fun print():Unit| печатает все элементы
\item \verb|fun print():Unit| печатает первый элемент списка
\item \verb|fun print():Unit| печатает последний элемент списка
\item \verb|fun print(n: Int): Unit| печатает элемент списка, имеющий позицию n
\item \verb|fun print(count: Int,n: Int): Unit| печатает count элементов списка, начиная с позиции n
\end{enumerate}

Кроме того должна быть реализована функция eraseAll, которая очищает весь список.

Список должен быть реализован в виде generic-класса.





%\subsection{Практическая работа <<Dagger>> (2 часа)} 
Реализуйте DI в проекте, реализованном в последней практической работе, с использованием Dagger.



\section{Разработка на языке Kotlin для платформы Android}

\subsection{Семинар <<Создание мобильного приложения>> (2 часа)}

Повторите действия с мастер-класса, проведённого на паре:
\begin{enumerate}
	\item осуществите инсталляцию Android Studio (при необходимости);
	\item создайте проект;
	\item разработайте программу, осуществляющую сложение двух чисел;
	\item реализуйте корректную поддержку интернационализации и смены конфигурации;
	\item модифицируйте программу таким образом, чтобы ответ выводился на втором Activity.
\end{enumerate}

\subsection{Практическая работа <<Простейшие программы для Android>> (2 часа)}

Разработайте программу, работающую под управлением Android с использованием Views. 
Проверьте, что программа корректно работает с различными размерами экрана,
а также при повороте экрана. 

\begin{enumerate}
 \item Программа решения квадратного уравнения
 \item Программа решения неравенства вида $ax+b>0$
    \item Программа решения неравенства вида $ax+b<0$
 \item Программа решения неравенства вида $ax+b\geqslant 0$
 \item Программа решения неравенства вида $ax+b\leqslant 0$
\item Программа поиска дня недели по числу и месяцу в текущем году
% \item [ПП1-0-3] Программа поиска определителя матрицы $2\times2$
 \item  Программа перевода числа из $10$-ой в $16$-ую, $8$-ую и $2$-ую систем.
 \item  Программа поиска времени, когда окончится интервал. 
 Дано: часы и минуты начала интервала и количество минут, сколько он идет.
 Результат: часы и минуты окончания интервала.
% \item Программа поиска обратной матрицы для матрицы $2\times2$.
  \item Программа поиска обратной матрицы для матрицы $3\times3$.
 \item
 Программа поиска длины интервала. 
 Дано: часы и минуты начала интервала и часы и минуты конца интервала.
 Результат: количество минут в интервале.
 \item
 Программа умножения и деления двух комплексных чисел.
 
% \item[ПП1-0-9]
% Программа нахождения смешанного произведения трех векторов.
 
 \item
 Программа нахождения площади треугольника по координатам вершин.
 
 \item
 Программа нахождения углов треугольника по координатам вершин (проще всего это сделать по теореме косинусов).

  
 \item
 Программа перевода числа из $16$-ой, $8$-ой и $2$-ой системы в $10$-ую систему счисления.
 
 
 \item
 Программа нахождения количества денег на вкладе после окончания его срока по начальному взносу, проценту и срока в годах.
 
 
 \item
 Программа нахождения степени комплексного числа. Исходные данные: действительная, мнимая часть числа и степень. 
 Результат: действительная и мнимая часть результата.
 
 
 \item
 Программа умножения и деления чисел, представленных в виде обыкновенных дробей (состоящих из целой части, числителя и 
 знаменателя). Не забудьте выполнить сокращение дроби и приведение ее к правильному виду.

  \item
 Программа сложения и вычитания чисел, представленных в виде обыкновенных дробей (состоящих из целой части, числителя и 
 знаменателя). Не забудьте выполнить сокращение дроби и приведение ее к правильному виду.

 

 \item
 Программа определения по дате (число и месяц) знака зодиака.
 
 \item
Программа определения по обыкновенной дроби (числителю и знаменателю) периода десятичной дроби.
 
 \item
 Программа перевода комплесного числа из обычной формы в тригонометрическую и наоборот.

 
 \item
 Программа-игра Баше. При реализации этого задания не требуется ничего рисовать, вся информация вводится и выводится в виде
 чисел в обычные элементы управления.
 
 \item
 Программа разложения числа на простые множители.
 
 \item
 Программа нахождения наибольшего общего делителя и наименьшего общего кратного двух натуральных чисел.
 
 
 \item Программа-тест по предмету <<Разработка мобильных приложений>>. Создайте программу-тест из 10 вопросов с выбором
вариантов ответов и показом результатов прохождения теста.
\end{enumerate}


%\subsection{Практическая работа <<Jetpack Compose>> (2 часа)} 

Перепишите программу, реализованную в прошлой практической работе с использованием Jetpack Compose


\subsection{Практическая работа <<Фоновые вычисления>> (4 часа)}

В данной работе необходимо осуществить с использованием длинной арифметики достаточно долгое вычисление. 
Программа в ходе выполнения вычисления не должна <<зависать>>. 
Должна быть возможность остановить вычисление по желанию пользователя. 

Подсказка: в работе разрешено использовать \href{https://developer.android.com/reference/java/math/BigInteger}{BigInteger} (не возбраняется реализовать длинную арифметику <<руками>>).

Вычисление должно осуществляться внутри Service в отдельном потоке, после вычисления результаты должны появиться в Activity, а если он неактивен, то должно
появиться оповещение, кликнув по которому будет осуществлен переход на Activity с ответом.

\begin{enumerate}
	\item Реализуйте программу вычисления $n!$ со всеми десятичными знаками, где $n\in [1\dots 100000]$
	\item Реализуйте программу вычисления $2^n$ со всеми десятичными знаками, где $n\in [1\dots 1000000000]$
	\item Реализуйте программу вычисления $f_n$ со всеми десятичными знаками, где $n\in [1\dots 1000000000]$, где $f_n$ -- числа Фибоначчи, $f_1=f_2=1$.
	\item Реализуйте программу вычисления $n!!$ со всеми десятичными знаками, где $n\in [1\dots 100000]$
	\item Реализуйте программу вычисления $f_{2n}$ со всеми десятичными знаками, где $n\in [1\dots 1000000000]$, где $f_n$ -- числа Фибоначчи, $f_1=f_2=1$.
	\item Реализуйте программу вычисления $3^n$ со всеми десятичными знаками, где $n\in [1\dots 1000000000]$
	\item Реализуйте программу вычисления $f_1+f_2+\dots+f_n$ со всеми десятичными знаками, где $n\in [1\dots 10000000]$, где $f_n$ -- числа Фибоначчи, $f_1=f_2=1$.
	\item Реализуйте программу вычисления $1!+2!+3!+\dots+n!$ со всеми десятичными знаками, где $n\in [1\dots 10000]$
	\item Реализуйте программу вычисления $1!!+2!!+3!!+\dots+n!!$ со всеми десятичными знаками, где $n\in [1\dots 10000]$
	\item Реализуйте программу вычисления $f_2+f_4+\dots+f_{2n}$ со всеми десятичными знаками, где $n\in [1\dots 10000000]$, где $f_n$ -- числа Фибоначчи, $f_1=f_2=1$.
	\item Реализуйте программу вычисления $2^1+2^2+\dots+2^n$ со всеми десятичными знаками, где $n\in [1\dots 100000000]$
	\item Реализуйте программу вычисления $3^1+3^2+\dots+3^n$ со всеми десятичными знаками, где $n\in [1\dots 100000000]$
	\item Реализуйте программу вычисления $(n!)!$ со всеми десятичными знаками, где $n\in [1\dots 9]$, где $f_n$ -- числа Фибоначчи, $f_1=f_2=1$.

	\item Реализуйте программу вычисления $(n!)!!$ со всеми десятичными знаками, где $n\in [1\dots 9]$, где $f_n$ -- числа Фибоначчи, $f_1=f_2=1$.

	\item Реализуйте программу вычисления $(n!!)!$ со всеми десятичными знаками, где $n\in [1\dots 14]$, где $f_n$ -- числа Фибоначчи, $f_1=f_2=1$.
	\item Реализуйте программу вычисления $(n!!)!!$ со всеми десятичными знаками, где $n\in [1\dots 14]$, где $f_n$ -- числа Фибоначчи, $f_1=f_2=1$.


	\item Реализуйте программу вычисления $f_{n!}$ со всеми десятичными знаками, где $n\in [1\dots 13]$, где $f_n$ -- числа Фибоначчи, $f_1=f_2=1$.

	\item Реализуйте программу вычисления $f_{n!!}$ со всеми десятичными знаками, где $n\in [1\dots 20]$, где $f_n$ -- числа Фибоначчи, $f_1=f_2=1$.

	\item Реализуйте программу вычисления $f_{2^n}$ со всеми десятичными знаками, где $n\in [1\dots 30]$, где $f_n$ -- числа Фибоначчи, $f_1=f_2=1$.

	\item Реализуйте программу вычисления $f_{3^n}$ со всеми десятичными знаками, где $n\in [1\dots 20]$, где $f_n$ -- числа Фибоначчи, $f_1=f_2=1$.

	\item Реализуйте программу вычисления $(f_n)!$ со всеми десятичными знаками, где $n\in [1\dots 21]$, где $f_n$ -- числа Фибоначчи, $f_1=f_2=1$.

	\item Реализуйте программу вычисления $(f_n)!!$ со всеми десятичными знаками, где $n\in [1\dots 21]$, где $f_n$ -- числа Фибоначчи, $f_1=f_2=1$.

	\item Реализуйте программу вычисления $f_{f_n}$ со всеми десятичными знаками, где $n\in [1\dots 45]$, где $f_n$ -- числа Фибоначчи, $f_1=f_2=1$.
		
	\item Реализуйте программу вычисления $2^{f_n}$ со всеми десятичными знаками, где $n\in [1\dots 45]$, где $f_n$ -- числа Фибоначчи, $f_1=f_2=1$.

	\item Реализуйте программу вычисления $3^{f_n}$ со всеми десятичными знаками, где $n\in [1\dots 45]$, где $f_n$ -- числа Фибоначчи, $f_1=f_2=1$.

\end{enumerate}




%\subsection{Практическая работа <<Работа со списками>> (4 часа)}

Скопируйте часть Model из результата практической работы <<Особенности ООП в Kotlin>>, реализуйте интерфейс с тем же функционалом в Android (Jetpack Compose)
с использованием паттерна MVVM.


%\subsection{Практическая работа <<Content provider, ROOM, фотографии>> (6 часов)}

В данной работе модифицируется результат практической работы <<Dependency Injection>> (Android).

\begin{enumerate}
	\item внедрите сохранение результатов между сеансами работи с использованием Room;
	\item внедрите возможность осуществлять прикрепление фотографии к хранимым сущностям, фотографии сохраняйте в файловой системе, а ссылки на них -- в базе данных (посредством Room).
	\item дайте возможность работать другим приложениям с вашей базой данных (создайте Content Provider);
	\item реализуйте протип приложения, работающий с вашим Content Provider.
\end{enumerate}


%\subsection{Лабораторная работа <<Сенсоры>> (2 часа)}

Осуществите повтор выполнения мастер-класса, посвященного работе с сенсорами. Ход работы фиксируйте путем записи видео или регулярного сохранения копий экрана. Видео или копии экрана должны быть выполнены так, чтобы личность автора была очевидна и доказана.

Содержание:

\begin{itemize}
	\item использование датчика движения;
	\item использование датчика позиции;
	\item использование датчика освещённости.
\end{itemize}

%\subsection{Практическая работа <<Сенсоры>> (6 часов)}

Разработайте мобильное приложение, предназначенное для отслеживания активности пользователя согласно варианту. 

Информация о сохраненных активностях должна сохраняться между сеансами, пользователь должен иметь возможность корректировать результаты неточных измерений.

Ваша задача повысить точность измерения до возможного максимума.

Обратите внимание на использование всех изученных рекомендованных подходов к проектированию приложения.

Замечание: задание апробируется, потому некоторые варианты могут быть недостаточно корректными.

\begin{enumerate}
	\item учёта количества и глубины приседаний (в предположении, что телефон в руке, а рука поднимается во время приседа вверх);
	\item учёта количества отжиманий (в предположении, что телефон в кармане брюк);
	\item учёта количества и высоты прыжков на месте (телефон -- в кармане брюк);
	\item учёт длины прыжка в длину (телефон -- в кармане брюк);
	\item учёт длительности выполнения планки (телефон -- в кармане брюк, изначально человек стоит, а потом переходит в позу планки, в конце -- встаёт);
	\item измерение расстояния между точками (пользователь идет в одну точку, нажимает кнопку, идет в другую точку, нажимает кнопку);
	\item учёта количества шагов и скорости при беге на месте (в предположении, что телефон в кармане брюк);
	\item учёта количества подтягиваний (в предположении, что телефон в кармане брюк);
	\item учёта количества подъема туловища из положения лёжа (в предположении, что телефон - в кармане толстовки);
	\item учёта количества отжиманий от стены (в предположении, что телефон в кармане толстовки);
	\item учёта количества выпадов вперёд (в положении стоя, телефон -- в кармане брюк);
	\item учёта количества выпадов в сторону (в положении стоя, телефон -- в кармане брюк);
	\item учёта количества выпадов назад (в положении стоя, телефон -- в кармане брюк);
	\item учёта количества вращений обруча (телефон -- в кармане брюк);
	\item учёта количества и качества выполнений виньясы (телефон -- в кармане брюк);
	\item учёта количества прыжков через скакалку (телефон -- в кармане брюк);
	\item измерение глубины и прогресса наклона вперёд (пользователь держит телефон в руке);
	\item измерение качества выполнения мостика (телефон лежит на животе);
	\item измерение высоты вытяжения (телефон поднимается максимально высоко над головой);
	\item измерение количества и качества выполнения взмахов рук в противоположные стороны (одна рука вверх, другая вниз, телефон в руке);
	\item подсчёт времени проведенном в сидячем положении (телефон в кармане брюк);
	\item подсчёт времени пробегания 30 метров;
	\item подсчёт количества отжиманий, выполненных за 60 секунд;
	\item подсчёт количества приседаний, выполненных за 60 секунд;
	\item подсчёт количества подтягиваний, выполненных за 60 секунд.
\end{enumerate}



%\subsection{Практическая работа <<Анимация>> (2 часа)}

Доработайте приложение, выполненное в предыдущей практической, чтобы оно в виде анимации показывало то, как выполнялось упражнение или измерение пользователем (качество графики оцениваться не будет).


%\subsection{Лабораторная работа <<Профилирование>> (2 часа)}

Осуществите профилирование результата практической работы <<Content Provider, ROOM, фотографии>>: 

\begin{itemize}
	\item выявите <<узкие>> места,
	\item опишите их,
	\item опишите гипотезы, описывающие методики ускорения работы или снижения использования ресурсов памяти, для исследуемой программы.
\end{itemize}

%\subsection{Практическая работа <<Оптимизация>> (2 часа)}

Осуществите оптимизацию в соответствии с анализом, проведённым в последней лабораторной работе. Проверьте результаты оптимизации и проанализируйте их 
(объясните, почему она прошла успешно или потерпела неудачу).


\subsection{Семинар <<Работа с базами данных/навигация>> (4 часа)}

Создайте приложение, позволяющее осуществлять CRUD-операции с использованием пользовательского интерфейса. Постоянное хранение информации
осуществляйте с использованием библиотеки Room. Переход между <<экранами>> осуществляется с использованием библиотеки Navigation (предназначенной для использования с Jetpack Compose).

\begin{enumerate}
	\item База данных студентов группы. Поля: фамилия, имя, отчество, пол, возраст. 
	\item База данных расходов семьи. Поля: товар, стоимость, количество, дата.
	\item База данных загрузки аудиторий. Поля: дата и время, начала, дата и время конца, аудитория, преподаватель. 
	\item База данных учета доходов и расходов предпринимателя. Поля: дата, тип операции (доход/расход), объем операции, описание, 
корреспондент. 
	\item База данных велоклуба. Поля: ФИО, тип велосипеда (MTB и др.), стаж участия в велоклубе.
	\item База данных рейсов авиакомпании. Поля: дата и время вылета, аэропорт вылета, аэропорт прилета, дата и время прилета, 
марка самолета.
	\item База данных автобусных маршрутов. Поля: номер маршрута, номер парка, времена начала и окончания движения,
длина маршрута в км. 
	\item База данных электричек. Поля: вокзал, номер поезда, количество вагонов, тип (экспресс/обычный/спутник).
	\item База данных товаров Интернет-магазина. Поля: название товара, категория, цена товара, описание товара. 
	\item База заказов Интернет-магазина. Поля: ФИО заказчика, стоимость заказа, скидка (в процентах), адрес доставки. 
	\item База данных выборов. Поля: участок, кандидат, количество голосов.
	\item База данных практических работ. Поля: практическая работа, студент, номер варианта, номер уровня, 
дата сдачи, оценка. 
	\item База данных операторов и телеканалов. Поля: Название, тип (спутник, кабель, Интернет), охват (кол-во миллионов домохозяйств), минимальная
стоимость подписки. 
	\item База данных тарифных планов оператора. Поля: название, тип вещания (обычный/HD), флаг общедоступности. 
	\item База данных незаконно огороженных берегов. Поля: водный объект, регион, GPS-координаты, длина недоступного участка берега, дата фиксации нарушения.
	\item База данных временного прекращения движения в метро. Поля: дата и время начала прекращения
движения, дата и время окончания прекращения движения, станция, станция (от какой до какой станции прекращено движение).
	\item База данных проката фильмов. Поля: дата, время, кинотеатр, фильм, номер зала, тип сеанса (3D/Imax/обычный).
	\item База данных эвакуированных автомобилей. Поля: улица, автостоянка, GPS-координаты, 
тип нарушения (стоянка на проезжей части в месте запрета, стояна на тротуаре, стоянка на газоне), 
номер автомобиля, тип автомобиля (легковой/грузовой малой тонажности/грузовой большой
тонажности). 
	\item База данных средних специальных учебных учреждений. Поля: название, адрес, тип подчинения (федеральный/региональный), 
год основания, номер лицензии, номер аккредитации, дата окончания действия аккредитации. 
	\item База данных поселков. Поля: название, девелопер, площадь, количество жителей.
	\item База данных сухопутной военной техники. Поля: название, модель, разработчик, предприятие, стоимость, тип. 
	\item База данных деревьев в городе. Поля: GPS-координаты, вид дерева, округ, год посадки. 
	\item База данных футбольных матчей. Поля: дата, команда, команда, счет, место проведения. 
	\item База данных обращений жителей. Поля: дата, время, объект, заявитель, содержание обращения (до 255 символов), 
дата ответа, ответ на обращение (до 255 символов).
	\item База данных студентов колледжа. Поля: ФИО, группа, признак бюджетности, 
стипендия (нет/обычная/повышенная), флаг наличия социальной стипендии, дата рождения.
\end{enumerate}




\subsection{Практическая работа <<REST API>> (6 часов)}

Обеспечьте работу с минимум одним rest-api запросом (запрос должен выполняться в фоне), имеющем аргумент, 
возможности просмотра загруженной информации при отсуствии Интернет-соединения. Реализуйте автоматическое UNIT- и UI-тестирование. UI-тестирование
можно осуществлять для классов ROOM (в этом случае Unit-тесты надо перенести в папку AndroidTest). Применяйте изученные архитектурные подходы.

Квалификационный экзамен модуля является вариантом данной практической работы.

\begin{enumerate}
	\item Разработайте клиент \url{https://countrylayer.com} с сохранением загруженной информации на мобильном устройстве и загрузкой
		информации в отдельных потоках. Реализуйте автоматическое UNIT-тестирование и тестирование UI.

		Осуществите поиск стран по языку (для работы используйте бесплатный тарифный план).

	\item Разработайте клиент \url{https://countrylayer.com} с сохранением загруженной информации на мобильном устройстве и загрузкой
		информации в отдельных потоках. реализуйте автоматическое unit-тестирование и тестирование ui.

		Осуществите поиск стран по валюте (для работы используйте бесплатный тарифный план).

	\item Разработайте клиент \url{https://countrylayer.com} с сохранением загруженной информации на мобильном устройстве и загрузкой
		информации в отдельных потоках. реализуйте автоматическое unit-тестирование и тестирование ui.

		Осуществите поиск стран по региону (для работы используйте бесплатный тарифный план).

	\item Разработайте клиент \url{https://countrylayer.com} с сохранением загруженной информации на мобильном устройстве и загрузкой
		информации в отдельных потоках. реализуйте автоматическое unit-тестирование и тестирование ui.

		Осуществите поиск стран по региону (для работы используйте бесплатный тарифный план).

	\item Разработайте клиент \url{https://countrylayer.com} с сохранением загруженной информации на мобильном устройстве и загрузкой
		информации в отдельных потоках. реализуйте автоматическое unit-тестирование и тестирование ui.

		Осуществите поиск стран по региональному блоку (для работы используйте бесплатный тарифный план).

 	\item Разработайте клиент \url{https://github.com/astrocatalogs/OACAPI} с сохранением загруженной информации на мобильном устройстве и загрузкой
		информации в отдельных потоках. Реализуйте автоматическое UNIT-тестирование и тестирование UI.

		Осуществите поиск всех объектов на данном расстоянии в световых секундах от заданной точки.

  	\item Разработайте клиент \url{https://github.com/astrocatalogs/OACAPI} с сохранением загруженной информации на мобильном устройстве и загрузкой
		информации в отдельных потоках. Реализуйте автоматическое UNIT-тестирование и тестирование UI.

		Осуществите поиск всех объектов на данном расстоянии в световых секундах от заданной точки.

 	





%	\item Разработайте клиент \url{https://api.oceandrivers.com/static/docs.html#!/ODWeather} с сохранением загруженной информации на мобильном устройстве и загрузкой
%		информации в отдельных потоках. Реализуйте автоматическое UNIT-тестирование и тестирование UI. Реализуйте обращение к одной функции


	\item Разработайте клиент \url{https://api.met.no/weatherapi/airqualityforecast/0.1/documentation} с сохранением загруженной информации на мобильном устройстве и загрузкой
		информации в отдельных потоках. 

		Осуществите вывод информации о качестве воздуха по идентификатору станции (достаточно выводить часть информации в RecyclerView -- только
		по выбранным вами характеристикам воздуха).



%	\item Разработайте клиент \url{https://api.met.no/weatherapi/extremeswwc/1.2/documentation} с сохранением загруженной информации на мобильном устройстве и загрузкой
%		информации в отдельных потоках. 



%	\item Раiзработайте клиент \url{https://api.met.no/weatherapi/forestfireindex/1.1/documentation} с сохранением загруженной информации на мобильном устройстве и загрузкой

%		информации в отдельных потоках. 

	\item Разработайте клиент \url{https://api.met.no/weatherapi/locationforecast/2.0/documentation} с сохранением загруженной информации на мобильном устройстве и загрузкой
		информации в отдельных потоках. Место, по которому отображается информация, можно выбирать из фиксированного списка.

		По данным координатам выведите информацию о прогнозе погоды (можно выводить часть информации -- только по выбранным вами характеристикам
		погоды).



%	\item Разработайте клиент \url{https://api.met.no/weatherapi/metalerts/1.1/documentation} с сохранением загруженной информации на мобильном устройстве и загрузкой
%		информации в отдельных потоках. Место, по которому отображается информация, можно выбирать из фиксированного списка.

%	\item Разработайте клиент \url{https://api.met.no/weatherapi/nowcast/0.9/documentation} с сохранением загруженной информации на мобильном устройстве и загрузкой
%		информации в отдельных потоках. Место, по которому отображается информация, можно выбирать из фиксированного списка.
%	\item Разработайте клиент \url{https://api.met.no/weatherapi/oceanforecast/0.9/documentation} с сохранением загруженной информации на мобильном устройстве и загрузкой
%		информации в отдельных потоках. Место, по которому отображается информация, можно выбирать из фиксированного списка.
%	\item Разработайте клиент \url{https://api.met.no/weatherapi/probabilityforecast/1.1/documentation} с сохранением загруженной информации на мобильном устройстве и загрузкой
%		информации в отдельных потоках. Место, по которому отображается информация, можно выбирать из фиксированного списка.
%	\item Разработайте клиент \url{https://api.met.no/weatherapi/sunrise/2.0/documentation} с сохранением загруженной информации на мобильном устройстве и загрузкой
%		информации в отдельных потоках. Место, по которому отображается информация, можно выбирать из фиксированного списка.

%	\item Разработайте клиент \url{https://api.met.no/weatherapi/tidalwater/1.1/documentation} с сохранением загруженной информации на мобильном устройстве и загрузкой
%		информации в отдельных потоках. Место, по которому отображается информация, можно выбирать из фиксированного списка.

	\item Разработайте клиент \url{http://www.tvmaze.com/api} с сохранением загруженной информации на мобильном устройстве и загрузкой
		информации в отдельных потоках.

		По данной поисковой строке найдите информацию о шоу.

	\item Разработайте клиент \url{http://www.tvmaze.com/api} с сохранением загруженной информации на мобильном устройстве и загрузкой
		информации в отдельных потоках.

		По данной поисковой строке найдите информацию о героях шоу.

	\item Разработайте клиент \url{http://www.tvmaze.com/api} с сохранением загруженной информации на мобильном устройстве и загрузкой
		информации в отдельных потоках.

		По данной дате и стране найдите информацию о расписании шоу (SCHEDULE).

	\item Разработайте клиент \url{http://www.tvmaze.com/api} с сохранением загруженной информации на мобильном устройстве и загрузкой
		информации в отдельных потоках.

		По данной дате и стране найдите информацию о расписании шоу в стримах (SCHEDULE/WEB).






%	\item Разработайте клиент \url{https://mbta.com/developers/v3-api} с сохранением загруженной информации на мобильном устройстве и загрузкой
%		информации в отдельных потоках. 
	\item Разработайте клиент \url{https://date.nager.at/Api} с сохранением загруженной информации на мобильном устройстве и загрузкой
		информации в отдельных потоках. 

		По данному году и коду страны найдите информацию о выходных днях.


	\item Разработайте клиент \url{https://newton.now.sh/} с сохранением загруженной информации на мобильном устройстве и загрузкой
		информации в отдельных потоках. 

		По данному выражению найдите его упрощенный вариант. Все результаты сохраняйте и выводите в списке (начальное выражение -- результат).



%	\item Разработайте клиент \url{https://jobs.github.com/api} с сохранением загруженной информации на мобильном устройстве и загрузкой
%		информации в отдельных потоках. 
%	\item Разработайте клиент \url{https://github.com/workforce-data-initiative/skills-api/wiki/API-Overview} с сохранением загруженной информации на мобильном устройстве и загрузкой
%		информации в отдельных потоках. 
%	\item Разработайте клиент \url{https://lyricsovh.docs.apiary.io/#reference/0/lyrics-of-a-song/search} с сохранением загруженной информации на мобильном устройстве и загрузкой
%		информации в отдельных потоках. 
%	\item Разработайте клиент \url{https://pokeapi.co/} с сохранением загруженной информации на мобильном устройстве и загрузкой
%		информации в отдельных потоках. 
%	\item Разработайте клиент \url{https://github.com/15Dkatz/official\_joke\_api} с сохранением загруженной информации на мобильном устройстве и загрузкой
%		информации в отдельных потоках. 
	\item Разработайте клиент \url{https://docs.tronalddump.io/} с сохранением загруженной информации на мобильном устройстве и загрузкой
		информации в отдельных потоках. 

		По данному фрагменту текста найдите подходящие цитаты. Для указания header можно использовать @Header перед описанием функции.

%	\item Разработайте клиент \url{http://www.recipepuppy.com/about/api/} с сохранением загруженной информации на мобильном устройстве и загрузкой
%		информации в отдельных потоках. 
%	\item Разработайте клиент \url{https://verse.pawelad.xyz/} с сохранением загруженной информации на мобильном устройстве и загрузкой
%		информации в отдельных потоках. 
	\item Разработайте клиент \url{https://alexwohlbruck.github.io/cat-facts/} с сохранением загруженной информации на мобильном устройстве и загрузкой
		информации в отдельных потоках. 

		По данному классу животных найдите факты о них.

	\item Разработайте клиент \url{https://github.com/RocktimSaikia/anime-chan} с сохранением загруженной информации на мобильном устройстве и загрузкой
		информации в отдельных потоках. 

		По данному аниме выведите цитаты из него.


	\item Разработайте клиент \url{https://github.com/RocktimSaikia/anime-chan} с сохранением загруженной информации на мобильном устройстве и загрузкой
		информации в отдельных потоках. 

		По данному персонажу выведите его цитаты.

	\item Разработайте клиент \url{https://openlibrary.org} с сохранением загруженной информации на мобильном устройстве и загрузкой
		информации в отдельных потоках. 

		По данной теме выведите информацию о книгах (не обязательно всех).

	\item Разработайте клиент \url{https://openlibrary.org} с сохранением загруженной информации на мобильном устройстве и загрузкой
		информации в отдельных потоках. 

		По данному автору выведите информацию о произведениях.

	\item Разработайте клиент \url{https://openlibrary.org} с сохранением загруженной информации на мобильном устройстве и загрузкой
		информации в отдельных потоках. 

		По данному фрагменту названия выведите информацию о произведениях.

	\item Разработайте клиент \url{https://github.com/thundercomb/poetrydb#readme} с сохранением загруженной информации на мобильном устройстве и загрузкой
		информации в отдельных потоках. 

		По данному авторы выведите его произведения (начальные фрагменты).

	\item Разработайте клиент \url{https://github.com/fawazahmed0/currency-api#readme} с сохранением загруженной информации на мобильном устройстве и загрузкой
		информации в отдельных потоках. 

		По данной валюте выведите обменные курсы с ней.
		
	\item Разработайте клиент \url{https://https://nationalize.io/} с сохранением загруженной информации на мобильном устройстве и загрузкой
		информации в отдельных потоках. 

		По данному имени выводите гипотетические страны и вероятности совпадения страны.

	\item Разработайте клиент \url{https://www.boredapi.com} с сохранением загруженной информации на мобильном устройстве и загрузкой
		информации в отдельных потоках. 

		По данному типу и количеству участников выведите подходящее занятие. Предыдущие запросы и результаты сохраняйте и выводите в 
		RecyclerView.

	\item Разработайте клиент \url{https://github.com/davemachado/public-api} с сохранением загруженной информации на мобильном устройстве и загрузкой
		информации в отдельных потоках. 

		По данному фрагменту названия выведите информацию о публичных сервисах.



	\item Разработайте клиент \url{https://newton.now.sh/} с сохранением загруженной информации на мобильном устройстве и загрузкой
		информации в отдельных потоках. 

		По данному выражению найдите его разложение на множители. Все результаты сохраняйте и выводите в списке (начальное выражение -- результат).

	\item Разработайте клиент \url{https://newton.now.sh/} с сохранением загруженной информации на мобильном устройстве и загрузкой
		информации в отдельных потоках. 

		По данному выражению найдите его производную. Все результаты сохраняйте и выводите в списке (начальное выражение -- результат).

	\item Разработайте клиент \url{https://newton.now.sh/} с сохранением загруженной информации на мобильном устройстве и загрузкой
		информации в отдельных потоках. 

		По данному выражению найдите его интеграл. Все результаты сохраняйте и выводите в списке (начальное выражение -- результат).
























\end{enumerate}




\subsection{Семинар <<Автоматизация тестирования>> (2 часа)}

Разработайте автоматические тесты программы, разработанной на семинаре <<Простые мобильные приложения>>. 
С этой целью: а) выделите вычислительные функции в отдельный класс;
б) разработайте UNIT-тесты для данного класса; в) разработайте UI-тест для приложения.

Используйте возможности фреймворка при реализации данной работы



\subsection{Семинар <<Проектирование приложения с использованием MVVM>> (2 часа)}

Осуществите рефакторинг проекта, полученного на предыдущем семинаре, путём внедрения паттерна MVVM. Обратите внимание на то, что 
для реализации прослушивания изменений значений во ViewModel (и в Model при необходимости) необходимо использовать Flow. Используйте библиотеки Jetpack везде, где они применимы.


\subsection{Семинар <<Паттерн репозиторий>> (2 часа)}

Осуществите рефакторинг проекта, полученного на прошлом семинаре, посредством использования паттерна репозиторий.


\subsection{Практическая работа <<Dependency Injection>> (2 часа)}

Реализуйте использование Hilt для реализации Dependency Injection в результате прошлой практической работы.



\subsection{Последний семинар}

Последнее занятие предназначено для подведения итогов, осуществления review кода последнего проекта, который, в идеале, включает корректное использование паттерна
MVVM, репозитория и Dependency Injection в проекте, использующем Rest API и кеширование результатов на стороне клиента.

\section{Список литературы}

\begin{enumerate}
	\item Официальная документация по языку Kotlin: \url{https://kotlinlang.org/docs/reference/}
	\item Официальная документация по платформе Android: \url{https://developer.android.com/docs}
	\item Быстрое введение в Kotlin от авторов языка: \url{https://stepik.org/course/4222/promo}
\end{enumerate}

\end{document}

