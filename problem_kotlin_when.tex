\subsection{Семинар <<Условные конструкции>> (2 часа)}

Во второй работе предполагается освоение оператора when.

При выполнении заданий обращайте внимание на использование специфических особенностей языка везде, где это возможно: 
if и when могут быть как операторами, так и частью выражений; фигурные скобки во многих случаях можно опускать;
точки с запятой почти никогда не используются, корректно выбирайте, как помечать переменные: ключевым словом var или val.

При выполнении работ обеспечивайте \textbf{оптимальность} предлагаемой программы как по скорости, так и по памяти. В
случае противоречия между двумя критериями, выбирайте алгоритм, который обеспечивает лучшее быстродействие.

В частности, это обозначает, что нельзя использовать дополнительные строки (в заданиях, кроме как для ввода, строки не нужны),
следует избегать сложных структур (списков, множеств).

\begin{enumerate}
\item Даны год, месяц и число (в виде чисел) -- выведите год, месяц и число для следующего дня (месяц надо выводить в виде названия, например <<1 января 2000 года>>).
\item Даны год, месяц и число (в виде чисел) -- выведите год, месяц и число для предыдущего дня (месяц надо выводить в виде названия, например <<1 января 2000 года>>).
\item Дано число от 1 до 1000 -- выведите его на русском языке, например <<семьсот семьдесят семь>>.
\item Дано число от 1 до 1000 -- выведите его на английском языке, например <<seven hundred seventy-seven>>.

\item Даны год, месяц и число (в виде чисел) -- выведите год, месяц и число для дня, который будет через неделю (месяц надо выводить в виде названия, например <<1 января 2000 года>>).
\item Даны год, месяц и число (в виде чисел) -- выведите год, месяц и число для дня, который был неделю назад (месяц надо выводить в виде названия, например <<1 января 2000 года>>).

\end{enumerate}



