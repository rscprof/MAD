\section{Задание на разработку Android-клиента для туристического сервера}

Необходимо разработать Android-клиент для работы с сервером, предоставляющим REST API для получения информации о туристических местах. Клиент должен выполнять следующие задачи:

\subsection{Подключение к серверу}
Клиент должен взаимодействовать с сервером по адресу \texttt{http://10.0.2.2:9080/searchByPopularity}. Используйте библиотеку \texttt{Retrofit} для выполнения запросов к API. Пример запроса:

\begin{verbatim}
GET http://10.0.2.2:9080/searchByPopularity?min=80&max=95
\end{verbatim}

Ответ от сервера будет содержать данные о туристических местах в формате JSON:

\begin{verbatim}
[
  {
    "id": "1",
    "name": "Eiffel Tower",
    "country": "France",
    "description": "Iconic iron lattice tower in Paris.",
    "popularity": 98
  }
]
\end{verbatim}

\subsection{Локальное хранение данных}
Для хранения результатов запросов используйте \texttt{Room}. После получения данных с сервера они должны сохраняться в локальной базе данных для работы в офлайн-режиме.

\subsection{Архитектура приложения}
Используйте архитектуру MVVM для разделения бизнес-логики и UI. ViewModel будет отвечать за выполнение запросов через \texttt{Retrofit}, сохранение данных в \texttt{Room} и их передачу в UI.

\subsection{Интерфейс пользователя}
Используйте \texttt{Jetpack Compose} для создания интерфейса. Интерфейс должен отображать список туристических мест, полученных с сервера или из локальной базы данных, если интернет-соединение отсутствует.

\subsection{Рекомендации}
- При подключении к интернету данные должны обновляться.
- В случае отсутствия интернета — отображать данные из локальной базы данных.
- Соблюдать принципы архитектуры MVVM и использовать \texttt{Retrofit} и \texttt{Room} для работы с данными.
- Приложение должно отображать название, страну, описание и рейтинг популярности для каждого туристического места.
