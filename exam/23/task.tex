\section{Задание на разработку клиента для REST API сервера}

Необходимо разработать клиент для взаимодействия с сервером, предоставляющим REST API для получения данных о маршрутах автобусов. Клиент должен выполнять следующие задачи:

\subsection{Подключение к серверу}

Для выполнения запросов к API необходимо использовать библиотеку \texttt{Retrofit}. Пример запроса:

\begin{verbatim}
GET http://10.0.2.2:9080/routes?min_length=10&max_length=20
\end{verbatim}

Ответ от сервера возвращается в формате JSON и содержит данные о маршрутах автобусов:

\begin{verbatim}
[
  {
    "id": "1",
    "name": "Route A",
    "length": 12.5,
    "number_of_stops": 8
  }
]
\end{verbatim}

\subsection{Локальное хранение данных}
Для хранения данных о маршрутах автобусов необходимо использовать библиотеку \texttt{Room}. После получения данных с сервера они должны быть сохранены в локальной базе данных, чтобы быть доступными для работы в офлайн-режиме.

\subsection{Архитектура приложения}
Использовать архитектуру MVVM (Model-View-ViewModel) для разделения бизнес-логики и пользовательского интерфейса. Компонент \texttt{ViewModel} должен выполнять следующие функции:
\begin{itemize}
    \item Выполнение запросов к серверу через библиотеку \texttt{Retrofit}.
    \item Сохранение данных в локальной базе данных с использованием \texttt{Room}.
    \item Передача данных в интерфейс приложения.
\end{itemize}

\subsection{Интерфейс пользователя}
Использовать библиотеку \texttt{Jetpack Compose} для создания пользовательского интерфейса.

При отсутствии интернет-соединения необходимо отображать данные из локальной базы данных.
