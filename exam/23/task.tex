\section{Задание на разработку Android-клиента}

Необходимо разработать Android-клиент для работы с сервером, который предоставляет REST API для получения информации о продуктах. Клиент должен выполнять следующие задачи:

\subsection{Подключение к серверу}
Клиент должен взаимодействовать с сервером по адресу \texttt{http://10.0.2.2:9080/products}. Используйте библиотеку \texttt{Retrofit} для выполнения запросов к API. Пример запроса:

\begin{verbatim}
GET http://10.0.2.2:9080/products?min_shelf_life=10&max_shelf_life=30
\end{verbatim}

Ответ от сервера будет содержать данные о продуктах в формате JSON, например:

\begin{verbatim}
[
  {
    "id": "1",
    "name": "Яблоко",
    "price": 120.5,
    "shelf_life": 30,
    "is_organic": true
  }
]
\end{verbatim}

\subsection{Локальное хранение данных}
Для хранения результатов запросов используйте \texttt{Room}. После получения данных с сервера они должны сохраняться в локальной базе данных, чтобы обеспечить возможность работы в офлайн-режиме.

\subsection{Архитектура приложения}
Используйте архитектуру MVVM для разделения бизнес-логики и UI. \texttt{ViewModel} будет отвечать за выполнение запросов через \texttt{Retrofit}, сохранение данных в \texttt{Room} и их передачу в UI.

\subsection{Интерфейс пользователя}
Используйте \texttt{Jetpack Compose} для создания интерфейса. Интерфейс должен отображать список продуктов, полученных с сервера или из локальной базы данных, если интернет-соединение отсутствует. При запросе данных по диапазону срока годности, пользователь должен иметь возможность ввести минимальный и максимальный срок годности продуктов через текстовые поля в интерфейсе.

\subsection{Рекомендации}
\begin{itemize}
  \item При подключении к интернету данные должны обновляться.
  \item В случае отсутствия интернета — отображать данные из локальной базы данных.
  \item Соблюдать принципы архитектуры MVVM и использовать \texttt{Retrofit}, \texttt{Room} и \texttt{Jetpack Compose}.
  \item В запросах должны передаваться параметры \texttt{min_shelf_life} и \texttt{max_shelf_life}.
  \item При неправильных или отсутствующих параметрах отобразить ошибку.
\end{itemize}
