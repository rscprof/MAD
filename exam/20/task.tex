%\section{Задание на разработку Android-клиента для работы с сервером недвижимости с фильтрацией по году постройки}

Необходимо разработать Android-клиент для работы с сервером, который предоставляет REST API для получения информации о квартирах. Клиент должен выполнять следующие задачи:

%\subsection{Подключение к серверу}
Сервер доступен по адресу \texttt{http://10.0.2.2:9080/apartments}, и предоставляет данные о квартирах с параметрами для фильтрации по году постройки. Клиент должен использовать библиотеку \texttt{Retrofit} для выполнения запросов к серверу.

Пример запроса:
\begin{verbatim}
GET http://10.0.2.2:9080/apartments?min_year=2000&max_year=2020
\end{verbatim}

Ответ от сервера будет содержать данные о квартирах в формате JSON, например:

\begin{verbatim}
[
  {
    "id": "1",
    "location": "New York",
    "price": 500000,
    "area": 70,
    "floor": 10,
    "rooms": 2,
    "year_built": 2000
  },
]
\end{verbatim}

%\subsection{Локальное хранение данных}
Для хранения результатов запросов используйте \texttt{Room}. 
После получения данных с сервера, они должны сохраняться в локальной базе данных, 
чтобы обеспечить возможность работы в офлайн-режиме. 

%\subsection{Архитектура приложения}
Используйте архитектуру \texttt{MVVM} для разделения бизнес-логики и UI. \texttt{ViewModel} будет отвечать за выполнение запросов через \texttt{Retrofit}, сохранение данных в \texttt{Room} и их передачу в UI.

%\subsection{Интерфейс пользователя}
Используйте \texttt{Jetpack Compose} для создания интерфейса. Интерфейс должен отображать список квартир, полученных с сервера или из локальной базы данных, если интернет-соединение отсутствует. Также, пользователь должен иметь возможность вводить минимальный и максимальный год постройки через текстовые поля в интерфейсе.

%\subsection{Рекомендации}
\begin{itemize}
  \item В случае отсутствия интернета — отображать данные из локальной базы данных.
  \item Соблюдать принципы архитектуры \texttt{MVVM} и использовать \texttt{Retrofit}, \texttt{Room} и \texttt{Jetpack Compose}.
  \item В запросах должны передаваться параметры \texttt{min_year} и \texttt{max_year}.
  \item При отсутствии одного из параметров или наличии некорректных значений отобразить ошибку.
\end{itemize}
