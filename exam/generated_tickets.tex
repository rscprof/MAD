\documentclass[a4paper]{article}
\usepackage{xltxtra}

\usepackage{polyglossia}
\setmainlanguage{russian}

%\setkeys{russian}{babelshorthands=true}

\setmainfont{Times New Roman}
\setromanfont{Times New Roman} 
\setsansfont{Arial} 
\setmonofont{Courier New} 

\newfontfamily{\cyrillicfont}{Times New Roman} 
\newfontfamily{\cyrillicfontrm}{Times New Roman}
\newfontfamily{\cyrillicfonttt}{Courier New}
\newfontfamily{\cyrillicfontsf}{Arial}

\usepackage{array}
\usepackage{verbatim}
%\usepackage[utf8]{inputenc} % Кодировка
%\usepackage[russian]{babel} % Поддержка русского языка
%\usepackage{fontspec} % Для указания шрифтов (только XeLaTeX или LuaLaTeX)
%\setmainfont{Times New Roman} % Установка Times New Roman как основного шрифта
\usepackage{geometry} % Настройка полей страницы
\geometry{top=2cm, bottom=2cm, left=2.5cm, right=2.5cm}
%\renewcommand{\baselinestretch}{1.5} % Межстрочный интервал
\pagestyle{empty}
\begin{document}

\begin{tabular}{!{\vrule width 2pt}p{5cm}|p{6cm}|p{4cm}!{\vrule width 2pt}}
    \noalign{\hrule height 2pt}

    {\centering 
    \fontsize{14pt}{16pt}\selectfont
    РУТ(МИИТ)

\vspace{14pt}

Академия «Высшая инженерная школа»

\vspace{14pt}

2024/2025 учебный год

    }
&
{
    \centering
\fontsize{14pt}{16pt}\selectfont

\textbf{ЭКЗАМЕНАЦИОННЫЙ
БИЛЕТ №1}


по дисциплине 

«Разработка мобильных приложений» 
\fontsize{12pt}{14pt}\selectfont
для студентов образовательной программы «IT-сервисы и технологии обработки данных на транспорте»

}
&
{
\centering
\fontsize{14pt}{16pt}\selectfont

УТВЕРЖДАЮ
Руководитель образовательной программы

\vspace{1cm}

\fontsize{12pt}{14pt}\selectfont
\underline{\hspace{3cm}}

к.т.н., \underline{Проневич О.Б.}

}
\\
\hline
\multicolumn{3}{!{\vrule width 2pt}p{16cm}!{\vrule width 2pt}}{
\begin{minipage}{16cm}
    \vspace{0.2cm}

\fontsize{14pt}{16pt}\selectfont\itshape
\begin{enumerate}
    \item Язык Kotlin: коллекции
    \item Dependency Injection: библиотека Hilt, отличия от Dagger
    \item %\section{Задание на разработку клиента для REST API сервера}

Необходимо разработать клиент для взаимодействия с сервером, предоставляющим REST API для получения данных о музыкальных альбомах. Клиент должен выполнять следующие задачи:

%\subsection{Подключение к серверу}

Для выполнения запросов к API необходимо использовать библиотеку \texttt{Retrofit}. Пример запроса:


GET http://10.0.2.2:9080/search?title=blue


Ответ от сервера возвращается в формате JSON и содержит данные о музыкальных альбомах:


[
  \{
    "id": "1",
    "title": "Blue Train",
    "artist": "John Coltrane",
    "price": 56.99
  \}
]


%\subsection{Локальное хранение данных}
Для хранения данных о музыкальных альбомах необходимо использовать библиотеку \texttt{Room}. После получения данных с сервера они должны быть сохранены в локальной базе данных, чтобы быть доступными для работы в офлайн-режиме.

%\subsection{Архитектура приложения}
Использовать архитектуру MVVM (Model-View-ViewModel) для разделения бизнес-логики и пользовательского интерфейса. Компонент \texttt{ViewModel} должен выполнять следующие функции:
\begin{itemize}
    \item Выполнение запросов к серверу через библиотеку \texttt{Retrofit}.
    \item Сохранение данных в локальной базе данных с использованием \texttt{Room}.
    \item Передача данных в интерфейс приложения.
\end{itemize}

%\subsection{Интерфейс пользователя}
Использовать библиотеку \texttt{Jetpack Compose} для создания пользовательского интерфейса.

При отсутствии интернет-соединения необходимо отображать данные из локальной базы данных. 
\end{enumerate}

\vspace{0.2cm}
    
\end{minipage}
}
\\
\noalign{\hrule height 2pt}
\end{tabular}

\newpage


\begin{tabular}{!{\vrule width 2pt}p{5cm}|p{6cm}|p{4cm}!{\vrule width 2pt}}
    \noalign{\hrule height 2pt}

    {\centering 
    \fontsize{14pt}{16pt}\selectfont
    РУТ(МИИТ)

\vspace{14pt}

Академия «Высшая инженерная школа»

\vspace{14pt}

2024/2025 учебный год

    }
&
{
    \centering
\fontsize{14pt}{16pt}\selectfont

\textbf{ЭКЗАМЕНАЦИОННЫЙ
БИЛЕТ №2}


по дисциплине 

«Разработка мобильных приложений» 
\fontsize{12pt}{14pt}\selectfont
для студентов образовательной программы «IT-сервисы и технологии обработки данных на транспорте»

}
&
{
\centering
\fontsize{14pt}{16pt}\selectfont

УТВЕРЖДАЮ
Руководитель образовательной программы

\vspace{1cm}

\fontsize{12pt}{14pt}\selectfont
\underline{\hspace{3cm}}

к.т.н., \underline{Проневич О.Б.}

}
\\
\hline
\multicolumn{3}{!{\vrule width 2pt}p{16cm}!{\vrule width 2pt}}{
\begin{minipage}{16cm}
    \vspace{0.2cm}

\fontsize{14pt}{16pt}\selectfont\itshape
\begin{enumerate}
    \item Язык Kotlin: особенности языка
    \item Разработка приложений под Android: методика использования Gradle в Android-разработке
    \item %\section{Задание на разработку клиента для REST API сервера}

Необходимо разработать клиент для взаимодействия с сервером, предоставляющим REST API для получения данных о гражданах. Клиент должен выполнять следующие задачи:

%\subsection{Подключение к серверу}

Для выполнения запросов к API необходимо использовать библиотеку \texttt{Retrofit}. Пример запроса:


GET http://10.0.2.2:9080/search?name=john


Ответ от сервера возвращается в формате JSON и содержит данные о гражданах:


[
  \{
    "id": "1",
    "name": "John Smith",
    "age": 35,
    "address": "123 Main St"
  \}
]


%\subsection{Локальное хранение данных}
Для хранения данных о гражданах необходимо использовать библиотеку \texttt{Room}. После получения данных с сервера они должны быть сохранены в локальной базе данных, чтобы быть доступными для работы в офлайн-режиме.

%\subsection{Архитектура приложения}
Использовать архитектуру MVVM (Model-View-ViewModel) для разделения бизнес-логики и пользовательского интерфейса. Компонент \texttt{ViewModel} должен выполнять следующие функции:
\begin{itemize}
    \item Выполнение запросов к серверу через библиотеку \texttt{Retrofit}.
    \item Сохранение данных в локальной базе данных с использованием \texttt{Room}.
    \item Передача данных в интерфейс приложения.
\end{itemize}

%\subsection{Интерфейс пользователя}
Использовать библиотеку \texttt{Jetpack Compose} для создания пользовательского интерфейса.

При отсутствии интернет-соединения необходимо отображать данные из локальной базы данных. 
\end{enumerate}

\vspace{0.2cm}
    
\end{minipage}
}
\\
\noalign{\hrule height 2pt}
\end{tabular}

\newpage


\begin{tabular}{!{\vrule width 2pt}p{5cm}|p{6cm}|p{4cm}!{\vrule width 2pt}}
    \noalign{\hrule height 2pt}

    {\centering 
    \fontsize{14pt}{16pt}\selectfont
    РУТ(МИИТ)

\vspace{14pt}

Академия «Высшая инженерная школа»

\vspace{14pt}

2024/2025 учебный год

    }
&
{
    \centering
\fontsize{14pt}{16pt}\selectfont

\textbf{ЭКЗАМЕНАЦИОННЫЙ
БИЛЕТ №3}


по дисциплине 

«Разработка мобильных приложений» 
\fontsize{12pt}{14pt}\selectfont
для студентов образовательной программы «IT-сервисы и технологии обработки данных на транспорте»

}
&
{
\centering
\fontsize{14pt}{16pt}\selectfont

УТВЕРЖДАЮ
Руководитель образовательной программы

\vspace{1cm}

\fontsize{12pt}{14pt}\selectfont
\underline{\hspace{3cm}}

к.т.н., \underline{Проневич О.Б.}

}
\\
\hline
\multicolumn{3}{!{\vrule width 2pt}p{16cm}!{\vrule width 2pt}}{
\begin{minipage}{16cm}
    \vspace{0.2cm}

\fontsize{14pt}{16pt}\selectfont\itshape
\begin{enumerate}
    \item Язык Kotlin: способы создания sequence
    \item Разработка приложений под Android: роль и назначение Manifest
    \item %\section{Задание на разработку клиента для REST API сервера}

Необходимо разработать клиент для взаимодействия с сервером, предоставляющим REST API для получения данных о гражданах. Клиент должен выполнять следующие задачи:

%\subsection{Подключение к серверу}

Для выполнения запросов к API необходимо использовать библиотеку \texttt{Retrofit}. Пример запроса:


GET http://10.0.2.2:9080/search?minAge=25\&maxAge=35


Ответ от сервера возвращается в формате JSON и содержит данные о гражданах:


[
  \{
    "id": "1",
    "name": "John Smith",
    "age": 35,
    "address": "123 Main St"
  \}
]


%\subsection{Локальное хранение данных}
Для хранения данных о гражданах необходимо использовать библиотеку \texttt{Room}. После получения данных с сервера они должны быть сохранены в локальной базе данных, чтобы быть доступными для работы в офлайн-режиме.

%\subsection{Архитектура приложения}
Использовать архитектуру MVVM (Model-View-ViewModel) для разделения бизнес-логики и пользовательского интерфейса. Компонент \texttt{ViewModel} должен выполнять следующие функции:
\begin{itemize}
    \item Выполнение запросов к серверу через библиотеку \texttt{Retrofit}.
    \item Сохранение данных в локальной базе данных с использованием \texttt{Room}.
    \item Передача данных в интерфейс приложения.
\end{itemize}

%\subsection{Интерфейс пользователя}
Использовать библиотеку \texttt{Jetpack Compose} для создания пользовательского интерфейса.

При отсутствии интернет-соединения необходимо отображать данные из локальной базы данных. 
\end{enumerate}

\vspace{0.2cm}
    
\end{minipage}
}
\\
\noalign{\hrule height 2pt}
\end{tabular}

\newpage


\begin{tabular}{!{\vrule width 2pt}p{5cm}|p{6cm}|p{4cm}!{\vrule width 2pt}}
    \noalign{\hrule height 2pt}

    {\centering 
    \fontsize{14pt}{16pt}\selectfont
    РУТ(МИИТ)

\vspace{14pt}

Академия «Высшая инженерная школа»

\vspace{14pt}

2024/2025 учебный год

    }
&
{
    \centering
\fontsize{14pt}{16pt}\selectfont

\textbf{ЭКЗАМЕНАЦИОННЫЙ
БИЛЕТ №4}


по дисциплине 

«Разработка мобильных приложений» 
\fontsize{12pt}{14pt}\selectfont
для студентов образовательной программы «IT-сервисы и технологии обработки данных на транспорте»

}
&
{
\centering
\fontsize{14pt}{16pt}\selectfont

УТВЕРЖДАЮ
Руководитель образовательной программы

\vspace{1cm}

\fontsize{12pt}{14pt}\selectfont
\underline{\hspace{3cm}}

к.т.н., \underline{Проневич О.Б.}

}
\\
\hline
\multicolumn{3}{!{\vrule width 2pt}p{16cm}!{\vrule width 2pt}}{
\begin{minipage}{16cm}
    \vspace{0.2cm}

\fontsize{14pt}{16pt}\selectfont\itshape
\begin{enumerate}
    \item Язык Kotlin: массивы
    \item Разработка приложений под Android: понятие, назначение и способ использования оповещений
    \item %\section{Задание на разработку клиента для REST API сервера}

Необходимо разработать клиент для взаимодействия с сервером, предоставляющим REST API для получения данных об аниме. Клиент должен выполнять следующие задачи:

%\subsection{Подключение к серверу}

Для выполнения запросов к API необходимо использовать библиотеку \texttt{Retrofit}. Пример запроса:


GET http://10.0.2.2:9080/search?genre=action


Ответ от сервера возвращается в формате JSON и содержит данные об аниме:


[
  \{
    "id": "1",
    "title": "Naruto",
    "genre": "Action, Adventure",
    "episodes": 220
  \}
]


%\subsection{Локальное хранение данных}
Для хранения данных об аниме необходимо использовать библиотеку \texttt{Room}. После получения данных с сервера они должны быть сохранены в локальной базе данных, чтобы быть доступными для работы в офлайн-режиме.

%\subsection{Архитектура приложения}
Использовать архитектуру MVVM (Model-View-ViewModel) для разделения бизнес-логики и пользовательского интерфейса. Компонент \texttt{ViewModel} должен выполнять следующие функции:
\begin{itemize}
    \item Выполнение запросов к серверу через библиотеку \texttt{Retrofit}.
    \item Сохранение данных в локальной базе данных с использованием \texttt{Room}.
    \item Передача данных в интерфейс приложения.
\end{itemize}

%\subsection{Интерфейс пользователя}
Использовать библиотеку \texttt{Jetpack Compose} для создания пользовательского интерфейса.

При отсутствии интернет-соединения необходимо отображать данные из локальной базы данных. 
\end{enumerate}

\vspace{0.2cm}
    
\end{minipage}
}
\\
\noalign{\hrule height 2pt}
\end{tabular}

\newpage


\begin{tabular}{!{\vrule width 2pt}p{5cm}|p{6cm}|p{4cm}!{\vrule width 2pt}}
    \noalign{\hrule height 2pt}

    {\centering 
    \fontsize{14pt}{16pt}\selectfont
    РУТ(МИИТ)

\vspace{14pt}

Академия «Высшая инженерная школа»

\vspace{14pt}

2024/2025 учебный год

    }
&
{
    \centering
\fontsize{14pt}{16pt}\selectfont

\textbf{ЭКЗАМЕНАЦИОННЫЙ
БИЛЕТ №5}


по дисциплине 

«Разработка мобильных приложений» 
\fontsize{12pt}{14pt}\selectfont
для студентов образовательной программы «IT-сервисы и технологии обработки данных на транспорте»

}
&
{
\centering
\fontsize{14pt}{16pt}\selectfont

УТВЕРЖДАЮ
Руководитель образовательной программы

\vspace{1cm}

\fontsize{12pt}{14pt}\selectfont
\underline{\hspace{3cm}}

к.т.н., \underline{Проневич О.Б.}

}
\\
\hline
\multicolumn{3}{!{\vrule width 2pt}p{16cm}!{\vrule width 2pt}}{
\begin{minipage}{16cm}
    \vspace{0.2cm}

\fontsize{14pt}{16pt}\selectfont\itshape
\begin{enumerate}
    \item Язык Kotlin: концепции и элементы языка для поддержки null-безопасности
    \item Разработка приложений под Android: основные принципы разработки с использованием Jetpack compose
    \item %\section{Задание на разработку клиента для REST API сервера}

Необходимо разработать клиент для взаимодействия с сервером, предоставляющим REST API для получения данных об аниме. Клиент должен выполнять следующие задачи:

%\subsection{Подключение к серверу}

Для выполнения запросов к API необходимо использовать библиотеку \texttt{Retrofit}. Пример запроса:


GET http://10.0.2.2:9080/search?minEpisodes=20\&maxEpisodes=50


Ответ от сервера возвращается в формате JSON и содержит данные об аниме:


[
  \{
    "id": "1",
    "title": "Naruto",
    "genre": "Action, Adventure",
    "episodes": 220
  \}
]


%\subsection{Локальное хранение данных}
Для хранения данных об аниме необходимо использовать библиотеку \texttt{Room}. После получения данных с сервера они должны быть сохранены в локальной базе данных, чтобы быть доступными для работы в офлайн-режиме.

%\subsection{Архитектура приложения}
Использовать архитектуру MVVM (Model-View-ViewModel) для разделения бизнес-логики и пользовательского интерфейса. Компонент \texttt{ViewModel} должен выполнять следующие функции:
\begin{itemize}
    \item Выполнение запросов к серверу через библиотеку \texttt{Retrofit}.
    \item Сохранение данных в локальной базе данных с использованием \texttt{Room}.
    \item Передача данных в интерфейс приложения.
\end{itemize}

%\subsection{Интерфейс пользователя}
Использовать библиотеку \texttt{Jetpack Compose} для создания пользовательского интерфейса.


При отсутствии интернет-соединения необходимо отображать данные из локальной базы данных. 
\end{enumerate}

\vspace{0.2cm}
    
\end{minipage}
}
\\
\noalign{\hrule height 2pt}
\end{tabular}

\newpage


\begin{tabular}{!{\vrule width 2pt}p{5cm}|p{6cm}|p{4cm}!{\vrule width 2pt}}
    \noalign{\hrule height 2pt}

    {\centering 
    \fontsize{14pt}{16pt}\selectfont
    РУТ(МИИТ)

\vspace{14pt}

Академия «Высшая инженерная школа»

\vspace{14pt}

2024/2025 учебный год

    }
&
{
    \centering
\fontsize{14pt}{16pt}\selectfont

\textbf{ЭКЗАМЕНАЦИОННЫЙ
БИЛЕТ №6}


по дисциплине 

«Разработка мобильных приложений» 
\fontsize{12pt}{14pt}\selectfont
для студентов образовательной программы «IT-сервисы и технологии обработки данных на транспорте»

}
&
{
\centering
\fontsize{14pt}{16pt}\selectfont

УТВЕРЖДАЮ
Руководитель образовательной программы

\vspace{1cm}

\fontsize{12pt}{14pt}\selectfont
\underline{\hspace{3cm}}

к.т.н., \underline{Проневич О.Б.}

}
\\
\hline
\multicolumn{3}{!{\vrule width 2pt}p{16cm}!{\vrule width 2pt}}{
\begin{minipage}{16cm}
    \vspace{0.2cm}

\fontsize{14pt}{16pt}\selectfont\itshape
\begin{enumerate}
    \item Язык Kotlin: особенности реализации объектно-ориентированной парадигмы в наследовании
    \item Разработка приложений под Android: библиотека Retrofit
    \item %\section{Задание на разработку Android-клиента}

Необходимо разработать Android-клиент для работы с сервером, предоставляющим REST API для поиска информации об аниме. Клиент должен выполнять следующие задачи:

%\subsection{Подключение к серверу} 

Клиент должен взаимодействовать с сервером по адресу 

\texttt{http://10.0.2.2:9080/search}. Используйте библиотеку \texttt{Retrofit} для выполнения запросов к API. Пример запроса:

 GET http://10.0.2.2:9080/search?title=Naruto 

Ответ от сервера будет содержать данные об аниме в формате JSON:

 [ \{ "id": "1", "title": "Naruto", 
    "genre": "Action, Adventure", "episodes": 220 \}] 

%\subsection{Локальное хранение данных} 

Для хранения результатов запросов используйте \texttt{Room}. После получения данных с сервера они должны сохраняться в локальной базе данных для работы в офлайн-режиме.

%\subsection{Архитектура приложения} 

Используйте архитектуру MVVM для разделения бизнес-логики и UI. ViewModel будет отвечать за выполнение запросов через \texttt{Retrofit}, сохранение данных в \texttt{Room} и их передачу в UI.

%\subsection{Интерфейс пользователя} 

Используйте \texttt{Jetpack Compose} для создания интерфейса. Интерфейс должен предоставлять следующие возможности:

Поле для ввода текста, позволяющее пользователю ввести название аниме для поиска.
Список аниме, соответствующих запросу, отображаемый с названием, жанром и количеством эпизодов.
Сообщение об отсутствии результатов, если аниме не найдено.
%\subsection{Функциональные требования}

При подключении к интернету данные должны обновляться и отображаться в реальном времени.
В случае отсутствия интернет-соединения данные должны браться из локальной базы данных.
%\subsection{Рекомендации} 
\end{enumerate}

\vspace{0.2cm}
    
\end{minipage}
}
\\
\noalign{\hrule height 2pt}
\end{tabular}

\newpage


\begin{tabular}{!{\vrule width 2pt}p{5cm}|p{6cm}|p{4cm}!{\vrule width 2pt}}
    \noalign{\hrule height 2pt}

    {\centering 
    \fontsize{14pt}{16pt}\selectfont
    РУТ(МИИТ)

\vspace{14pt}

Академия «Высшая инженерная школа»

\vspace{14pt}

2024/2025 учебный год

    }
&
{
    \centering
\fontsize{14pt}{16pt}\selectfont

\textbf{ЭКЗАМЕНАЦИОННЫЙ
БИЛЕТ №7}


по дисциплине 

«Разработка мобильных приложений» 
\fontsize{12pt}{14pt}\selectfont
для студентов образовательной программы «IT-сервисы и технологии обработки данных на транспорте»

}
&
{
\centering
\fontsize{14pt}{16pt}\selectfont

УТВЕРЖДАЮ
Руководитель образовательной программы

\vspace{1cm}

\fontsize{12pt}{14pt}\selectfont
\underline{\hspace{3cm}}

к.т.н., \underline{Проневич О.Б.}

}
\\
\hline
\multicolumn{3}{!{\vrule width 2pt}p{16cm}!{\vrule width 2pt}}{
\begin{minipage}{16cm}
    \vspace{0.2cm}

\fontsize{14pt}{16pt}\selectfont\itshape
\begin{enumerate}
    \item Язык Kotlin: StateFlow и SharedFlow
    \item Разработка приложений под Android: автоматическое UI-тестирование
    \item %\section{Задание на разработку Android-клиента для поиска туристических мест}

Необходимо разработать Android-клиент для взаимодействия с сервером, предоставляющим REST API для поиска туристических мест. Сервер доступен по адресу \texttt{http://10.0.2.2:9080/search}. Приложение должно выполнять следующие задачи:

%\subsection{Подключение к серверу} 

Клиент должен отправлять запросы к API сервера для поиска мест по имени. Используйте библиотеку \texttt{Retrofit} для выполнения запросов. Пример запроса:

 GET http://10.0.2.2:9080/search?name=eiffel 

Ответ от сервера будет содержать данные в формате JSON:

 [ \{ "id": "1", "name": "Eiffel Tower", 
    "country": "France", 
    "description": "Iconic iron lattice tower in Paris.", 
    "popularity": 98 \} ] 

%\subsection{Локальное хранение данных} 

Используйте библиотеку \texttt{Room} для сохранения данных о туристических местах в локальную базу данных. Это позволит отображать сохранённые результаты в офлайн-режиме.

%\subsection{Архитектура приложения} 

Организуйте архитектуру приложения по шаблону MVVM:

ViewModel отвечает за выполнение запросов через \texttt{Retrofit} и работу с \texttt{Room}.
UI получает данные из ViewModel.
%\subsection{Интерфейс пользователя} 

Создайте интерфейс с помощью \texttt{Jetpack Compose}, который включает:

Поле для ввода имени места для поиска.
Кнопку "Найти".
Список найденных мест, отображающий имя, страну и описание.
%\subsection{Функциональные требования}

При вводе имени места приложение отправляет запрос к серверу и отображает список найденных мест.
Результаты поиска автоматически сохраняются в локальной базе данных.
Если подключение к серверу недоступно, приложение отображает данные из локальной базы данных.
%\subsection{Рекомендации} 
\end{enumerate}

\vspace{0.2cm}
    
\end{minipage}
}
\\
\noalign{\hrule height 2pt}
\end{tabular}

\newpage


\begin{tabular}{!{\vrule width 2pt}p{5cm}|p{6cm}|p{4cm}!{\vrule width 2pt}}
    \noalign{\hrule height 2pt}

    {\centering 
    \fontsize{14pt}{16pt}\selectfont
    РУТ(МИИТ)

\vspace{14pt}

Академия «Высшая инженерная школа»

\vspace{14pt}

2024/2025 учебный год

    }
&
{
    \centering
\fontsize{14pt}{16pt}\selectfont

\textbf{ЭКЗАМЕНАЦИОННЫЙ
БИЛЕТ №8}


по дисциплине 

«Разработка мобильных приложений» 
\fontsize{12pt}{14pt}\selectfont
для студентов образовательной программы «IT-сервисы и технологии обработки данных на транспорте»

}
&
{
\centering
\fontsize{14pt}{16pt}\selectfont

УТВЕРЖДАЮ
Руководитель образовательной программы

\vspace{1cm}

\fontsize{12pt}{14pt}\selectfont
\underline{\hspace{3cm}}

к.т.н., \underline{Проневич О.Б.}

}
\\
\hline
\multicolumn{3}{!{\vrule width 2pt}p{16cm}!{\vrule width 2pt}}{
\begin{minipage}{16cm}
    \vspace{0.2cm}

\fontsize{14pt}{16pt}\selectfont\itshape
\begin{enumerate}
    \item Язык Kotlin: синтаксис и семантика создания lambda-функций
    \item Разработка приложений под Android: методика создания приложений
    \item %\section{Задание на разработку Android-клиента для туристического сервера}

Необходимо разработать Android-клиент для работы с сервером, предоставляющим REST API для получения информации о туристических местах. Клиент должен выполнять следующие задачи:

%\subsection{Подключение к серверу} 

Клиент должен взаимодействовать с сервером по адресу 

\texttt{http://10.0.2.2:9080/searchByCountry}. Используйте библиотеку \texttt{Retrofit} для выполнения запросов к API. Пример запроса:

 GET http://10.0.2.2:9080/searchByCountry?country=France 

Ответ от сервера будет содержать данные о туристических местах в формате JSON:

 \{ "places": [ \{ "id": "1", "name": "Eiffel Tower", 
    "country": "France", 
    "description": "Iconic iron lattice tower in Paris.", 
    "popularity": 98 \} ] \} 

%\subsection{Локальное хранение данных} 

Для хранения результатов запросов используйте \texttt{Room}. После получения данных с сервера они должны сохраняться в локальной базе данных для работы в офлайн-режиме.

%\subsection{Архитектура приложения} 

Используйте архитектуру MVVM для разделения бизнес-логики и UI. ViewModel будет отвечать за выполнение запросов через \texttt{Retrofit}, сохранение данных в \texttt{Room} и их передачу в UI.

%\subsection{Интерфейс пользователя} 

Используйте \texttt{Jetpack Compose} для создания интерфейса. Интерфейс должен отображать список туристических мест, полученных с сервера или из локальной базы данных, если интернет-соединение отсутствует.

%\subsection{Рекомендации}

При подключении к интернету данные должны обновляться.
В случае отсутствия интернета — отображать данные из локальной базы данных.
Соблюдать принципы архитектуры MVVM и использовать \texttt{Retrofit} и \texttt{Room} для работы с данными.
Приложение должно отображать название, страну, описание и рейтинг популярности для каждого туристического места. 
\end{enumerate}

\vspace{0.2cm}
    
\end{minipage}
}
\\
\noalign{\hrule height 2pt}
\end{tabular}

\newpage


\begin{tabular}{!{\vrule width 2pt}p{5cm}|p{6cm}|p{4cm}!{\vrule width 2pt}}
    \noalign{\hrule height 2pt}

    {\centering 
    \fontsize{14pt}{16pt}\selectfont
    РУТ(МИИТ)

\vspace{14pt}

Академия «Высшая инженерная школа»

\vspace{14pt}

2024/2025 учебный год

    }
&
{
    \centering
\fontsize{14pt}{16pt}\selectfont

\textbf{ЭКЗАМЕНАЦИОННЫЙ
БИЛЕТ №9}


по дисциплине 

«Разработка мобильных приложений» 
\fontsize{12pt}{14pt}\selectfont
для студентов образовательной программы «IT-сервисы и технологии обработки данных на транспорте»

}
&
{
\centering
\fontsize{14pt}{16pt}\selectfont

УТВЕРЖДАЮ
Руководитель образовательной программы

\vspace{1cm}

\fontsize{12pt}{14pt}\selectfont
\underline{\hspace{3cm}}

к.т.н., \underline{Проневич О.Б.}

}
\\
\hline
\multicolumn{3}{!{\vrule width 2pt}p{16cm}!{\vrule width 2pt}}{
\begin{minipage}{16cm}
    \vspace{0.2cm}

\fontsize{14pt}{16pt}\selectfont\itshape
\begin{enumerate}
    \item Язык Kotlin: особенности реализации объектно-ориентированной парадигмы в структуре класса
    \item Архитектурный шаблон MVVM: описание и структура программы на примере Android-приложения
    \item %\section{Задание на разработку Android-клиента для туристического сервера}

Необходимо разработать Android-клиент для работы с сервером, предоставляющим REST API для получения информации о туристических местах. Клиент должен выполнять следующие задачи:

%\subsection{Подключение к серверу}
Клиент должен взаимодействовать с сервером по адресу 

\texttt{http://10.0.2.2:9080/searchByPopularity}. Используйте библиотеку \texttt{Retrofit} для выполнения запросов к API. Пример запроса:


GET http://10.0.2.2:9080/searchByPopularity?min=80\&max=95


Ответ от сервера будет содержать данные о туристических местах в формате JSON:


[
  \{
    "id": "1",
    "name": "Eiffel Tower",
    "country": "France",
    "description": "Iconic iron lattice tower in Paris.",
    "popularity": 98
  \}
]


%\subsection{Локальное хранение данных}
Для хранения результатов запросов используйте \texttt{Room}. После получения данных с сервера они должны сохраняться в локальной базе данных для работы в офлайн-режиме.

%\subsection{Архитектура приложения}
Используйте архитектуру MVVM для разделения бизнес-логики и UI. ViewModel будет отвечать за выполнение запросов через \texttt{Retrofit}, сохранение данных в \texttt{Room} и их передачу в UI.

%\subsection{Интерфейс пользователя}
Используйте \texttt{Jetpack Compose} для создания интерфейса. Интерфейс должен отображать список туристических мест, полученных с сервера или из локальной базы данных, если интернет-соединение отсутствует.

%\subsection{Рекомендации}
При подключении к интернету данные должны обновляться.
В случае отсутствия интернета — отображать данные из локальной базы данных.
Соблюдать принципы архитектуры MVVM и использовать \texttt{Retrofit} и \texttt{Room} для работы с данными.
Приложение должно отображать название, страну, описание и рейтинг популярности для каждого туристического места. 
\end{enumerate}

\vspace{0.2cm}
    
\end{minipage}
}
\\
\noalign{\hrule height 2pt}
\end{tabular}

\newpage


\begin{tabular}{!{\vrule width 2pt}p{5cm}|p{6cm}|p{4cm}!{\vrule width 2pt}}
    \noalign{\hrule height 2pt}

    {\centering 
    \fontsize{14pt}{16pt}\selectfont
    РУТ(МИИТ)

\vspace{14pt}

Академия «Высшая инженерная школа»

\vspace{14pt}

2024/2025 учебный год

    }
&
{
    \centering
\fontsize{14pt}{16pt}\selectfont

\textbf{ЭКЗАМЕНАЦИОННЫЙ
БИЛЕТ №10}


по дисциплине 

«Разработка мобильных приложений» 
\fontsize{12pt}{14pt}\selectfont
для студентов образовательной программы «IT-сервисы и технологии обработки данных на транспорте»

}
&
{
\centering
\fontsize{14pt}{16pt}\selectfont

УТВЕРЖДАЮ
Руководитель образовательной программы

\vspace{1cm}

\fontsize{12pt}{14pt}\selectfont
\underline{\hspace{3cm}}

к.т.н., \underline{Проневич О.Б.}

}
\\
\hline
\multicolumn{3}{!{\vrule width 2pt}p{16cm}!{\vrule width 2pt}}{
\begin{minipage}{16cm}
    \vspace{0.2cm}

\fontsize{14pt}{16pt}\selectfont\itshape
\begin{enumerate}
    \item Язык Kotlin: функции, предназначенные для агрегирования коллекций
    \item Dependency Injection: понятие, библиотека Dagger
    \item %\section{Задание на разработку мобильного клиента для поиска книг по году издания}

Необходимо разработать мобильный клиент для работы с сервером, предоставляющим API для поиска книг по диапазону годов издания. Клиент должен выполнять следующие задачи:

%\subsection{Подключение к серверу} 

Клиент должен взаимодействовать с сервером по адресу 

\texttt{http://localhost:9090/search}. Для выполнения запросов к API использовать стандартные HTTP-запросы. Пример запроса:

 GET http://localhost:9090/search?start=1900\&end=2000 

Ответ от сервера будет содержать список книг в формате JSON:

 [ \{ "id": "1", "title": "1984", "author": "George Orwell", 
    "year": 1949 \}] 

%\subsection{Локальное хранение данных} 

Для хранения результатов запросов необходимо использовать SQLite или другую подходящую локальную базу данных. После получения данных с сервера, их нужно сохранять в базе данных для дальнейшей работы в офлайн-режиме.

%\subsection{Архитектура приложения} 

Рекомендуется использовать архитектуру MVVM для разделения бизнес-логики и интерфейса пользователя. ViewModel будет отвечать за выполнение запросов к API, обработку данных и передачу их в UI.

%\subsection{Интерфейс пользователя} 

Интерфейс должен отображать список книг, полученных с сервера или из локальной базы данных, если нет интернет-соединения. Для отображения данных использовать стандартные компоненты интерфейса мобильной платформы.

%\subsection{Рекомендации}

В случае отсутствия интернета отображать данные из локальной базы данных.
Соблюдать принципы архитектуры MVVM.
Использовать библиотеки Retrofit и Room. 
\end{enumerate}

\vspace{0.2cm}
    
\end{minipage}
}
\\
\noalign{\hrule height 2pt}
\end{tabular}

\newpage


\begin{tabular}{!{\vrule width 2pt}p{5cm}|p{6cm}|p{4cm}!{\vrule width 2pt}}
    \noalign{\hrule height 2pt}

    {\centering 
    \fontsize{14pt}{16pt}\selectfont
    РУТ(МИИТ)

\vspace{14pt}

Академия «Высшая инженерная школа»

\vspace{14pt}

2024/2025 учебный год

    }
&
{
    \centering
\fontsize{14pt}{16pt}\selectfont

\textbf{ЭКЗАМЕНАЦИОННЫЙ
БИЛЕТ №11}


по дисциплине 

«Разработка мобильных приложений» 
\fontsize{12pt}{14pt}\selectfont
для студентов образовательной программы «IT-сервисы и технологии обработки данных на транспорте»

}
&
{
\centering
\fontsize{14pt}{16pt}\selectfont

УТВЕРЖДАЮ
Руководитель образовательной программы

\vspace{1cm}

\fontsize{12pt}{14pt}\selectfont
\underline{\hspace{3cm}}

к.т.н., \underline{Проневич О.Б.}

}
\\
\hline
\multicolumn{3}{!{\vrule width 2pt}p{16cm}!{\vrule width 2pt}}{
\begin{minipage}{16cm}
    \vspace{0.2cm}

\fontsize{14pt}{16pt}\selectfont\itshape
\begin{enumerate}
    \item Язык Kotlin: делегирование классов
    \item Разработка приложений под Android: компоненты Jetpack compose, предназначенные для вёрстки
    \item %\section{Задание на разработку мобильного клиента}

Необходимо разработать мобильный клиент для работы с сервером, предоставляющим API для поиска книг по названию. Клиент должен выполнять следующие задачи:

%\subsection{Подключение к серверу} 

Клиент должен взаимодействовать с сервером по адресу 

\texttt{http://localhost:8080/books/search}. Используйте стандартные HTTP-запросы для выполнения запросов к API. Пример запроса:

 GET http://localhost:8080/books/search?title=1984 

Ответ от сервера будет содержать данные о книгах в формате JSON:

 [ \{ "id": "1", "title": "1984", "author": "George Orwell",
     "year": 1949 \} ] 

%\subsection{Локальное хранение данных} 

Для хранения результатов запросов используйте SQLite или другую подходящую локальную базу данных. После получения данных с сервера они должны сохраняться в локальной базе данных для работы в офлайн-режиме.

%\subsection{Архитектура приложения} 

Используйте архитектуру MVVM для разделения бизнес-логики и UI. ViewModel будет отвечать за выполнение запросов к API, сохранение данных в базу данных и их передачу в UI.

%\subsection{Интерфейс пользователя} 

Используйте нативные компоненты интерфейса для создания интерфейса. Интерфейс должен отображать список книг, полученных с сервера или из локальной базы данных в случае отсутствия интернет-соединения.

%\subsection{Рекомендации}

В случае отсутствия интернета — отображать данные из локальной базы данных.
Соблюдать принципы архитектуры MVVM и использовать подходящие библиотеки для HTTP-запросов и локального хранения данных. 
\end{enumerate}

\vspace{0.2cm}
    
\end{minipage}
}
\\
\noalign{\hrule height 2pt}
\end{tabular}

\newpage


\begin{tabular}{!{\vrule width 2pt}p{5cm}|p{6cm}|p{4cm}!{\vrule width 2pt}}
    \noalign{\hrule height 2pt}

    {\centering 
    \fontsize{14pt}{16pt}\selectfont
    РУТ(МИИТ)

\vspace{14pt}

Академия «Высшая инженерная школа»

\vspace{14pt}

2024/2025 учебный год

    }
&
{
    \centering
\fontsize{14pt}{16pt}\selectfont

\textbf{ЭКЗАМЕНАЦИОННЫЙ
БИЛЕТ №12}


по дисциплине 

«Разработка мобильных приложений» 
\fontsize{12pt}{14pt}\selectfont
для студентов образовательной программы «IT-сервисы и технологии обработки данных на транспорте»

}
&
{
\centering
\fontsize{14pt}{16pt}\selectfont

УТВЕРЖДАЮ
Руководитель образовательной программы

\vspace{1cm}

\fontsize{12pt}{14pt}\selectfont
\underline{\hspace{3cm}}

к.т.н., \underline{Проневич О.Б.}

}
\\
\hline
\multicolumn{3}{!{\vrule width 2pt}p{16cm}!{\vrule width 2pt}}{
\begin{minipage}{16cm}
    \vspace{0.2cm}

\fontsize{14pt}{16pt}\selectfont\itshape
\begin{enumerate}
    \item Язык Kotlin: поддержка парадигмы абстрактного программирования, контра- и ковариантность
    \item Разработка приложений под Android: работа с базой данных, библиотека Room
    \item %\section{Задание на разработку Android-клиента}

Необходимо разработать Android-клиент для работы с сервером, предоставляющим REST API для поиска книг по описанию. Клиент должен выполнять следующие задачи:

%\subsection{Подключение к серверу}
Клиент должен взаимодействовать с сервером по адресу 

\texttt{http://10.0.2.2:8080/books/search}. Используйте библиотеку \texttt{Retrofit} для выполнения запросов к API. Пример запроса:


GET http://10.0.2.2:8080/books/search?description=love


Ответ от сервера будет содержать данные о книгах в формате JSON:


[
  \{
    "id": "1",
    "title": "1984",
    "author": "George Orwell",
    "year": 1949,
    "description": "Dystopian novel set in a totalitarian society."
  \}
]


%\subsection{Локальное хранение данных}
Для хранения результатов запросов используйте \texttt{Room}. После получения данных с сервера они должны сохраняться в локальной базе данных для работы в офлайн-режиме.

%\subsection{Архитектура приложения}
Используйте архитектуру MVVM для разделения бизнес-логики и UI. ViewModel будет отвечать за выполнение запросов через \texttt{Retrofit}, сохранение данных в \texttt{Room} и их передачу в UI.

%\subsection{Интерфейс пользователя}
Используйте \texttt{Jetpack Compose} для создания интерфейса. Интерфейс должен отображать список книг, полученных с сервера или из локальной базы данных, если интернет-соединение отсутствует.

%\subsection{Рекомендации}
В случае отсутствия интернета — отображать данные из локальной базы данных.
Соблюдать принципы архитектуры MVVM и использовать \texttt{Retrofit}, \texttt{Room} и \texttt{Jetpack Compose}. 
\end{enumerate}

\vspace{0.2cm}
    
\end{minipage}
}
\\
\noalign{\hrule height 2pt}
\end{tabular}

\newpage


\begin{tabular}{!{\vrule width 2pt}p{5cm}|p{6cm}|p{4cm}!{\vrule width 2pt}}
    \noalign{\hrule height 2pt}

    {\centering 
    \fontsize{14pt}{16pt}\selectfont
    РУТ(МИИТ)

\vspace{14pt}

Академия «Высшая инженерная школа»

\vspace{14pt}

2024/2025 учебный год

    }
&
{
    \centering
\fontsize{14pt}{16pt}\selectfont

\textbf{ЭКЗАМЕНАЦИОННЫЙ
БИЛЕТ №13}


по дисциплине 

«Разработка мобильных приложений» 
\fontsize{12pt}{14pt}\selectfont
для студентов образовательной программы «IT-сервисы и технологии обработки данных на транспорте»

}
&
{
\centering
\fontsize{14pt}{16pt}\selectfont

УТВЕРЖДАЮ
Руководитель образовательной программы

\vspace{1cm}

\fontsize{12pt}{14pt}\selectfont
\underline{\hspace{3cm}}

к.т.н., \underline{Проневич О.Б.}

}
\\
\hline
\multicolumn{3}{!{\vrule width 2pt}p{16cm}!{\vrule width 2pt}}{
\begin{minipage}{16cm}
    \vspace{0.2cm}

\fontsize{14pt}{16pt}\selectfont\itshape
\begin{enumerate}
    \item Сравнение функционального и императивного стиля в разработке
    \item Разработка приложений под Android: понятие, назначение и способ использования Navigation
    \item %\section{Задание на разработку Android-клиента для работы с точками интереса}

Необходимо разработать Android-клиент для работы с сервером, предоставляющим REST API для получения информации о точках интереса. Клиент должен выполнять следующие задачи:

%\subsection{Подключение к серверу} 

Клиент должен взаимодействовать с сервером по адресу 

\texttt{http://10.0.2.2:8080/points-of-interest}. Используйте библиотеку \texttt{Retrofit} для выполнения запросов к API. Пример запроса:

 GET http://10.0.2.2:8080/points-of-interest?category=Landmark 

Ответ от сервера будет содержать данные о точках интереса в формате JSON:

 [ \{ "id": "1", "name": "Eiffel Tower", 
    "description": "Iconic symbol of Paris", "category": "Landmark", 
    "latitude": 48.8584, "longitude": 2.2945 \} ] 

%\subsection{Локальное хранение данных} 

Для хранения результатов запросов используйте \texttt{Room}. После получения данных с сервера они должны сохраняться в локальной базе данных, чтобы обеспечить возможность работы в офлайн-режиме. Кэшированные данные должны быть доступны при отсутствии интернета.

%\subsection{Архитектура приложения} 

Используйте архитектуру MVVM для разделения бизнес-логики и UI. \texttt{ViewModel} будет отвечать за выполнение запросов через \texttt{Retrofit}, сохранение данных в \texttt{Room} и их передачу в UI.

%\subsection{Интерфейс пользователя} 

Используйте \texttt{Jetpack Compose} для создания интерфейса. Интерфейс должен отображать список точек интереса, полученных с сервера или из локальной базы данных, если интернет-соединение отсутствует.

%\subsection{Рекомендации} 

\begin{itemize}  \item В случае отсутствия интернета — отображать данные из локальной базы данных. \item Соблюдать принципы архитектуры MVVM и использовать \texttt{Retrofit}, \texttt{Room} и \texttt{Jetpack Compose}. \end{itemize} 
\end{enumerate}

\vspace{0.2cm}
    
\end{minipage}
}
\\
\noalign{\hrule height 2pt}
\end{tabular}

\newpage


\begin{tabular}{!{\vrule width 2pt}p{5cm}|p{6cm}|p{4cm}!{\vrule width 2pt}}
    \noalign{\hrule height 2pt}

    {\centering 
    \fontsize{14pt}{16pt}\selectfont
    РУТ(МИИТ)

\vspace{14pt}

Академия «Высшая инженерная школа»

\vspace{14pt}

2024/2025 учебный год

    }
&
{
    \centering
\fontsize{14pt}{16pt}\selectfont

\textbf{ЭКЗАМЕНАЦИОННЫЙ
БИЛЕТ №14}


по дисциплине 

«Разработка мобильных приложений» 
\fontsize{12pt}{14pt}\selectfont
для студентов образовательной программы «IT-сервисы и технологии обработки данных на транспорте»

}
&
{
\centering
\fontsize{14pt}{16pt}\selectfont

УТВЕРЖДАЮ
Руководитель образовательной программы

\vspace{1cm}

\fontsize{12pt}{14pt}\selectfont
\underline{\hspace{3cm}}

к.т.н., \underline{Проневич О.Б.}

}
\\
\hline
\multicolumn{3}{!{\vrule width 2pt}p{16cm}!{\vrule width 2pt}}{
\begin{minipage}{16cm}
    \vspace{0.2cm}

\fontsize{14pt}{16pt}\selectfont\itshape
\begin{enumerate}
    \item Язык Kotlin: фукции высшего порядка, понятие и примеры
    \item Разработка приложений под Android: понятие, назначение, способ реализации сервисов
    \item %\section{Задание на разработку Android-клиента для работы с сервером достопримечательностей}

Необходимо разработать Android-клиент для работы с сервером, предоставляющим REST API для получения информации о достопримечательностях. Клиент должен выполнять следующие задачи:

%\subsection{Подключение к серверу} 

Клиент должен взаимодействовать с сервером по адресу 

\texttt{http://10.0.2.2:8080/locations}. Используйте библиотеку \texttt{Retrofit} для выполнения запросов к API. Пример запроса:

 GET http://10.0.2.2:8080/locations?

     category=Historical\&latitude=40.7128\&longitude=-74.0060 

Ответ от сервера будет содержать данные о достопримечательностях в формате JSON:

 [ \{ "id": "1", "name": "Statue of Liberty", 
     "description": "Famous American monument",
     "category": "Landmark", "latitude": 40.6892, 
     "longitude": -74.0445 \} ] 

%\subsection{Локальное хранение данных} 

Для хранения результатов запросов используйте \texttt{Room}. После получения данных с сервера они должны сохраняться в локальной базе данных для работы в офлайн-режиме.

%\subsection{Архитектура приложения} 

Используйте архитектуру MVVM для разделения бизнес-логики и UI. ViewModel будет отвечать за выполнение запросов через \texttt{Retrofit}, сохранение данных в \texttt{Room} и их передачу в UI.

%\subsection{Интерфейс пользователя} 

Используйте \texttt{Jetpack Compose} для создания интерфейса. Интерфейс должен отображать список достопримечательностей, полученных с сервера или из локальной базы данных, если интернет-соединение отсутствует.

%\subsection{Рекомендации}

В случае отсутствия интернета — отображать данные из локальной базы данных.
Соблюдать принципы архитектуры MVVM и использовать \texttt{Retrofit}, \texttt{Room} и \texttt{Jetpack Compose}. 
\end{enumerate}

\vspace{0.2cm}
    
\end{minipage}
}
\\
\noalign{\hrule height 2pt}
\end{tabular}

\newpage


\begin{tabular}{!{\vrule width 2pt}p{5cm}|p{6cm}|p{4cm}!{\vrule width 2pt}}
    \noalign{\hrule height 2pt}

    {\centering 
    \fontsize{14pt}{16pt}\selectfont
    РУТ(МИИТ)

\vspace{14pt}

Академия «Высшая инженерная школа»

\vspace{14pt}

2024/2025 учебный год

    }
&
{
    \centering
\fontsize{14pt}{16pt}\selectfont

\textbf{ЭКЗАМЕНАЦИОННЫЙ
БИЛЕТ №15}


по дисциплине 

«Разработка мобильных приложений» 
\fontsize{12pt}{14pt}\selectfont
для студентов образовательной программы «IT-сервисы и технологии обработки данных на транспорте»

}
&
{
\centering
\fontsize{14pt}{16pt}\selectfont

УТВЕРЖДАЮ
Руководитель образовательной программы

\vspace{1cm}

\fontsize{12pt}{14pt}\selectfont
\underline{\hspace{3cm}}

к.т.н., \underline{Проневич О.Б.}

}
\\
\hline
\multicolumn{3}{!{\vrule width 2pt}p{16cm}!{\vrule width 2pt}}{
\begin{minipage}{16cm}
    \vspace{0.2cm}

\fontsize{14pt}{16pt}\selectfont\itshape
\begin{enumerate}
    \item Язык Kotlin: функции, предназначенные для преобразования коллекций
    \item Архитектурный шаблон MVVM: использование StateFlow при реализации MVVM
    \item %\section{Задание на разработку Android-клиента}

Необходимо разработать Android-клиент для работы с сервером, предоставляющим REST API для получения информации о достопримечательностях. Клиент должен выполнять следующие задачи:

%\subsection{Подключение к серверу}
Клиент должен взаимодействовать с сервером по адресу 

\texttt{http://10.0.2.2:9080/locations}. Используйте библиотеку \texttt{Retrofit} для выполнения запросов к API. Пример запроса:


GET http://10.0.2.2:9080/locations?category=Natural


Ответ от сервера будет содержать данные о достопримечательностях в формате JSON:


[
  \{
    "id": "1",
    "name": "Eiffel Tower",
    "description": "Iconic symbol of Paris",
    "category": "Landmark",
    "latitude": 48.8584,
    "longitude": 2.2945
  \}
]


%\subsection{Локальное хранение данных}
Для хранения результатов запросов используйте \texttt{Room}. После получения данных с сервера они должны сохраняться в локальной базе данных для работы в офлайн-режиме.

%\subsection{Архитектура приложения}
Используйте архитектуру MVVM для разделения бизнес-логики и UI. ViewModel будет отвечать за выполнение запросов через \texttt{Retrofit}, сохранение данных в \texttt{Room} и их передачу в UI.

%\subsection{Интерфейс пользователя}
Используйте \texttt{Jetpack Compose} для создания интерфейса. Интерфейс должен отображать список достопримечательностей, полученных с сервера или из локальной базы данных, если интернет-соединение отсутствует.

%\subsection{Рекомендации}
В случае отсутствия интернета — отображать данные из локальной базы данных.
Соблюдать принципы архитектуры MVVM и использовать \texttt{Retrofit}, \texttt{Room} и \texttt{Jetpack Compose}. 
\end{enumerate}

\vspace{0.2cm}
    
\end{minipage}
}
\\
\noalign{\hrule height 2pt}
\end{tabular}

\newpage


\begin{tabular}{!{\vrule width 2pt}p{5cm}|p{6cm}|p{4cm}!{\vrule width 2pt}}
    \noalign{\hrule height 2pt}

    {\centering 
    \fontsize{14pt}{16pt}\selectfont
    РУТ(МИИТ)

\vspace{14pt}

Академия «Высшая инженерная школа»

\vspace{14pt}

2024/2025 учебный год

    }
&
{
    \centering
\fontsize{14pt}{16pt}\selectfont

\textbf{ЭКЗАМЕНАЦИОННЫЙ
БИЛЕТ №16}


по дисциплине 

«Разработка мобильных приложений» 
\fontsize{12pt}{14pt}\selectfont
для студентов образовательной программы «IT-сервисы и технологии обработки данных на транспорте»

}
&
{
\centering
\fontsize{14pt}{16pt}\selectfont

УТВЕРЖДАЮ
Руководитель образовательной программы

\vspace{1cm}

\fontsize{12pt}{14pt}\selectfont
\underline{\hspace{3cm}}

к.т.н., \underline{Проневич О.Б.}

}
\\
\hline
\multicolumn{3}{!{\vrule width 2pt}p{16cm}!{\vrule width 2pt}}{
\begin{minipage}{16cm}
    \vspace{0.2cm}

\fontsize{14pt}{16pt}\selectfont\itshape
\begin{enumerate}
    \item Язык Kotlin: синтаксис и семантика условных выражений языка
    \item Разработка приложений под Android: понятие, назначение и способы использования Intent
    \item %\section{Задание на разработку Android-клиента для работы с сервером поиска ресторанов}

Разработать Android-клиент для взаимодействия с сервером, предоставляющим API для поиска ресторанов по названию.

%\subsection{Основные требования}
\begin{itemize}
    \item Сервер доступен по адресу \texttt{http://10.0.2.2:9080/search}.
    \item Использовать \texttt{Retrofit} для выполнения HTTP-запросов. Пример запроса:

GET http://10.0.2.2:9080/search?name=sushi

Ответ возвращается в формате JSON и содержит список ресторанов с полями \texttt{id}, 
\texttt{name}, \texttt{cuisine}, \texttt{location}, \texttt{is\_open}.
    \item Если рестораны не найдены, сервер возвращает ошибку \texttt{404 Not Found}.
\end{itemize}

%\subsection{Хранение данных}
Реализовать локальное кэширование данных с помощью \texttt{Room} для обеспечения офлайн-режима.

%\subsection{Архитектура}
Использовать архитектуру MVVM для разделения ответственности:
\begin{itemize}
    \item \texttt{ViewModel} управляет данными и выполняет запросы к серверу.
    \item \texttt{Repository} реализует логику взаимодействия с API и локальной базой данных.
    \item Пользовательский интерфейс разработан на базе \texttt{Jetpack Compose}.
\end{itemize}

%\subsection{Интерфейс}
Реализовать экран с поисковой строкой для ввода части названия ресторана и отображения результатов в виде списка. Каждый элемент списка должен содержать информацию о названии, типе кухни, местоположении и статусе (\texttt{is\_open}).

%\subsection{Дополнительные требования}
\begin{itemize}
    \item Реализовать обработку ошибок (например, отсутствие результатов или проблемы с подключением к серверу).
    \item Обновлять локальные данные при наличии подключения к интернету.
\end{itemize} 
\end{enumerate}

\vspace{0.2cm}
    
\end{minipage}
}
\\
\noalign{\hrule height 2pt}
\end{tabular}

\newpage


\begin{tabular}{!{\vrule width 2pt}p{5cm}|p{6cm}|p{4cm}!{\vrule width 2pt}}
    \noalign{\hrule height 2pt}

    {\centering 
    \fontsize{14pt}{16pt}\selectfont
    РУТ(МИИТ)

\vspace{14pt}

Академия «Высшая инженерная школа»

\vspace{14pt}

2024/2025 учебный год

    }
&
{
    \centering
\fontsize{14pt}{16pt}\selectfont

\textbf{ЭКЗАМЕНАЦИОННЫЙ
БИЛЕТ №17}


по дисциплине 

«Разработка мобильных приложений» 
\fontsize{12pt}{14pt}\selectfont
для студентов образовательной программы «IT-сервисы и технологии обработки данных на транспорте»

}
&
{
\centering
\fontsize{14pt}{16pt}\selectfont

УТВЕРЖДАЮ
Руководитель образовательной программы

\vspace{1cm}

\fontsize{12pt}{14pt}\selectfont
\underline{\hspace{3cm}}

к.т.н., \underline{Проневич О.Б.}

}
\\
\hline
\multicolumn{3}{!{\vrule width 2pt}p{16cm}!{\vrule width 2pt}}{
\begin{minipage}{16cm}
    \vspace{0.2cm}

\fontsize{14pt}{16pt}\selectfont\itshape
\begin{enumerate}
    \item Язык Kotlin: основные типы данных
    \item Методы создания мобильных приложений
    \item %\section{Задание на разработку Android-клиента для работы с сервером ресторанов}

Разработать Android-клиент для взаимодействия с сервером, предоставляющим REST API для поиска ресторанов по типу кухни. 

%\subsection{Основные требования}
\begin{itemize}
    \item Адрес сервера: \texttt{http://10.0.2.2:9080/restaurants}.
    \item Для выполнения запросов используйте \texttt{Retrofit}. Пример запроса:

GET http://10.0.2.2:9080/restaurants?cuisine=Japanese

Ответ в формате JSON содержит список ресторанов с полями \texttt{id}, \texttt{name}, \texttt{cuisine}, \texttt{city} и \texttt{is\_open}.
\end{itemize}

%\subsection{Хранение данных}
Сохранять данные запросов в локальной базе с помощью \texttt{Room} для работы в офлайн-режиме.

%\subsection{Архитектура}
Использовать архитектуру MVVM для разделения логики и интерфейса:
\begin{itemize}
    \item \texttt{ViewModel} отвечает за запросы к API и управление данными.
    \item \texttt{Repository} для взаимодействия с сервером и базой данных.
    \item UI на основе \texttt{Jetpack Compose}.
\end{itemize}

%\subsection{Интерфейс}
Реализовать экран для ввода типа кухни и отображения списка ресторанов, с поддержкой офлайн-режима.

%\subsection{Рекомендации}
Реализовать обработку ошибок (например, отсутствие параметра \texttt{cuisine}). 
\end{enumerate}

\vspace{0.2cm}
    
\end{minipage}
}
\\
\noalign{\hrule height 2pt}
\end{tabular}

\newpage


\begin{tabular}{!{\vrule width 2pt}p{5cm}|p{6cm}|p{4cm}!{\vrule width 2pt}}
    \noalign{\hrule height 2pt}

    {\centering 
    \fontsize{14pt}{16pt}\selectfont
    РУТ(МИИТ)

\vspace{14pt}

Академия «Высшая инженерная школа»

\vspace{14pt}

2024/2025 учебный год

    }
&
{
    \centering
\fontsize{14pt}{16pt}\selectfont

\textbf{ЭКЗАМЕНАЦИОННЫЙ
БИЛЕТ №18}


по дисциплине 

«Разработка мобильных приложений» 
\fontsize{12pt}{14pt}\selectfont
для студентов образовательной программы «IT-сервисы и технологии обработки данных на транспорте»

}
&
{
\centering
\fontsize{14pt}{16pt}\selectfont

УТВЕРЖДАЮ
Руководитель образовательной программы

\vspace{1cm}

\fontsize{12pt}{14pt}\selectfont
\underline{\hspace{3cm}}

к.т.н., \underline{Проневич О.Б.}

}
\\
\hline
\multicolumn{3}{!{\vrule width 2pt}p{16cm}!{\vrule width 2pt}}{
\begin{minipage}{16cm}
    \vspace{0.2cm}

\fontsize{14pt}{16pt}\selectfont\itshape
\begin{enumerate}
    \item Язык Kotlin: синтаксис и семантика циклов
    \item Разработка приложений под Android: локализация приложения
    \item %\section{Задание на разработку Android-клиента для работы с сервером ресторанов}

Необходимо разработать Android-клиент для работы с сервером, предоставляющим REST API для получения информации о ресторанах. Клиент должен выполнять следующие задачи:

%\subsection{Подключение к серверу}
Клиент должен взаимодействовать с сервером по адресу 

\texttt{http://10.0.2.2:9080/restaurants}. Используйте библиотеку \texttt{Retrofit} для выполнения запросов к API. Пример запроса:


GET http://10.0.2.2:9080/restaurants?city=New York


Ответ от сервера будет содержать данные о ресторанах в формате JSON:


[
  \{
    "id": "1",
    "name": "Pasta House",
    "cuisine": "Italian",
    "city": "New York",
    "is\_open": true
  \}
]


%\subsection{Локальное хранение данных}
Для хранения результатов запросов используйте \texttt{Room}. После получения данных с сервера они должны сохраняться в локальной базе данных, чтобы обеспечить возможность работы в офлайн-режиме. Кэшированные данные должны быть доступны при отсутствии интернета.

%\subsection{Архитектура приложения}
Используйте архитектуру MVVM для разделения бизнес-логики и UI. \texttt{ViewModel} будет отвечать за выполнение запросов через \texttt{Retrofit}, сохранение данных в \texttt{Room} и их передачу в UI.

%\subsection{Интерфейс пользователя}
Используйте \texttt{Jetpack Compose} для создания интерфейса. Интерфейс должен отображать список ресторанов, полученных с сервера или из локальной базы данных, если интернет-соединение отсутствует. Рестораны должны быть отображены с информацией о названии, кухне, городе и статусе открытия.

%\subsection{Рекомендации}
\begin{itemize}
  \item В случае отсутствия интернета — отображать данные из локальной базы данных.
  \item Соблюдать принципы архитектуры MVVM и использовать \texttt{Retrofit}, \texttt{Room} и \texttt{Jetpack Compose}.
  \item Реализовать обработку ошибок, например, при отсутствии параметра города в запросе.
\end{itemize} 
\end{enumerate}

\vspace{0.2cm}
    
\end{minipage}
}
\\
\noalign{\hrule height 2pt}
\end{tabular}

\newpage


\begin{tabular}{!{\vrule width 2pt}p{5cm}|p{6cm}|p{4cm}!{\vrule width 2pt}}
    \noalign{\hrule height 2pt}

    {\centering 
    \fontsize{14pt}{16pt}\selectfont
    РУТ(МИИТ)

\vspace{14pt}

Академия «Высшая инженерная школа»

\vspace{14pt}

2024/2025 учебный год

    }
&
{
    \centering
\fontsize{14pt}{16pt}\selectfont

\textbf{ЭКЗАМЕНАЦИОННЫЙ
БИЛЕТ №19}


по дисциплине 

«Разработка мобильных приложений» 
\fontsize{12pt}{14pt}\selectfont
для студентов образовательной программы «IT-сервисы и технологии обработки данных на транспорте»

}
&
{
\centering
\fontsize{14pt}{16pt}\selectfont

УТВЕРЖДАЮ
Руководитель образовательной программы

\vspace{1cm}

\fontsize{12pt}{14pt}\selectfont
\underline{\hspace{3cm}}

к.т.н., \underline{Проневич О.Б.}

}
\\
\hline
\multicolumn{3}{!{\vrule width 2pt}p{16cm}!{\vrule width 2pt}}{
\begin{minipage}{16cm}
    \vspace{0.2cm}

\fontsize{14pt}{16pt}\selectfont\itshape
\begin{enumerate}
    \item Язык Kotlin: свойства (properties) класса
    \item Разработка приложений под Android: компоненты Jetpack compose, предназначенные для ввода-вывода
    \item %\section{Задание на разработку Android-клиента для работы с сервером квартир}

Необходимо разработать Android-клиент для работы с сервером, предоставляющим REST API для получения информации о квартирах. Клиент должен выполнять следующие задачи:

%\subsection{Подключение к серверу}
Клиент должен взаимодействовать с сервером по адресу 

\texttt{http://10.0.2.2:9080/apartments}. Используйте библиотеку \texttt{Retrofit} для выполнения запросов к API. Пример запроса:


GET http://10.0.2.2:9080/apartments?min\_price=30000\&max\_price=60000


Ответ от сервера будет содержать данные о квартирах в формате JSON:


[
  \{
    "id": "1",
    "location": "New York",
    "price": 50000,
    "area": 70,
    "floor": 10,
    "rooms": 2,
    "year\_built": 2000
  \}
]


%\subsection{Локальное хранение данных}
Для хранения результатов запросов используйте \texttt{Room}. После получения данных с сервера они должны сохраняться в локальной базе данных, чтобы обеспечить возможность работы в офлайн-режиме. Кэшированные данные должны быть доступны при отсутствии интернета.

%\subsection{Архитектура приложения}
Используйте архитектуру MVVM для разделения бизнес-логики и UI. \texttt{ViewModel} будет отвечать за выполнение запросов через \texttt{Retrofit}, сохранение данных в \texttt{Room} и их передачу в UI.

%\subsection{Интерфейс пользователя}
Используйте \texttt{Jetpack Compose} для создания интерфейса. Интерфейс должен отображать список квартир, полученных с сервера или из локальной базы данных, если интернет-соединение отсутствует.

%\subsection{Рекомендации}
\begin{itemize}
  \item В случае отсутствия интернета — отображать данные из локальной базы данных.
  \item Соблюдать принципы архитектуры MVVM и использовать \texttt{Retrofit}, \texttt{Room} и \texttt{Jetpack Compose}.
\end{itemize} 
\end{enumerate}

\vspace{0.2cm}
    
\end{minipage}
}
\\
\noalign{\hrule height 2pt}
\end{tabular}

\newpage


\begin{tabular}{!{\vrule width 2pt}p{5cm}|p{6cm}|p{4cm}!{\vrule width 2pt}}
    \noalign{\hrule height 2pt}

    {\centering 
    \fontsize{14pt}{16pt}\selectfont
    РУТ(МИИТ)

\vspace{14pt}

Академия «Высшая инженерная школа»

\vspace{14pt}

2024/2025 учебный год

    }
&
{
    \centering
\fontsize{14pt}{16pt}\selectfont

\textbf{ЭКЗАМЕНАЦИОННЫЙ
БИЛЕТ №20}


по дисциплине 

«Разработка мобильных приложений» 
\fontsize{12pt}{14pt}\selectfont
для студентов образовательной программы «IT-сервисы и технологии обработки данных на транспорте»

}
&
{
\centering
\fontsize{14pt}{16pt}\selectfont

УТВЕРЖДАЮ
Руководитель образовательной программы

\vspace{1cm}

\fontsize{12pt}{14pt}\selectfont
\underline{\hspace{3cm}}

к.т.н., \underline{Проневич О.Б.}

}
\\
\hline
\multicolumn{3}{!{\vrule width 2pt}p{16cm}!{\vrule width 2pt}}{
\begin{minipage}{16cm}
    \vspace{0.2cm}

\fontsize{14pt}{16pt}\selectfont\itshape
\begin{enumerate}
    \item Язык Kotlin: синтаксис и семантика создания обычных функций
    \item Разработка приложений под Android: понятие, назначение и цикл жизни Activity
    \item %\section{Задание на разработку Android-клиента для работы с сервером недвижимости с фильтрацией по году постройки}

Необходимо разработать Android-клиент для работы с сервером, который предоставляет REST API для получения информации о квартирах. Клиент должен выполнять следующие задачи:

%\subsection{Подключение к серверу}
Сервер доступен по адресу \texttt{http://10.0.2.2:9080/apartments}, и предоставляет данные о квартирах с параметрами для фильтрации по году постройки. Клиент должен использовать библиотеку \texttt{Retrofit} для выполнения запросов к серверу.

Пример запроса:

GET http://10.0.2.2:9080/apartments?min\_year=2000\&max\_year=2020


Ответ от сервера будет содержать данные о квартирах в формате JSON, например:


[
  \{
    "id": "1",
    "location": "New York",
    "price": 500000,
    "area": 70,
    "floor": 10,
    "rooms": 2,
    "year\_built": 2000
  \},
]


%\subsection{Локальное хранение данных}
Для хранения результатов запросов используйте \texttt{Room}. 
После получения данных с сервера, они должны сохраняться в локальной базе данных, 
чтобы обеспечить возможность работы в офлайн-режиме. 

%\subsection{Архитектура приложения}
Используйте архитектуру \texttt{MVVM} для разделения бизнес-логики и UI. \texttt{ViewModel} будет отвечать за выполнение запросов через \texttt{Retrofit}, сохранение данных в \texttt{Room} и их передачу в UI.

%\subsection{Интерфейс пользователя}
Используйте \texttt{Jetpack Compose} для создания интерфейса. Интерфейс должен отображать список квартир, полученных с сервера или из локальной базы данных, если интернет-соединение отсутствует. Также, пользователь должен иметь возможность вводить минимальный и максимальный год постройки через текстовые поля в интерфейсе.

%\subsection{Рекомендации}
\begin{itemize}
  \item В случае отсутствия интернета — отображать данные из локальной базы данных.
  \item Соблюдать принципы архитектуры \texttt{MVVM} и использовать \texttt{Retrofit}, \texttt{Room} и \texttt{Jetpack Compose}.
\end{itemize} 
\end{enumerate}

\vspace{0.2cm}
    
\end{minipage}
}
\\
\noalign{\hrule height 2pt}
\end{tabular}

\newpage


\begin{tabular}{!{\vrule width 2pt}p{5cm}|p{6cm}|p{4cm}!{\vrule width 2pt}}
    \noalign{\hrule height 2pt}

    {\centering 
    \fontsize{14pt}{16pt}\selectfont
    РУТ(МИИТ)

\vspace{14pt}

Академия «Высшая инженерная школа»

\vspace{14pt}

2024/2025 учебный год

    }
&
{
    \centering
\fontsize{14pt}{16pt}\selectfont

\textbf{ЭКЗАМЕНАЦИОННЫЙ
БИЛЕТ №21}


по дисциплине 

«Разработка мобильных приложений» 
\fontsize{12pt}{14pt}\selectfont
для студентов образовательной программы «IT-сервисы и технологии обработки данных на транспорте»

}
&
{
\centering
\fontsize{14pt}{16pt}\selectfont

УТВЕРЖДАЮ
Руководитель образовательной программы

\vspace{1cm}

\fontsize{12pt}{14pt}\selectfont
\underline{\hspace{3cm}}

к.т.н., \underline{Проневич О.Б.}

}
\\
\hline
\multicolumn{3}{!{\vrule width 2pt}p{16cm}!{\vrule width 2pt}}{
\begin{minipage}{16cm}
    \vspace{0.2cm}

\fontsize{14pt}{16pt}\selectfont\itshape
\begin{enumerate}
    \item Язык Kotlin: делегирование свойств, lazy
    \item Разработка приложений под Android: особенности параллельного программирования
    \item %\section{Задание на разработку Android-клиента для работы с сервером недвижимости}

Необходимо разработать Android-клиент для работы с сервером, который предоставляет REST API для получения информации о квартирах. Клиент должен выполнять следующие задачи:

%\subsection{Подключение к серверу}
Сервер доступен по адресу \texttt{http://10.0.2.2:9080/apartments}, и предоставляет данные о квартирах с параметрами для фильтрации по количеству комнат. Клиент должен использовать библиотеку \texttt{Retrofit} для выполнения запросов к серверу.

Пример запроса:

GET http://10.0.2.2:9080/apartments?min\_rooms=2\&max\_rooms=3


Ответ от сервера будет содержать данные о квартирах в формате JSON, например:


[
  \{
    "id": "1",
    "location": "New York",
    "price": 500000,
    "area": 70,
    "floor": 10,
    "rooms": 2,
    "year\_built": 2000
  \},
]


%\subsection{Локальное хранение данных}
Для хранения результатов запросов используйте \texttt{Room}. 
После получения данных с сервера, они должны сохраняться в локальной базе данных, 
чтобы обеспечить возможность работы в офлайн-режиме. 

%\subsection{Архитектура приложения}
Используйте архитектуру \texttt{MVVM} для разделения бизнес-логики и UI. \texttt{ViewModel} будет отвечать за выполнение запросов через \texttt{Retrofit}, сохранение данных в \texttt{Room} и их передачу в UI.

%\subsection{Интерфейс пользователя}
Используйте \texttt{Jetpack Compose} для создания интерфейса. Интерфейс должен отображать список квартир, полученных с сервера или из локальной базы данных, если интернет-соединение отсутствует. Также, пользователь должен иметь возможность вводить минимальное и максимальное количество комнат через текстовые поля в интерфейсе.

%\subsection{Рекомендации}
\begin{itemize}
  \item В случае отсутствия интернета — отображать данные из локальной базы данных.
  \item Соблюдать принципы архитектуры \texttt{MVVM} и использовать \texttt{Retrofit}, \texttt{Room} и \texttt{Jetpack Compose}.
  \item В запросах должны передаваться параметры \texttt{min\_rooms} и \texttt{max\_rooms}.
\end{itemize} 
\end{enumerate}

\vspace{0.2cm}
    
\end{minipage}
}
\\
\noalign{\hrule height 2pt}
\end{tabular}

\newpage


\begin{tabular}{!{\vrule width 2pt}p{5cm}|p{6cm}|p{4cm}!{\vrule width 2pt}}
    \noalign{\hrule height 2pt}

    {\centering 
    \fontsize{14pt}{16pt}\selectfont
    РУТ(МИИТ)

\vspace{14pt}

Академия «Высшая инженерная школа»

\vspace{14pt}

2024/2025 учебный год

    }
&
{
    \centering
\fontsize{14pt}{16pt}\selectfont

\textbf{ЭКЗАМЕНАЦИОННЫЙ
БИЛЕТ №22}


по дисциплине 

«Разработка мобильных приложений» 
\fontsize{12pt}{14pt}\selectfont
для студентов образовательной программы «IT-сервисы и технологии обработки данных на транспорте»

}
&
{
\centering
\fontsize{14pt}{16pt}\selectfont

УТВЕРЖДАЮ
Руководитель образовательной программы

\vspace{1cm}

\fontsize{12pt}{14pt}\selectfont
\underline{\hspace{3cm}}

к.т.н., \underline{Проневич О.Б.}

}
\\
\hline
\multicolumn{3}{!{\vrule width 2pt}p{16cm}!{\vrule width 2pt}}{
\begin{minipage}{16cm}
    \vspace{0.2cm}

\fontsize{14pt}{16pt}\selectfont\itshape
\begin{enumerate}
    \item Язык Kotlin: поддержка параллельного программирования посредством корутин
    \item Паттерн репозиторий
    \item %\section{Задание на разработку Android-клиента для работы с сервером недвижимости}

Необходимо разработать Android-клиент для работы с сервером, который предоставляет REST API для получения информации о квартирах. Клиент должен выполнять следующие задачи:

%\subsection{Подключение к серверу}
Клиент должен взаимодействовать с сервером по адресу 

\texttt{http://10.0.2.2:9080/apartments}. Используйте библиотеку \texttt{Retrofit} для выполнения запросов к API. Пример запроса:


GET http://10.0.2.2:9080/apartments?min\_area=50\&max\_area=100


Ответ от сервера будет содержать данные о квартирах в формате JSON, например:


[
  \{
    "id": "1",
    "location": "New York",
    "price": 500000,
    "area": 70,
    "floor": 10,
    "rooms": 2,
    "year\_built": 2000
  \}
]


%\subsection{Локальное хранение данных}
Для хранения результатов запросов используйте \texttt{Room}. После получения данных с сервера они должны сохраняться в локальной базе данных, чтобы обеспечить возможность работы в офлайн-режиме. Важно, чтобы при отсутствии интернет-соединения, клиент показывал данные, сохранённые локально.

%\subsection{Архитектура приложения}
Используйте архитектуру \texttt{MVVM} для разделения бизнес-логики и UI. \texttt{ViewModel} будет отвечать за выполнение запросов через \texttt{Retrofit}, сохранение данных в \texttt{Room} и их передачу в UI.

%\subsection{Интерфейс пользователя}
Используйте \texttt{Jetpack Compose} для создания интерфейса. Интерфейс должен отображать список квартир, полученных с сервера или из локальной базы данных, если интернет-соединение отсутствует. Также, пользователь должен иметь возможность вводить минимальную и максимальную площадь квартир через текстовые поля в интерфейсе.

%\subsection{Рекомендации}
\begin{itemize}
  \item В случае отсутствия интернета — отображать данные из локальной базы данных.
  \item Соблюдать принципы архитектуры \texttt{MVVM} и использовать \texttt{Retrofit}, \texttt{Room} и \texttt{Jetpack Compose}.
  \item В запросах должны передаваться параметры \texttt{min\_area} и \texttt{max\_area}.
\end{itemize} 
\end{enumerate}

\vspace{0.2cm}
    
\end{minipage}
}
\\
\noalign{\hrule height 2pt}
\end{tabular}

\newpage


\begin{tabular}{!{\vrule width 2pt}p{5cm}|p{6cm}|p{4cm}!{\vrule width 2pt}}
    \noalign{\hrule height 2pt}

    {\centering 
    \fontsize{14pt}{16pt}\selectfont
    РУТ(МИИТ)

\vspace{14pt}

Академия «Высшая инженерная школа»

\vspace{14pt}

2024/2025 учебный год

    }
&
{
    \centering
\fontsize{14pt}{16pt}\selectfont

\textbf{ЭКЗАМЕНАЦИОННЫЙ
БИЛЕТ №23}


по дисциплине 

«Разработка мобильных приложений» 
\fontsize{12pt}{14pt}\selectfont
для студентов образовательной программы «IT-сервисы и технологии обработки данных на транспорте»

}
&
{
\centering
\fontsize{14pt}{16pt}\selectfont

УТВЕРЖДАЮ
Руководитель образовательной программы

\vspace{1cm}

\fontsize{12pt}{14pt}\selectfont
\underline{\hspace{3cm}}

к.т.н., \underline{Проневич О.Б.}

}
\\
\hline
\multicolumn{3}{!{\vrule width 2pt}p{16cm}!{\vrule width 2pt}}{
\begin{minipage}{16cm}
    \vspace{0.2cm}

\fontsize{14pt}{16pt}\selectfont\itshape
\begin{enumerate}
    \item Язык Kotlin: сранение коллекций и sequence
    \item Разработка приложений под Android: структура проекта
    \item %\section{Задание на разработку клиента для REST API сервера}

Необходимо разработать клиент для взаимодействия с сервером, предоставляющим REST API для получения данных о маршрутах автобусов. Клиент должен выполнять следующие задачи:

%\subsection{Подключение к серверу}

Для выполнения запросов к API необходимо использовать библиотеку \texttt{Retrofit}. Пример запроса:


GET http://10.0.2.2:9080/routes?min\_length=10\&max\_length=20


Ответ от сервера возвращается в формате JSON и содержит данные о маршрутах автобусов:


[
  \{
    "id": "1",
    "name": "Route A",
    "length": 12.5,
    "number\_of\_stops": 8
  \}
]


%\subsection{Локальное хранение данных}
Для хранения данных о маршрутах автобусов необходимо использовать библиотеку \texttt{Room}. После получения данных с сервера они должны быть сохранены в локальной базе данных, чтобы быть доступными для работы в офлайн-режиме.

%\subsection{Архитектура приложения}
Использовать архитектуру MVVM (Model-View-ViewModel) для разделения бизнес-логики и пользовательского интерфейса. Компонент \texttt{ViewModel} должен выполнять следующие функции:
\begin{itemize}
    \item Выполнение запросов к серверу через библиотеку \texttt{Retrofit}.
    \item Сохранение данных в локальной базе данных с использованием \texttt{Room}.
    \item Передача данных в интерфейс приложения.
\end{itemize}

%\subsection{Интерфейс пользователя}
Использовать библиотеку \texttt{Jetpack Compose} для создания пользовательского интерфейса.

При отсутствии интернет-соединения необходимо отображать данные из локальной базы данных. 
\end{enumerate}

\vspace{0.2cm}
    
\end{minipage}
}
\\
\noalign{\hrule height 2pt}
\end{tabular}

\newpage


\begin{tabular}{!{\vrule width 2pt}p{5cm}|p{6cm}|p{4cm}!{\vrule width 2pt}}
    \noalign{\hrule height 2pt}

    {\centering 
    \fontsize{14pt}{16pt}\selectfont
    РУТ(МИИТ)

\vspace{14pt}

Академия «Высшая инженерная школа»

\vspace{14pt}

2024/2025 учебный год

    }
&
{
    \centering
\fontsize{14pt}{16pt}\selectfont

\textbf{ЭКЗАМЕНАЦИОННЫЙ
БИЛЕТ №24}


по дисциплине 

«Разработка мобильных приложений» 
\fontsize{12pt}{14pt}\selectfont
для студентов образовательной программы «IT-сервисы и технологии обработки данных на транспорте»

}
&
{
\centering
\fontsize{14pt}{16pt}\selectfont

УТВЕРЖДАЮ
Руководитель образовательной программы

\vspace{1cm}

\fontsize{12pt}{14pt}\selectfont
\underline{\hspace{3cm}}

к.т.н., \underline{Проневич О.Б.}

}
\\
\hline
\multicolumn{3}{!{\vrule width 2pt}p{16cm}!{\vrule width 2pt}}{
\begin{minipage}{16cm}
    \vspace{0.2cm}

\fontsize{14pt}{16pt}\selectfont\itshape
\begin{enumerate}
    \item Язык Kotlin: перегрузка операторов
    \item Разработка приложений под Android: автоматическое Unit-тестирование
    \item %\section{Задание на разработку Android-клиента}

Необходимо разработать Android-клиент для работы с сервером, предоставляющим REST API для получения информации о продуктах. Клиент должен выполнять следующие задачи:

%\subsection{Подключение к серверу}
Клиент должен взаимодействовать с сервером по адресу 

\texttt{http://10.0.2.2:9080/products}. Используйте библиотеку \texttt{Retrofit} для выполнения запросов к API. Пример запроса:


GET http://10.0.2.2:9080/products?
min\_price=50\&max\_price=200


Ответ от сервера будет содержать данные о продуктах в формате JSON:


[
  \{
    "id": "2",
    "name": "Груша",
    "price": 110.75,
    "shelf\_life": 25,
    "is\_organic": false
  \}
]


%\subsection{Локальное хранение данных}
Для хранения результатов запросов используйте \texttt{Room}. После получения данных с сервера они должны сохраняться в локальной базе данных, чтобы обеспечить возможность работы в офлайн-режиме. Кэшированные данные должны быть доступны при отсутствии интернета.

%\subsection{Архитектура приложения}
Используйте архитектуру MVVM для разделения бизнес-логики и UI. \texttt{ViewModel} будет отвечать за выполнение запросов через \texttt{Retrofit}, сохранение данных в \texttt{Room} и их передачу в UI.

%\subsection{Интерфейс пользователя}
Используйте \texttt{Jetpack Compose} для создания интерфейса. Интерфейс должен отображать список продуктов, полученных с сервера или из локальной базы данных, если интернет-соединение отсутствует.

%\subsection{Рекомендации}
\begin{itemize}
  \item В случае отсутствия интернета — отображать данные из локальной базы данных.
  \item Соблюдать принципы архитектуры MVVM и использовать \texttt{Retrofit}, \texttt{Room} и \texttt{Jetpack Compose}.
\end{itemize} 
\end{enumerate}

\vspace{0.2cm}
    
\end{minipage}
}
\\
\noalign{\hrule height 2pt}
\end{tabular}

\newpage


\begin{tabular}{!{\vrule width 2pt}p{5cm}|p{6cm}|p{4cm}!{\vrule width 2pt}}
    \noalign{\hrule height 2pt}

    {\centering 
    \fontsize{14pt}{16pt}\selectfont
    РУТ(МИИТ)

\vspace{14pt}

Академия «Высшая инженерная школа»

\vspace{14pt}

2024/2025 учебный год

    }
&
{
    \centering
\fontsize{14pt}{16pt}\selectfont

\textbf{ЭКЗАМЕНАЦИОННЫЙ
БИЛЕТ №25}


по дисциплине 

«Разработка мобильных приложений» 
\fontsize{12pt}{14pt}\selectfont
для студентов образовательной программы «IT-сервисы и технологии обработки данных на транспорте»

}
&
{
\centering
\fontsize{14pt}{16pt}\selectfont

УТВЕРЖДАЮ
Руководитель образовательной программы

\vspace{1cm}

\fontsize{12pt}{14pt}\selectfont
\underline{\hspace{3cm}}

к.т.н., \underline{Проневич О.Б.}

}
\\
\hline
\multicolumn{3}{!{\vrule width 2pt}p{16cm}!{\vrule width 2pt}}{
\begin{minipage}{16cm}
    \vspace{0.2cm}

\fontsize{14pt}{16pt}\selectfont\itshape
\begin{enumerate}
    \item Язык Kotlin: передача данных между корутинами с помощью Flow
    \item Разработка приложений под Android: основные компоненты
    \item %\section{Задание на разработку Android-клиента}

Необходимо разработать Android-клиент для работы с сервером, предоставляющим REST API для получения информации о продуктах. Клиент должен выполнять следующие задачи:

%\subsection{Подключение к серверу}
Клиент должен взаимодействовать с сервером по адресу 

\texttt{http://10.0.2.2:9080/products}. Используйте библиотеку \texttt{Retrofit} для выполнения запросов к API. Пример запроса:


GET http://10.0.2.2:9080/products?
min\_shelf\_life=2023-01-01\&max\_shelf\_life=2024-01-01


Ответ от сервера будет содержать данные о продуктах в формате JSON:


[
  \{
    "id": 1,
    "name": "Молоко",
    "category": "Молочные продукты",
    "shelf\_life": "2024-01-01",
    "price": 100.50,
    "is\_organic": true
  \}
]


%\subsection{Локальное хранение данных}
Для хранения результатов запросов используйте \texttt{Room}. После получения данных с сервера они должны сохраняться в локальной базе данных для работы в офлайн-режиме.

%\subsection{Архитектура приложения}
Используйте архитектуру MVVM для разделения бизнес-логики и UI. ViewModel будет отвечать за выполнение запросов через \texttt{Retrofit}, сохранение данных в \texttt{Room} и их передачу в UI.

%\subsection{Интерфейс пользователя}
Используйте \texttt{Jetpack Compose} для создания интерфейса. Интерфейс должен отображать список продуктов, полученных с сервера или из локальной базы данных, если интернет-соединение отсутствует.

%\subsection{Рекомендации}
В случае отсутствия интернета — отображать данные из локальной базы данных.
Соблюдать принципы архитектуры MVVM и использовать \texttt{Retrofit}, \texttt{Room} и \texttt{Jetpack Compose}. 
\end{enumerate}

\vspace{0.2cm}
    
\end{minipage}
}
\\
\noalign{\hrule height 2pt}
\end{tabular}

\newpage
\end{document}
