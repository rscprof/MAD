\section{Задание на разработку Android-клиента для работы с сервером поиска ресторанов}

Разработать Android-клиент для взаимодействия с сервером, предоставляющим API для поиска ресторанов по названию.

\subsection{Основные требования}
\begin{itemize}
    \item Сервер доступен по адресу \texttt{http://10.0.2.2:9080/search}.
    \item Использовать \texttt{Retrofit} для выполнения HTTP-запросов. Пример запроса:
\begin{verbatim}
GET http://10.0.2.2:9080/search?name=sushi
\end{verbatim}
Ответ возвращается в формате JSON и содержит список ресторанов с полями \texttt{id}, \texttt{name}, \texttt{cuisine}, \texttt{location}, \texttt{is\_open}.
    \item Если рестораны не найдены, сервер возвращает ошибку \texttt{404 Not Found}.
\end{itemize}

\subsection{Хранение данных}
Реализовать локальное кэширование данных с помощью \texttt{Room} для обеспечения офлайн-режима.

\subsection{Архитектура}
Использовать архитектуру MVVM для разделения ответственности:
\begin{itemize}
    \item \texttt{ViewModel} управляет данными и выполняет запросы к серверу.
    \item \texttt{Repository} реализует логику взаимодействия с API и локальной базой данных.
    \item Пользовательский интерфейс разработан на базе \texttt{Jetpack Compose}.
\end{itemize}

\subsection{Интерфейс}
Реализовать экран с поисковой строкой для ввода части названия ресторана и отображения результатов в виде списка. Каждый элемент списка должен содержать информацию о названии, типе кухни, местоположении и статусе (\texttt{is\_open}).

\subsection{Дополнительные требования}
\begin{itemize}
    \item Реализовать обработку ошибок (например, отсутствие результатов или проблемы с подключением к серверу).
    \item Обновлять локальные данные при наличии подключения к интернету.
    \item Обеспечить поддержку тёмной темы интерфейса.
\end{itemize}
