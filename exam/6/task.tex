%\section{Задание на разработку Android-клиента}

Необходимо разработать Android-клиент для работы с сервером, предоставляющим REST API для поиска информации об аниме. Клиент должен выполнять следующие задачи:

%\subsection{Подключение к серверу} 

Клиент должен взаимодействовать с сервером по адресу 

\texttt{http://10.0.2.2:9080/search}. Используйте библиотеку \texttt{Retrofit} для выполнения запросов к API. Пример запроса:

\begin{verbatim} GET http://10.0.2.2:9080/search?title=Naruto \end{verbatim}

Ответ от сервера будет содержать данные об аниме в формате JSON:

\begin{verbatim} [ { "id": "1", "title": "Naruto", 
    "genre": "Action, Adventure", "episodes": 220 }] \end{verbatim}

%\subsection{Локальное хранение данных} 

Для хранения результатов запросов используйте \texttt{Room}. После получения данных с сервера они должны сохраняться в локальной базе данных для работы в офлайн-режиме.

%\subsection{Архитектура приложения} 

Используйте архитектуру MVVM для разделения бизнес-логики и UI. ViewModel будет отвечать за выполнение запросов через \texttt{Retrofit}, сохранение данных в \texttt{Room} и их передачу в UI.

%\subsection{Интерфейс пользователя} 

Используйте \texttt{Jetpack Compose} для создания интерфейса. Интерфейс должен предоставлять следующие возможности:

Поле для ввода текста, позволяющее пользователю ввести название аниме для поиска.
Список аниме, соответствующих запросу, отображаемый с названием, жанром и количеством эпизодов.
Сообщение об отсутствии результатов, если аниме не найдено.
%\subsection{Функциональные требования}

При подключении к интернету данные должны обновляться и отображаться в реальном времени.
В случае отсутствия интернет-соединения данные должны браться из локальной базы данных.
%\subsection{Рекомендации}

