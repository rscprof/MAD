\section{Задание на разработку Android-клиента}

Необходимо разработать Android-клиент для работы с сервером, предоставляющим REST API для получения информации о достопримечательностях. Клиент должен выполнять следующие задачи:

\subsection{Подключение к серверу}
Клиент должен взаимодействовать с сервером по адресу \texttt{http://10.0.2.2:9080/locations}. Используйте библиотеку \texttt{Retrofit} для выполнения запросов к API. Пример запроса:

\begin{verbatim}
GET http://10.0.2.2:9080/locations?category=Natural
\end{verbatim}

Ответ от сервера будет содержать данные о достопримечательностях в формате JSON:

\begin{verbatim}
[
  {
    "id": "1",
    "name": "Eiffel Tower",
    "description": "Iconic symbol of Paris",
    "category": "Landmark",
    "latitude": 48.8584,
    "longitude": 2.2945
  }
]
\end{verbatim}

\subsection{Локальное хранение данных}
Для хранения результатов запросов используйте \texttt{Room}. После получения данных с сервера они должны сохраняться в локальной базе данных для работы в офлайн-режиме.

\subsection{Архитектура приложения}
Используйте архитектуру MVVM для разделения бизнес-логики и UI. ViewModel будет отвечать за выполнение запросов через \texttt{Retrofit}, сохранение данных в \texttt{Room} и их передачу в UI.

\subsection{Интерфейс пользователя}
Используйте \texttt{Jetpack Compose} для создания интерфейса. Интерфейс должен отображать список достопримечательностей, полученных с сервера или из локальной базы данных, если интернет-соединение отсутствует.

\subsection{Рекомендации}
- При подключении к интернету данные должны обновляться.
- В случае отсутствия интернета — отображать данные из локальной базы данных.
- Соблюдать принципы архитектуры MVVM и использовать \texttt{Retrofit}, \texttt{Room} и \texttt{Jetpack Compose}.
