%\section{Задание на разработку мобильного клиента}

Необходимо разработать мобильный клиент для работы с сервером, предоставляющим API для поиска книг по названию. Клиент должен выполнять следующие задачи:

%\subsection{Подключение к серверу} 

Клиент должен взаимодействовать с сервером по адресу 

\texttt{http://localhost:8080/books/search}. Используйте стандартные HTTP-запросы для выполнения запросов к API. Пример запроса:

\begin{verbatim} GET http://localhost:8080/books/search?title=1984 \end{verbatim}

Ответ от сервера будет содержать данные о книгах в формате JSON:

\begin{verbatim} [ { "id": "1", "title": "1984", "author": "George Orwell",
     "year": 1949 } ] \end{verbatim}

%\subsection{Локальное хранение данных} 

Для хранения результатов запросов используйте SQLite или другую подходящую локальную базу данных. После получения данных с сервера они должны сохраняться в локальной базе данных для работы в офлайн-режиме.

%\subsection{Архитектура приложения} 

Используйте архитектуру MVVM для разделения бизнес-логики и UI. ViewModel будет отвечать за выполнение запросов к API, сохранение данных в базу данных и их передачу в UI.

%\subsection{Интерфейс пользователя} 

Используйте нативные компоненты интерфейса для создания интерфейса. Интерфейс должен отображать список книг, полученных с сервера или из локальной базы данных в случае отсутствия интернет-соединения.

%\subsection{Рекомендации}

В случае отсутствия интернета — отображать данные из локальной базы данных.
Соблюдать принципы архитектуры MVVM и использовать подходящие библиотеки для HTTP-запросов и локального хранения данных.
