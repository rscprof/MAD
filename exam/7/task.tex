\section{Задание на разработку Android-клиента для поиска туристических мест}

Необходимо разработать Android-клиент для взаимодействия с сервером, предоставляющим REST API для поиска туристических мест. Сервер доступен по адресу \texttt{http://10.0.2.2:9080/search}. Приложение должно выполнять следующие задачи:

\subsection{Подключение к серверу} Клиент должен отправлять запросы к API сервера для поиска мест по имени. Используйте библиотеку \texttt{Retrofit} для выполнения запросов. Пример запроса:

\begin{verbatim} GET http://10.0.2.2:9080/search?name=eiffel \end{verbatim}

Ответ от сервера будет содержать данные в формате JSON:

\begin{verbatim} [ { "id": "1", "name": "Eiffel Tower", 
    "country": "France", "description": "Iconic iron lattice tower in Paris.", 
    "popularity": 98 } ] \end{verbatim}

\subsection{Локальное хранение данных} Используйте библиотеку \texttt{Room} для сохранения данных о туристических местах в локальную базу данных. Это позволит отображать сохранённые результаты в офлайн-режиме.

\subsection{Архитектура приложения} Организуйте архитектуру приложения по шаблону MVVM:

ViewModel отвечает за выполнение запросов через \texttt{Retrofit} и работу с \texttt{Room}.
UI получает данные из ViewModel.
\subsection{Интерфейс пользователя} Создайте интерфейс с помощью \texttt{Jetpack Compose}, который включает:

Поле для ввода имени места для поиска.
Кнопку "Найти".
Список найденных мест, отображающий имя, страну и описание.
\subsection{Функциональные требования}

При вводе имени места приложение отправляет запрос к серверу и отображает список найденных мест.
Результаты поиска автоматически сохраняются в локальной базе данных.
Если подключение к серверу недоступно, приложение отображает данные из локальной базы данных.
Реализуйте возможность обновления данных при наличии интернет-соединения.
\subsection{Рекомендации}

Для обработки запросов используйте корутины (\texttt{Kotlin Coroutines}).
Реализуйте обработку ошибок, таких как отсутствие интернета или недоступность сервера.
Предусмотрите автоматическое обновление данных в случае изменения состояния сети.
