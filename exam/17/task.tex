%\section{Задание на разработку Android-клиента для работы с сервером ресторанов}

Разработать Android-клиент для взаимодействия с сервером, предоставляющим REST API для поиска ресторанов по типу кухни. 

%\subsection{Основные требования}
\begin{itemize}
    \item Адрес сервера: \texttt{http://10.0.2.2:9080/restaurants}.
    \item Для выполнения запросов используйте \texttt{Retrofit}. Пример запроса:
\begin{verbatim}
GET http://10.0.2.2:9080/restaurants?cuisine=Japanese
\end{verbatim}
Ответ в формате JSON содержит список ресторанов с полями \texttt{id}, \texttt{name}, \texttt{cuisine}, \texttt{city} и \texttt{is_open}.
\end{itemize}

%\subsection{Хранение данных}
Сохранять данные запросов в локальной базе с помощью \texttt{Room} для работы в офлайн-режиме.

%\subsection{Архитектура}
Использовать архитектуру MVVM для разделения логики и интерфейса:
\begin{itemize}
    \item \texttt{ViewModel} отвечает за запросы к API и управление данными.
    \item \texttt{Repository} для взаимодействия с сервером и базой данных.
    \item UI на основе \texttt{Jetpack Compose}.
\end{itemize}

%\subsection{Интерфейс}
Реализовать экран для ввода типа кухни и отображения списка ресторанов, с поддержкой офлайн-режима.

%\subsection{Рекомендации}
Реализовать обработку ошибок (например, отсутствие параметра \texttt{cuisine}).
