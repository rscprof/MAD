\section{Задание на разработку Android-клиента}

Необходимо разработать Android-клиент для работы с сервером, предоставляющим REST API для получения информации о продуктах. Клиент должен выполнять следующие задачи:

\subsection{Подключение к серверу}
Клиент должен взаимодействовать с сервером по адресу \texttt{http://10.0.2.2:9080/products}. Используйте библиотеку \texttt{Retrofit} для выполнения запросов к API. Пример запроса:

\begin{verbatim}
GET http://10.0.2.2:9080/products?
min_price=50&max_price=200
\end{verbatim}

Ответ от сервера будет содержать данные о продуктах в формате JSON:

\begin{verbatim}
[
  {
    "id": "2",
    "name": "Груша",
    "price": 110.75,
    "shelf_life": 25,
    "is_organic": false
  }
]
\end{verbatim}

\subsection{Локальное хранение данных}
Для хранения результатов запросов используйте \texttt{Room}. После получения данных с сервера они должны сохраняться в локальной базе данных, чтобы обеспечить возможность работы в офлайн-режиме. Кэшированные данные должны быть доступны при отсутствии интернета.

\subsection{Архитектура приложения}
Используйте архитектуру MVVM для разделения бизнес-логики и UI. \texttt{ViewModel} будет отвечать за выполнение запросов через \texttt{Retrofit}, сохранение данных в \texttt{Room} и их передачу в UI.

\subsection{Интерфейс пользователя}
Используйте \texttt{Jetpack Compose} для создания интерфейса. Интерфейс должен отображать список продуктов, полученных с сервера или из локальной базы данных, если интернет-соединение отсутствует.

\subsection{Рекомендации}
\begin{itemize}
  \item При подключении к интернету данные должны обновляться.
  \item В случае отсутствия интернета — отображать данные из локальной базы данных.
  \item Соблюдать принципы архитектуры MVVM и использовать \texttt{Retrofit}, \texttt{Room} и \texttt{Jetpack Compose}.
\end{itemize}
