%\section{Задание на разработку мобильного клиента для поиска книг по году издания}

Необходимо разработать мобильный клиент для работы с сервером, предоставляющим API для поиска книг по диапазону годов издания. Клиент должен выполнять следующие задачи:

%\subsection{Подключение к серверу} 

Клиент должен взаимодействовать с сервером по адресу 

\texttt{http://localhost:9090/search}. Для выполнения запросов к API использовать стандартные HTTP-запросы. Пример запроса:

\begin{verbatim} GET http://localhost:9090/search?start=1900&end=2000 \end{verbatim}

Ответ от сервера будет содержать список книг в формате JSON:

\begin{verbatim} [ { "id": "1", "title": "1984", "author": "George Orwell", 
    "year": 1949 }] \end{verbatim}

%\subsection{Локальное хранение данных} 

Для хранения результатов запросов необходимо использовать SQLite или другую подходящую локальную базу данных. После получения данных с сервера, их нужно сохранять в базе данных для дальнейшей работы в офлайн-режиме.

%\subsection{Архитектура приложения} 

Рекомендуется использовать архитектуру MVVM для разделения бизнес-логики и интерфейса пользователя. ViewModel будет отвечать за выполнение запросов к API, обработку данных и передачу их в UI.

%\subsection{Интерфейс пользователя} 

Интерфейс должен отображать список книг, полученных с сервера или из локальной базы данных, если нет интернет-соединения. Для отображения данных использовать стандартные компоненты интерфейса мобильной платформы.

%\subsection{Рекомендации}

В случае отсутствия интернета отображать данные из локальной базы данных.
Соблюдать принципы архитектуры MVVM.
Использовать библиотеки Retrofit и Room.