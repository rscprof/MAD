\subsection{Семинары <<Функции высшего порядка в Kotlin/Функциональный подход в Kotlin>> (4 часа)}
% TODO доделать количество вариантов до 25 
В данной работе требуется написать не самую оптимальную реализацию, а реализацию, 
которая наиболее полноценно использует функции над
коллекциями, использующие функциональный подход, и строковые функции. В работе запрещено использовать mutable коллекции и var переменные.

Реализация должна состоять из одной строки с точечными вызовами, включая ввод и вывод, использовать рекурсию запрещено.

Примечание: данный способ реализации программы нужен исключительно в учебных целях, в дальнейшем разбивайте подобные решения на небольшие функции,
которые удобно повторно использовать.

\textbf{Задания №№1-3} Реализуйте задания первой, третьей и четвертой практических работ.

\textbf{Задание №4} С клавиатуры вводится несколько целых значений через пробел. Найдите (без учета тех чисел, где соответствующей цифры нет):

\begin{enumerate} %fold
%	\item Побитовое И последней цифры всех чисел
%	\item Побитовое ИЛИ последней цифры  всех чисел
%	\item Побитовое исключающее ИЛИ последней цифры всех чисел
	\item Побитовое И предпоследней цифры всех чисел
	\item Побитовое ИЛИ предпоследней цифры  всех чисел
	\item Побитовое исключающее ИЛИ предпоследней цифры всех чисел
	\item Побитовый штрих Шеффера последней цифры всех чисел (операции выполняются слева направо)
	\item Побитовый штрих Шеффера предпоследней цифры всех чисел (операции выполняются слева направо)
	\item Побитовую стрелку Пирса последней цифры всех чисел (операции выполняются слева направо)
	\item Побитовую стрелку Пирса предпоследней цифры всех чисел (операции выполняются слева направо)
	\item Побитовый штрих Шеффера последней цифры всех чисел (операции выполняются справа налево)
	\item Побитовый штрих Шеффера предпоследней цифры всех чисел (операции выполняются справа налево)
	\item Побитовую стрелку Пирса последней цифры всех чисел (операции выполняются справа налево)
	\item Побитовую стрелку Пирса предпоследней цифры всех чисел (операции выполняются справа налево)

	\item Побитовое И первой цифры всех чисел
	\item Побитовое ИЛИ первой цифры  всех чисел
	\item Побитовое исключающее ИЛИ первой цифры всех чисел
	\item Побитовое И второй цифры всех чисел
	\item Побитовое ИЛИ второй цифры  всех чисел
	\item Побитовое исключающее ИЛИ второй цифры всех чисел
	\item Побитовый штрих Шеффера первой цифры всех чисел (операции выполняются слева направо)
	\item Побитовый штрих Шеффера второй цифры всех чисел (операции выполняются слева направо)
	\item Побитовую стрелку Пирса первой цифры всех чисел (операции выполняются слева направо)
	\item Побитовую стрелку Пирса второй цифры всех чисел (операции выполняются слева направо)
	\item Побитовый штрих Шеффера первой цифры всех чисел (операции выполняются справа налево)
	\item Побитовый штрих Шеффера второй цифры всех чисел (операции выполняются справа налево)
	\item Побитовую стрелку Пирса первой цифры всех чисел (операции выполняются справа налево)
	\item Побитовую стрелку Пирса второй цифры всех чисел (операции выполняются справа налево)


\end{enumerate}

\textbf{Задание №5}

\begin{enumerate}
	\item С клавиатуры вводится информация о студентах: фамилия, имя, оценки. Выведите на экран информацию о трех лучших студентах по среднему баллу. В случае,
		если у нескольких студентов средний балл совпадает, то выведите большее число студентов (пока не будут выведены все студенты или 
		не будут полностью исчерпаны студенты с тремя лучшими баллами). Вывод надо осуществлять в порядке убывания среднего балла, а 
		для одинаковых средних баллов -- в алфавитном порядке по фамилии и имени.
	\item С клавиатуры вводится информация о студентах: фамилия, имя, оценки. Выведите на экран информацию о трех лучших студентах по максимальному баллу. В случае,
		если у нескольких студентов средний балл совпадает, то выведите большее число студентов (пока не будут выведены все студенты или 
		не будут полностью исчерпаны студенты с тремя лучшими баллами). Вывод надо осуществлять в порядке убывания максимального балла, а 
		для одинаковых максимальных баллов -- в алфавитном порядке по фамилии и имени.
	\item С клавиатуры вводится информация о студентах: фамилия, имя, оценки. Выведите на экран информацию о трех лучших студентах по минимальному баллу. В случае,
		если у нескольких студентов средний балл совпадает, то выведите большее число студентов (пока не будут выведены все студенты или 
		не будут полностью исчерпаны студенты с тремя лучшими баллами). Вывод надо осуществлять в порядке убывания минимального балла, а 
		для одинаковых минимальных баллов -- в алфавитном порядке по фамилии и имени.
	\item С клавиатуры вводится информация о студентах: фамилия, имя, оценки. Выведите на экран информацию о трех худших студентах по среднему баллу. В случае,
		если у нескольких студентов средний балл совпадает, то выведите большее число студентов (пока не будут выведены все студенты или 
		не будут полностью исчерпаны студенты с тремя худшими баллами). Вывод надо осуществлять в порядке возрастания среднего балла, а 
		для одниковых средних баллов -- в алфавитном порядке по фамилии и имени.
	\item С клавиатуры вводится информация о студентах: фамилия, имя, оценки. Выведите на экран информацию о трех худших студентах по максимальному баллу. В случае,
		если у нескольких студентов максимальный балл совпадает, то выведите большее число студентов (пока не будут выведены все студенты или 
		не будут полностью исчерпаны студенты с тремя худшими баллами). Вывод надо осуществлять в порядке возрастания максимального балла, а 
		для одинаковых максимальных баллов -- в алфавитном порядке по фамилии и имени.
	\item С клавиатуры вводится информация о студентах: фамилия, имя, оценки. Выведите на экран информацию о трех худших студентах по минимальному баллу. В случае,
		если у нескольких студентов минимальный балл совпадает, то выведите большее число студентов (пока не будут выведены все студенты или 
		не будут полностью исчерпаны студенты с тремя минимальными баллами). Вывод надо осуществлять в порядке возрастания минимального балла, а 
		для одинаковых минимальных баллов -- в алфавитном порядке по фамилии и имени.

	\item С клавиатуры вводится информация об абитуриентах: фамилия, имя, а далее названия предметов и оценки ЕГЭ по ним. Выведите на экран информацию о трех лучших абитуриентах по максимальному баллу за сумму трех ЕГЭ. В случае,
		если у нескольких абитуриентов средний балл совпадает, то выведите большее число абитуриентов (пока не будут выведены все абитуриенты или 
		не будут полностью исчерпаны абитуриентами с тремя лучшими баллами). Вывод надо осуществлять в порядке убывания максимальной суммы
		баллов за три ЕГЭ, а 
		для одинаковых сумм баллов -- в алфавитном порядке по фамилии и имени.
	\item С клавиатуры вводится информация об абитуриентах: фамилия, имя, а далее названия предметов и оценки ЕГЭ по ним. Выведите на экран информацию о трех худших абитуриентах по максимальному баллу за сумму трех ЕГЭ. В случае,
		если у нескольких абитуриентов средний балл совпадает, то выведите большее число абитуриентов (пока не будут выведены все абитуриенты или 
		не будут полностью исчерпаны абитуриентами с тремя худшими баллами). Вывод надо осуществлять в порядке возрастания максимальной суммы
		баллов за три ЕГЭ, а 
		для одинаковых сумм баллов -- в алфавитном порядке по фамилии и имени.
\end{enumerate}

\textbf{Задание №6}

\begin{enumerate}
	\item По номеру числа Фибоначчи найдите число Фибоначчи  (не используйте факты, которые вы не можете доказать самостоятельно)
	\item По числу Фибоначчи найдите его номер (не используйте факты, которые вы не можете доказать самостоятельно)
	\item По натуральному числу найдите его факториал
	\item По факториалу найдите исходное число
        \item По данному числу найдите простое число с таким номером (если простые числа нумеровать в порядке возрастания)
	\item По простому числу определите его номер в последовательности всех простых чисел, расположенных по возрастанию
	\item По данному числу найдите все его простые делители
	\item По числу $n$ найдите $n$-ое совершенное число (не используйте факты, которые вы не можете доказать самостоятельно)
	\item По совершенному числу найдите его номер в последовательности всех совершенных чисел, расположенных в порядке возрастания
                   (не используйте факты, которые вы не можете доказать самостоятельно)
	\item По натуральному числу найдите его двойной факториал
	\item По двойному факториалу найдите исходное число
	\item Рассмотрим все тройки натуральных чисел, удовлетворяющих уравнению $a^2+b^2=c^2$. Для данного $n$ найдите такую 
		тройку чисел $a$, $b$, $c$, что $a^2+b^2=c^2$, чтобы $a+b+c$ было меньше $n$ и наиболее близко к $n$.
	\item Рассмотрим все тройки натуральных чисел, удовлетворяющих уравнению $a^2+b^2=c^2$. Для данного $n$ найдите такую 
		тройку чисел $a$, $b$, $c$, что $a^2+b^2=c^2$, чтобы $a+b+c$ было больше $n$ и наиболее близко к $n$.
	\item По данному натуральному числу $n$ найдите наименьшее простое число, большее $n$
	\item По данному натуральному числу $n$ найдите наибольшее простое число, меньшее $n$
	\item По данному натуральному числу $n$ найдите наименьший факториал, больший $n$
	\item По данному натуральному числу $n$ найдите наибольший факториал, меньший $n$
	\item По данному натуральному числу $n$ найдите наименьший двойной факториал, больший $n$
	\item По данному натуральному числу $n$ найдите наибольший двойной факториал, меньший $n$
	\item По данному натуральному числу $n$ найдите наименьшее число Фибоначчи, большее $n$ (не используйте факты, которые вы не можете доказать самостоятельно)

	\item По данному натуральному числу $n$ найдите наибольшее число Фибоначчи, меньшее $n$ (не используйте факты, которые вы не можете доказать самостоятельно)

	\item По данному натуральному числу $n$ найдите наименьшее совершенное число, большее $n$ (не используйте факты, которые вы не можете доказать самостоятельно)

	\item По данному натуральному числу $n$ найдите наибольшее совершенное число, меньшее $n$ (не используйте факты, которые вы не можете доказать самостоятельно)
	\item Для данного натурального числа $n$ найдите такое простое число $p$, что входит в разложение на простые множители числа $n$ наибольшее число
		раз.
	\item Для данного натурального числа $n$ найдите такое простое число $p$, что входит в разложение на простые множители числа $n$ наименьшее число
		раз.
\end{enumerate}
