\subsection{Практическая работа <<функции Kotlin>> (2 часа)}

\textbf{Задание №1 (функции и лямбда-функции)} 

Разработайте программу в соответствии с вашим заданием при этом:

\begin{itemize}
\item основной алгоритм без ввода-вывода оформите в виде функции
\item в функцию передавайте как исходное число, так и условие отбора в виде лямбда-функции (проверка на четность/нечетность/кратность трем/некратность трем) в зависимости от варианта; функция должна корректно работать при изменении условия отбора
\end{itemize}

\begin{enumerate}
\item сумма четных цифр
\item сумма нечетных цифр
\item произведение четных цифр
\item произведение нечетных цифр
\item максимальную четную цифру
\item минимальную четную цифру
\item максимальную нечетную цифру
\item минимальную нечетную цифру

\item сумма цифр, кратных трем
\item сумма цифр, некратных трем
\item произведение цифр, кратных трем
\item произведение цифр, некратных трем
\item максимальную цифру, кратную трем
\item минимальную цифру, кратную трем
\item максимальную цифру, некратную трем
\item минимальную цифру, некратную трем

\item сумма цифр, стоящих на четных позициях в числе (если нумеровать цифры с конца): для числа 1234 ответ 4
\item произведение цифр, стоящих на четных позициях в числе (если нумеровать цифры с конца): для числа 1234 ответ 3
\item максимальная цифра среди стоящих на четных позициях в числе (если нумеровать цифры с конца): для числа 1234 ответ 3
\item минимальная цифра среди стоящих на четных позициях в числе (если нумеровать цифры с конца): для числа 1234 ответ 1

\item сумма цифр, стоящих на позициях в числе, номера которых кратны трем (если нумеровать цифры с конца): для числа 1234 ответ 2
\item произведение цифр, стоящих на позициях в числе, номера которых кратны трем (если нумеровать цифры с конца): для числа 1234 ответ 2
\item максимальная цифра среди стоящих на позициях в числе, номера которых кратны трем (если нумеровать цифры с конца): для числа 1234 ответ 2
\item минимальная цифра среди стоящих на позициях в числе, номера которых кратны трем (если нумеровать цифры с конца): для числа 1234 ответ 2

\item максимальная цифра среди стоящих на позициях в числе, номера которых кратны четырем (если нумеровать цифры с конца): для числа 1234 ответ 1 
\end{enumerate}

\textbf{Задание №2 (функции высшего порядка)} 



\begin{enumerate}
	\item Создайте функцию, которая по данным функциям с параметром типа Int и результатами типа Int возвращает новую функцию -- сумму данных 
		(количество исходных функций -- любое).
	\item Создайте функцию, которая по данным функциям с параметром типа Int и результатами типа Int возвращает новую функцию -- произведение данных
		(количество исходных функций -- любое).
	\item Создайте функцию, которая по данным функциям с параметром типа Int и результатами типа Int возвращает новую функцию -- максимум данных
		(количество исходных функций -- любое).
  	\item Создайте функцию, которая по данным функциям с параметром типа Int и результатами типа Int возвращает новую функцию -- минимум данных
		(количество исходных функций -- любое).
	\item Создайте функцию, которая по данной функции $f:Int->Int$ и числу $n$ возвращает функцию $f(f(f(...f(x)...)$, где $f$ вызывается $n$ раз.
  	\item Создайте функцию, которая по данным функциям без параметров и результатам типа String возвращает новую функцию без параметров,
		что возвращает конкатенацию данных (количество исходных функций -- любое).
	\item Создайте функцию, которая по данным функциям с параметром типа Int и результатами типа Int возвращает новую функцию с аргументом $x$ типа Int,
		которая возвращает номер первой функции, имеющей максимальное значение, при подстановке в качестве аргумента $x$.
		(количество исходных функций -- любое).
	\item Создайте функцию, которая по данным функциям с параметром типа Int и результатами типа Int возвращает новую функцию с аргументом $x$ типа Int,
		которая возвращает номер первой функции, имеющей минимальное значение, при подстановке в качестве аргумента $x$.
		(количество исходных функций -- любое).
	\item Создайте функцию, которая по данным функциям с параметром типа Int и результатами типа Int возвращает новую функцию с аргументом $x$ типа Int,
		которая возвращает номер последней функции, имеющей максимальное значение, при подстановке в качестве аргумента $x$.
		(количество исходных функций -- любое).
	\item Создайте функцию, которая по данным функциям с параметром типа Int и результатами типа Int возвращает новую функцию с аргументом $x$ типа Int,
		которая возвращает номер последней функции, имеющей минимальное значение, при подстановке в качестве аргумента $x$.
		(количество исходных функций -- любое).
	\item Создайте функцию, которая по данным двум функциям с параметром типа Int и результатами типа Int? возвращает новую функцию -- сумму данных.
		Если результат хотя бы одной из суммируемых функций -- null, то и результат возвращаемой функции -- null.
	\item Создайте функцию, которая по данным двум функциям с параметром типа Int и результатами типа Int? возвращает новую функцию -- произведение данных.
	Если результат хотя бы одной из умножаемых функций -- null, то и результат возвращаемой функции -- null.
	\item Создайте функцию, которая по данным двум функциям с параметром типа Int и результатами типа Int? возвращает новую функцию -- максимум данных.
		Если результат хотя бы одной из исходных функций -- null, то и результат возвращаемой функции -- null.
\item Создайте функцию, которая по данным двум функциям с параметром типа Int и результатами типа Int? возвращает новую функцию -- минимум данных.
			Если результат хотя бы одной из исходных функций -- null, то и результат возвращаемой функции -- null.
		\item Создайте функцию, которая по двум данным функциям f(x) и g(x) возвращает функцию f(g(x)), параметры всех упомянутых функций
			имеют тип Int, результат -- Int?. Если функция g для данного x дает результат null, то результирующая функция так же
			равна null.
	\item Создайте функцию, которая по данной функции с параметром типа Int и результатом типа Int, а также целому числу $n$
		возвращает новую функцию, которая по массиву из $n$ элементов типа Int возвращает массив результатов применения функции $f$ 
		к каждому элементу данного массива.
	\item Создайте функцию, которая по данному массиву целых чисел возвращает функцию, которая при каждом вызове последовательно
		возвращает элементы массива, а когда элементы кончатся -- null.
	\item Создайте функцию, которая по данной функции, имеющей аргумент типа Int и результат типа Int, возвращает функцию, 
		которая при каждом вызове последовательно возвращает результаты применения функции-аргумента к числам $1$, $2$, $3$, \dots.
	\item Создайте функцию, которая по данному массиву целых чисел возвращает функцию, которая при каждом вызове последовательно
		возвращает элементы массива в обратном порядке, а когда элементы кончатся -- null.
	\item Создайте функцию, которая по данной строке возвращает функцию, которая при каждом вызове последовательно
		возвращает символы строки, а когда символы кончатся -- null.
	\item Создайте функцию, которая по данной строке возвращает функцию, которая при каждом вызове последовательно
		возвращает символы строки в обратном порядке, а когда символы кончатся -- null.
	\item Создайте функцию, которая по данным функциям с параметром типа Float и результатами типа Float возвращает новую функцию -- среднее 
		арифметическое данных
		(количество исходных функций -- любое).
	\item Создайте функцию, которая по данным функциям с параметром типа Float и результатами типа Float возвращает новую функцию -- среднее 
		квадратическое данных
		(количество исходных функций -- любое).
	\item Создайте функцию, которая по данным функциям с параметром типа Float и результатами типа Float возвращает новую функцию -- среднее 
		геометрическое данных
		(количество исходных функций -- любое).
	\item Создайте функцию, которая по данной функции $f:Float->Float$ и числу $x$ возвращает функцию, которая 
		при каждом вызове последовательно возвращает $f(x)$, $f(f(x))$, $f(f(f(x)))$, $\dots$.

\end{enumerate}
