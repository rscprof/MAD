\subsection{Практическая работа <<функции Kotlin>> (2 часа)}

\textbf{Задание №1} %15 минут

Измените программу, сделанную в задании №1 практической работы №1: основной алгоритм вынесите в функцию. 
Опишите функцию несколькими способами: 
\begin{enumerate}
	\item как обычную функцию;
	\item как tailrec-функцию.
\end{enumerate}

Обратите внимание на то, что основной алгоритм не должен содержать ввод-вывод и работу с любыми глобальными самописными объектами.

\textbf{Задание №2} %5 минут

В программах, сделанных в задании №1, вынесите проверяемое условие (то есть то условие, что сформулировано в задании)
в отдельную single-expression функцию. 

\textbf{Задание №3} %5 минут

Функции, созданные в задании №1, модифицируйте таким образом, чтобы условие, по которому происходит отбор, можно было передавать как
аргумент (один из аргументов функции должен быть функционального типа со значением по умолчанию -- условием, что указано было в вашем варианте).

\textbf{Задание №4} %15 минут 

В задании №3 первой практической работы реализуйте следующее: выделите основной алгоритм в отдельную функцию (без ввода и вывода, использования
глобальных переменных), куда в качестве аргумента передавайте функцию, имеющую смысл -- способ сравнения двух чисел; 
изменяя данную функцию, пользователь вашей функции (то есть другой программист) должен иметь возможность получить
информацию либо о самых длинных, либо о самых коротких словах. Аналогичным образом передавайте в вашу функцию функцию, которая будет
определять условие отбора слов.

\textbf{Задание №5} %10 минут

Создайте функцию, которая реализует алгоритм второго задания первой практической работы, в которую все числа, слова или пары (в зависимости от варианта)
передаются в аргументах функции. Например: f(123,25,222); f("dfd","dd","ddd"); d (Pair(2,3),Pair(3,5),Pair(4,1)). В этом задании вывод может осуществлять
во вновь создаваемой функции.

\textbf{Задание №6} 



\begin{enumerate}
	\item Создайте функцию, которая по данным функциям с параметром типа Int и результатами типа Int возвращает новую функцию -- сумму данных 
		(количество исходных функций -- любое).
	\item Создайте функцию, которая по данным функциям с параметром типа Int и результатами типа Int возвращает новую функцию -- произведение данных
		(количество исходных функций -- любое).
	\item Создайте функцию, которая по данным функциям с параметром типа Int и результатами типа Int возвращает новую функцию -- максимум данных
		(количество исходных функций -- любое).
  	\item Создайте функцию, которая по данным функциям с параметром типа Int и результатами типа Int возвращает новую функцию -- минимум данных
		(количество исходных функций -- любое).
	\item Создайте функцию, которая по данной функции $f:Int->Int$ и числу $n$ возвращает функцию $f(f(f(...f(x)...)$, где $f$ вызывается $n$ раз.
  	\item Создайте функцию, которая по данным функциям без параметров и результатам типа String возвращает новую функцию без параметров,
		что возвращает конкатенацию данных (количество исходных функций -- любое).
	\item Создайте функцию, которая по данным функциям с параметром типа Int и результатами типа Int возвращает новую функцию с аргументом $x$ типа Int,
		которая возвращает номер первой функции, имеющей максимальное значение, при подстановке в качестве аргумента $x$.
		(количество исходных функций -- любое).
	\item Создайте функцию, которая по данным функциям с параметром типа Int и результатами типа Int возвращает новую функцию с аргументом $x$ типа Int,
		которая возвращает номер первой функции, имеющей минимальное значение, при подстановке в качестве аргумента $x$.
		(количество исходных функций -- любое).
	\item Создайте функцию, которая по данным функциям с параметром типа Int и результатами типа Int возвращает новую функцию с аргументом $x$ типа Int,
		которая возвращает номер последней функции, имеющей максимальное значение, при подстановке в качестве аргумента $x$.
		(количество исходных функций -- любое).
	\item Создайте функцию, которая по данным функциям с параметром типа Int и результатами типа Int возвращает новую функцию с аргументом $x$ типа Int,
		которая возвращает номер последней функции, имеющей минимальное значение, при подстановке в качестве аргумента $x$.
		(количество исходных функций -- любое).
	\item Создайте функцию, которая по данным двум функциям с параметром типа Int и результатами типа Int? возвращает новую функцию -- сумму данных.
		Если результат хотя бы одной из суммируемых функций -- null, то и результат возвращаемой функции -- null.
	\item Создайте функцию, которая по данным двум функциям с параметром типа Int и результатами типа Int? возвращает новую функцию -- произведение данных.
	Если результат хотя бы одной из умножаемых функций -- null, то и результат возвращаемой функции -- null.
	\item Создайте функцию, которая по данным двум функциям с параметром типа Int и результатами типа Int? возвращает новую функцию -- максимум данных.
		Если результат хотя бы одной из исходных функций -- null, то и результат возвращаемой функции -- null.
\item Создайте функцию, которая по данным двум функциям с параметром типа Int и результатами типа Int? возвращает новую функцию -- минимум данных.
			Если результат хотя бы одной из исходных функций -- null, то и результат возвращаемой функции -- null.
		\item Создайте функцию, которая по двум данным функциям f(x) и g(x) возвращает функцию f(g(x)), параметры всех упомянутых функций
			имеют тип Int, результат -- Int?. Если функция g для данного x дает результат null, то результирующая функция так же
			равна null.
	\item Создайте функцию, которая по данной функции с параметром типа Int и результатом типа Int, а также целому числу $n$
		возвращает новую функцию, которая по массиву из $n$ элементов типа Int возвращает массив результатов применения функции $f$ 
		к каждому элементу данного массива.
	\item Создайте функцию, которая по данному массиву целых чисел возвращает функцию, которая при каждом вызове последовательно
		возвращает элементы массива, а когда элементы кончатся -- null.
	\item Создайте функцию, которая по данной функции, имеющей аргумент типа Int и результат типа Int, возвращает функцию, 
		которая при каждом вызове последовательно возвращает результаты применения функции-аргумента к числам $1$, $2$, $3$, \dots.
	\item Создайте функцию, которая по данному массиву целых чисел возвращает функцию, которая при каждом вызове последовательно
		возвращает элементы массива в обратном порядке, а когда элементы кончатся -- null.
	\item Создайте функцию, которая по данной строке возвращает функцию, которая при каждом вызове последовательно
		возвращает символы строки, а когда символы кончатся -- null.
	\item Создайте функцию, которая по данной строке возвращает функцию, которая при каждом вызове последовательно
		возвращает символы строки в обратном порядке, а когда символы кончатся -- null.
	\item Создайте функцию, которая по данным функциям с параметром типа Float и результатами типа Float возвращает новую функцию -- среднее 
		арифметическое данных
		(количество исходных функций -- любое).
	\item Создайте функцию, которая по данным функциям с параметром типа Float и результатами типа Float возвращает новую функцию -- среднее 
		квадратическое данных
		(количество исходных функций -- любое).
	\item Создайте функцию, которая по данным функциям с параметром типа Float и результатами типа Float возвращает новую функцию -- среднее 
		геометрическое данных
		(количество исходных функций -- любое).
	\item Создайте функцию, которая по данной функции $f:Float->Float$ и числу $x$ возвращает функцию, которая 
		при каждом вызове последовательно возвращает $f(x)$, $f(f(x))$, $f(f(f(x)))$, $\dots$.

\end{enumerate}
