\subsection{Семинар <<Простейшие программы для Android>> (2 часа)}

Разработайте программу, работающую под управлением Android с использованием Jetpack Compose. 
Проверьте, что программа корректно работает с различными размерами экрана,
а также при повороте экрана. 

\begin{enumerate}
 \item Программа решения квадратного уравнения
 \item Программа решения неравенства вида $ax+b>0$
    \item Программа решения неравенства вида $ax+b<0$
 \item Программа решения неравенства вида $ax+b\geqslant 0$
 \item Программа решения неравенства вида $ax+b\leqslant 0$
\item Программа поиска дня недели по числу и месяцу в текущем году
% \item [ПП1-0-3] Программа поиска определителя матрицы $2\times2$
 \item  Программа перевода числа из $10$-ой в $16$-ую, $8$-ую и $2$-ую систем.
 \item  Программа поиска времени, когда окончится интервал. 
 Дано: часы и минуты начала интервала и количество минут, сколько он идет.
 Результат: часы и минуты окончания интервала.
% \item Программа поиска обратной матрицы для матрицы $2\times2$.
  \item Программа поиска обратной матрицы для матрицы $3\times3$.
 \item
 Программа поиска длины интервала. 
 Дано: часы и минуты начала интервала и часы и минуты конца интервала.
 Результат: количество минут в интервале.
 \item
 Программа умножения и деления двух комплексных чисел.
 
% \item[ПП1-0-9]
% Программа нахождения смешанного произведения трех векторов.
 
 \item
 Программа нахождения площади треугольника по координатам вершин.
 
 \item
 Программа нахождения углов треугольника по координатам вершин (проще всего это сделать по теореме косинусов).

  
 \item
 Программа перевода числа из $16$-ой, $8$-ой и $2$-ой системы в $10$-ую систему счисления.
 
 
 \item
 Программа нахождения количества денег на вкладе после окончания его срока по начальному взносу, проценту и срока в годах.
 
 
 \item
 Программа нахождения степени комплексного числа. Исходные данные: действительная, мнимая часть числа и степень. 
 Результат: действительная и мнимая часть результата.
 
 
 \item
 Программа умножения и деления чисел, представленных в виде обыкновенных дробей (состоящих из целой части, числителя и 
 знаменателя). Не забудьте выполнить сокращение дроби и приведение ее к правильному виду.

  \item
 Программа сложения и вычитания чисел, представленных в виде обыкновенных дробей (состоящих из целой части, числителя и 
 знаменателя). Не забудьте выполнить сокращение дроби и приведение ее к правильному виду.

 

 \item
 Программа определения по дате (число и месяц) знака зодиака.
 
 \item
Программа определения по обыкновенной дроби (числителю и знаменателю) периода десятичной дроби.
 
 \item
 Программа перевода комплесного числа из обычной формы в тригонометрическую и наоборот.

 
 \item
 Программа-игра Баше. При реализации этого задания не требуется ничего рисовать, вся информация вводится и выводится в виде
 чисел в обычные элементы управления.
 
 \item
 Программа разложения числа на простые множители.
 
 \item
 Программа нахождения наибольшего общего делителя и наименьшего общего кратного двух натуральных чисел.
 
 
 \item Программа-тест по предмету <<Разработка мобильных приложений>>. Создайте программу-тест из 10 вопросов с выбором
вариантов ответов и показом результатов прохождения теста.
\end{enumerate}

