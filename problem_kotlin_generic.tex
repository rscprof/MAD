\subsection{Практическая работа <<Абстрактное программирование в Kotlin>>}
% TODO усложнить, подумать об in- и out- параметрах, а также where

\textbf{Задание №1}

\begin{enumerate}
	\item Создайте функцию, которая по данным функциям с параметром любого типа и результатами типа Int возвращает новую функцию -- сумму данных 
		(количество исходных функций -- любое).
	\item Создайте функцию, которая по данным функциям с параметром любого типа и результатами типа Int возвращает новую функцию -- произведение данных
		(количество исходных функций -- любое).
	\item Создайте функцию, которая по данным функциям с параметром любого типа и результатами типа Int возвращает новую функцию -- максимум данных
		(количество исходных функций -- любое).
  	\item Создайте функцию, которая по данным функциям с параметром любого типа и результатами типа Int возвращает новую функцию -- минимум данных
		(количество исходных функций -- любое).
	\item Создайте функцию, которая по данной функции $f:T->T$ и числу $n$ возвращает функцию $f(f(f(...f(x)...)$, где $f$ вызывается $n$ раз. Здесь $T$ -- любой тип.
  	\item Создайте функцию, которая по данным функциям с единственным параметром типа $T$ и результатами типа String возвращает 
		новую функцию с параметром типа $T$, что возвращает конкатенацию данных (количество исходных функций -- любое).
	\item Создайте функцию, которая по данным функциям с параметром типа $T$ и результатами типа Int возвращает новую функцию с аргументом $x$ типа $T$,
		которая возвращает номер первой функции, имеющей максимальное значение, при подстановке в качестве аргумента $x$.
		(количество исходных функций -- любое). Здесь $T$ -- любой тип.
	\item Создайте функцию, которая по данным функциям с параметром типа $T$ и результатами типа Int возвращает новую функцию с аргументом $x$ типа $T$,
		которая возвращает номер первой функции, имеющей минимальное значение, при подстановке в качестве аргумента $x$.
		(количество исходных функций -- любое). Здесь $T$ -- любой тип.

	\item Создайте функцию, которая по данным функциям с параметром типа $T$ и результатами типа Int возвращает новую функцию с аргументом $x$ типа $T$,
		которая возвращает номер последней функции, имеющей максимальное значение, при подстановке в качестве аргумента $x$.
		(количество исходных функций -- любое). Здесь $T$ -- любой тип.





	\item Создайте функцию, которая по данным функциям с параметром типа $T$ и результатами типа Int возвращает новую функцию с аргументом $x$ типа $T$,
		которая возвращает номер последней функции, имеющей минимальное значение, при подстановке в качестве аргумента $x$.
		(количество исходных функций -- любое). Здесь $T$ -- любой тип.

	\item Создайте функцию, которая по данным двум функциям с параметром типа $T$ и результатами типа Int? возвращает новую функцию -- сумму данных.
		Если результат хотя бы одной из суммируемых функций -- null, то и результат возвращаемой функции -- null. Здесь $T$ -- любой тип.






	\item Создайте функцию, которая по данным двум функциям с параметром типа $T$ и результатами типа Int? возвращает новую функцию -- произведение данных.
	Если результат хотя бы одной из умножаемых функций -- null, то и результат возвращаемой функции -- null. Здесь $T$ -- любой тип.
	\item Создайте функцию, которая по данным двум функциям с параметром типа $T$ и результатами типа Int? возвращает новую функцию -- максимум данных.
	Если результат хотя бы одной из исходных функций -- null, то и результат возвращаемой функции -- null. Здесь $T$ -- любой тип.
	\item Создайте функцию, которая по данным двум функциям с параметром типа $T$ и результатами типа Int? возвращает новую функцию -- минимум данных.
	Если результат хотя бы одной из исходных функций -- null, то и результат возвращаемой функции -- null. Здесь $T$ -- любой тип.

	\item Создайте функцию, которая по двум данным функциям f(x) и g(x) возвращает функцию f(g(x)), параметры всех упомянутых функций
			имеют тип $T$, результат -- $T?$. Если функция g для данного x дает результат null, то результирующая функция так же
			равна null. Здесь $T$ -- любой тип.

	\item Создайте функцию, которая по данной функции с параметром типа $T$ и результатом типа Int, а также целому числу $n$
		возвращает новую функцию, которая по массиву из $n$ элементов типа $T$ возвращает массив результатов применения функции $f$ 
		к каждому элементу данного массива. Здесь $T$ -- любой тип.

	\item Создайте функцию, которая по данному массиву значений типа $T$ возвращает функцию, которая при каждом вызове последовательно
		возвращает элементы массива, а когда элементы кончатся -- null. Здесь $T$ -- любой тип.

	\item Создайте функцию, которая по данной функции, имеющей аргумент типа Int и результат произвольного типа, возвращает функцию, 
		которая при каждом вызове последовательно возвращает результаты применения функции-аргумента к числам $1$, $2$, $3$, \dots.
	\item Создайте функцию, которая по данному массиву значений произвольного типа возвращает функцию, которая при каждом вызове последовательно
		возвращает элементы массива в обратном порядке, а когда элементы кончатся -- null.
%	\item Создайте функцию, которая по данной строке возвращает функцию, которая при каждом вызове последовательно
%		возвращает символы строки, а когда символы кончатся -- null.
%	\item Создайте функцию, которая по данной строке возвращает функцию, которая при каждом вызове последовательно
%		возвращает символы строки в обратном порядке, а когда символы кончатся -- null.
%	\item Создайте функцию, которая по данным функциям с параметром типа Float и результатами типа Float возвращает новую функцию -- среднее 
%		арифметическое данных
%		(количество исходных функций -- любое).
%	\item Создайте функцию, которая по данным функциям с параметром типа Float и результатами типа Float возвращает новую функцию -- среднее 
%		квадратическое данных
%		(количество исходных функций -- любое).
%	\item Создайте функцию, которая по данным функциям с параметром типа Float и результатами типа Float возвращает новую функцию -- среднее 
%		геометрическое данных
%		(количество исходных функций -- любое).
%	\item Создайте функцию, которая по данной функции $f:Float->Float$ и числу $x$ возвращает функцию, которая 
%		при каждом вызове последовательно возвращает $f(x)$, $f(f(x))$, $f(f(f(x)))$, $\dots$.







	
\end{enumerate}

\textbf{Задание №2}


Обозначим $n$ -- номер варианта.

Функция добавления элемента в список выбирается учащимся исходя из значения \verb|n % 4+1|

\begin{enumerate}
\item \verb|fun push (el: T): Bool|
вставляет элемент в начало списка;
\item \verb|fun add (el: T): Bool|
вставляет элемент в конец списка;
\item \verb|fun insert (el: T,n: Int): Bool|
вставляет элемент на позицию n (нумерация идет с единицы) с начала списка;
\item \verb|fun insert (els:Array<T>,count: Int,n: Int): Bool|
вставляет count элементов массива els начиная с позиции n
\end{enumerate}

Процедура удаления элемента из списка выбирается учащимся исходя из значения \verb|n/4%4+1|:

\begin{enumerate}
	\item \verb|fun delete(): Bool| удаляет элемент из начала списка
	\item \verb|fun delete(): Bool| удаляет элемент с конца списка
	\item \verb|fun delete (n: Int): Bool| удаляет элемент с позиции n
	\item \verb|fun delete (count: Int,n: Int): Bool| удаляет count элементов начиная с позиции n
\end{enumerate}

Процедура печати элементов списка выбирается учащимся исходя из значения \verb|n mod 5+1|:

\begin{enumerate}
	\item \verb|fun print():Unit| печатает все элементы
\item \verb|fun print():Unit| печатает первый элемент списка
\item \verb|fun print():Unit| печатает последний элемент списка
\item \verb|fun print(n: Int): Unit| печатает элемент списка, имеющий позицию n
\item \verb|fun print(count: Int,n: Int): Unit| печатает count элементов списка, начиная с позиции n
\end{enumerate}

Кроме того должна быть реализована функция eraseAll, которая очищает весь список.

Список должен быть реализован в виде generic-класса.

