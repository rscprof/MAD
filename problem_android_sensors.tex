\subsection{Практическая работа <<Сенсоры>> (6 часов)}

Разработайте мобильное приложение, предназначенное для отслеживания активности пользователя согласно варианту. 

Информация о сохраненных активностях должна сохраняться между сеансами, пользователь должен иметь возможность корректировать результаты неточных измерений.

Ваша задача повысить точность измерения до возможного максимума.

Обратите внимание на использование всех изученных рекомендованных подходов к проектированию приложения.

Замечание: задание апробируется, потому некоторые варианты могут быть недостаточно корректными.

\begin{enumerate}
	\item учёта количества и глубины приседаний (в предположении, что телефон в руке, а рука поднимается во время приседа вверх);
	\item учёта количества отжиманий (в предположении, что телефон в кармане брюк);
	\item учёта количества и высоты прыжков на месте (телефон -- в кармане брюк);
	\item учёт длины прыжка в длину (телефон -- в кармане брюк);
	\item учёт длительности выполнения планки (телефон -- в кармане брюк, изначально человек стоит, а потом переходит в позу планки, в конце -- встаёт);
	\item измерение расстояния между точками (пользователь идет в одну точку, нажимает кнопку, идет в другую точку, нажимает кнопку);
	\item учёта количества шагов и скорости при беге на месте (в предположении, что телефон в кармане брюк);
	\item учёта количества подтягиваний (в предположении, что телефон в кармане брюк);
	\item учёта количества подъема туловища из положения лёжа (в предположении, что телефон - в кармане толстовки);
	\item учёта количества отжиманий от стены (в предположении, что телефон в кармане толстовки);
	\item учёта количества выпадов вперёд (в положении стоя, телефон -- в кармане брюк);
	\item учёта количества выпадов в сторону (в положении стоя, телефон -- в кармане брюк);
	\item учёта количества выпадов назад (в положении стоя, телефон -- в кармане брюк);
	\item учёта количества вращений обруча (телефон -- в кармане брюк);
	\item учёта количества и качества выполнений виньясы (телефон -- в кармане брюк);
	\item учёта количества прыжков через скакалку (телефон -- в кармане брюк);
	\item измерение глубины и прогресса наклона вперёд (пользователь держит телефон в руке);
	\item измерение качества выполнения мостика (телефон лежит на животе);
	\item измерение высоты вытяжения (телефон поднимается максимально высоко над головой);
	\item измерение количества и качества выполнения взмахов рук в противоположные стороны (одна рука вверх, другая вниз, телефон в руке);
	\item подсчёт времени проведенном в сидячем положении (телефон в кармане брюк);
	\item подсчёт времени пробегания 30 метров;
	\item подсчёт количества отжиманий, выполненных за 60 секунд;
	\item подсчёт количества приседаний, выполненных за 60 секунд;
	\item подсчёт количества подтягиваний, выполненных за 60 секунд.
\end{enumerate}


