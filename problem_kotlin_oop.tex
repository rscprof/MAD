\subsection{Практическая работа <<Особенности ООП в Kotlin>> (12 часов)} 
% TODO неплохо бы добавить UNIT-тестирование

Реализуйте с использованием ООП простейшую консольную базу данных (без красивого интерфейса) в соответствии со своим вариантом. 
Функции: добавление, изменение, удаление, сортировка, поиск, вывод на экран содержимого. База данных не предполагает сохранении информации между
сеансами работы, но подразумевает, что программа не <<падает>> при (почти) любых действиях пользователя. Исключение: переполнение памяти.

Осуществите распределение ваших классов, объектов и интерфейсов по различным файлам, а файлы сгруппируйте по папкам по какому-либо принципу.

Обратите внимание на то, что вам следует разделить ваш проект на Model и View, где Model будет без изменений перенесена в проект на Android, 
а View содержит логику консольного взаимодействия.

В ходе реализации строго используйте следующие принципы:
\begin{itemize}
	\item DRY
	\item SOLID
\end{itemize}

\begin{enumerate}
	\item База данных студентов группы. Поля: фамилия, имя, отчество, пол, возраст. 
	\item База данных расходов семьи. Поля: товар, стоимость, количество, дата.
	\item База данных загрузки аудиторий. Поля: дата и время, начала, дата и время конца, аудитория, преподаватель. 
	\item База данных учета доходов и расходов предпринимателя. Поля: дата, тип операции (доход/расход), объем операции, описание, 
корреспондент. 
	\item База данных велоклуба. Поля: ФИО, тип велосипеда (MTB и др.), стаж участия в велоклубе.
	\item База данных рейсов авиакомпании. Поля: дата и время вылета, аэропорт вылета, аэропорт прилета, дата и время прилета, 
марка самолета.
	\item База данных автобусных маршрутов. Поля: номер маршрута, номер парка, времена начала и окончания движения,
длина маршрута в км. 
	\item База данных электричек. Поля: вокзал, номер поезда, количество вагонов, тип (экспресс/обычный/спутник).
	\item База данных товаров Интернет-магазина. Поля: название товара, категория, цена товара, описание товара. 
	\item База заказов Интернет-магазина. Поля: ФИО заказчика, стоимость заказа, скидка (в процентах), адрес доставки. 
	\item База данных выборов. Поля: участок, кандидат, количество голосов.
	\item База данных практических работ. Поля: практическая работа, студент, номер варианта, номер уровня, 
дата сдачи, оценка. 
	\item База данных операторов и телеканалов. Поля: Название, тип (спутник, кабель, Интернет), охват (кол-во миллионов домохозяйств), минимальная
стоимость подписки. 
	\item База данных тарифных планов оператора. Поля: название, тип вещания (обычный/HD), флаг общедоступности. 
	\item База данных незаконно огороженных берегов. Поля: водный объект, регион, GPS-координаты, длина недоступного участка берега, дата фиксации нарушения.
	\item База данных временного прекращения движения в метро. Поля: дата и время начала прекращения
движения, дата и время окончания прекращения движения, станция, станция (от какой до какой станции прекращено движение).
	\item База данных проката фильмов. Поля: дата, время, кинотеатр, фильм, номер зала, тип сеанса (3D/Imax/обычный).
	\item База данных эвакуированных автомобилей. Поля: улица, автостоянка, GPS-координаты, 
тип нарушения (стоянка на проезжей части в месте запрета, стояна на тротуаре, стоянка на газоне), 
номер автомобиля, тип автомобиля (легковой/грузовой малой тонажности/грузовой большой
тонажности). 
	\item База данных средних специальных учебных учреждений. Поля: название, адрес, тип подчинения (федеральный/региональный), 
год основания, номер лицензии, номер аккредитации, дата окончания действия аккредитации. 
	\item База данных поселков. Поля: название, девелопер, площадь, количество жителей.
	\item База данных сухопутной военной техники. Поля: название, модель, разработчик, предприятие, стоимость, тип. 
	\item База данных деревьев в городе. Поля: GPS-координаты, вид дерева, округ, год посадки. 
	\item База данных футбольных матчей. Поля: дата, команда, команда, счет, место проведения. 
	\item База данных обращений жителей. Поля: дата, время, объект, заявитель, содержание обращения (до 255 символов), 
дата ответа, ответ на обращение (до 255 символов).
	\item База данных студентов колледжа. Поля: ФИО, группа, признак бюджетности, 
стипендия (нет/обычная/повышенная), флаг наличия социальной стипендии, дата рождения.
\end{enumerate}

