\subsection{Семинар <<Поддержка ООП в Kotlin>> (2 часа)} 
% TODO неплохо бы добавить UNIT-тестирование


% взять из основ программирования
Напишите программу осуществляющую ввод информации о сущностях, описанных в вашем варианте задания и вывод на экран
некоторых из них. Количество вводимых сущностей не ограничено; обязательно использовать
ООП, инкапсуляцию, наследование и полиморфизм. Проверять корректность входных данных и делать проверку того,
что хватает памяти не обязательно. 

\begin{enumerate}
\item Товары Интернет-магазина -- книги и диски. Все товары определяются ценой, книги имеют название,
автора, количество страниц; диски -- название, количество треков.
 Выведите на экран все товары со стоимостью меньше 100 рублей.
\item Преподаватели определяются ФИО. Для тех, кто имеют диссертацию дополнительно вводится ее название;
для остальных -- стаж работы. Вывести всех преподавателей, у которых ФИО начинается на букву <<А>>.
\item Телефоны определяются названием модели. Проводные телефоны дополнительно определяются типом номеронабирателя
(диск или кнопки); а беспроводные -- дальностью действия радиосигнала.
Вывести все телефоны, название которых начинается на <<А>>.
\item Покатушки определяются названием и расстоянием. Однодневные катушки дополнительно определяются
плановым временем поездки (в часах). Многодневные катушки определяются количеством дней и категорией сложности
похода (от 1 до 6). Вывести все покатушки длиной более 100 км.
\item Музыкальная композиция определяется названием и композитором. Дополнительно для песни указывается
автор стихов. Выведите информацию о всех композициях, у которых композитор начинается на букву <<А>>.
\item Олимпиада определяется названием. Если олимпиада участвует в программе приема в ВУЗы дополнительно
указывается уровень олимпиады (1--3), если олимпиада -- этап всероссийской, то указывается название этапа
(школьная, окружная, региональная, всероссийская), в остальных случаях -- размер призового фонда.
Выведите все олимпиады, название которых начинается на букву <<А>>.
\item Проездной билет определяется стоимостью. Билет на количество поездок определяется 
количеством поездок. Билет на неограниченное количество поездок
определяется сроком действия (1 день, 5 дней, 10 дней, 15 дней, месяц, три месяца, 6 месяцев, год).
Выведите информацию о билетах, стоимостью меньше 300 рублей.
\item Информация о студенте определяется ФИО. Для студентов, не имеющих автомата, указывается балл,
полученный на экзамене (2--5); для студентов, имеющих автомат указывается основание (олимпиада или контрольные работы).
В случае, если контрольная работа -- то также указывается средний балл за к/р.
Выведите всю информацию о студентах с фамилией, начинающейся на буквы от А до К.
\item Сотовый телефон определяется названием. Для смартфонов указывается операционная система. А для других телефонов
-- наличие браузера. Выведите информацию  о телефонах, название которых содержит слово <<Nokia>>.

\item Куртка определяется названием модели, наличием капюшона. 
Для мембранных курток указывается степень водонепроницаемости 
(число в мм рт. ст.), для остальных -- наличием пропитки.
Выведите информацию обо всех куртках, имеющих капюшон.

\item Жесткий диск определяется названием и емкостью. Внешние жесткие диски определяются дополнительно
наличием системы, смягчающей последствия падения. Внутренние жесткие диски -- размером (2.5/3.5 дюйма).
Выведите информацию о дисках, емкостью больше 200 Гб.

\item Велосипед определяется названием модели. Горному велосипеду соответствует количество скоростей, BMX -- тип
конструкции (фривил, кассетная, фрикостер). Выведите информацию обо всех велосипедах,
содержащих в названии <<Norco>>.

\item Электронная книга определяется названием и размером экрана. Для EInk-дисплея указывается поколение
(pearl, vizplex); для LCD -- количество поддерживаемых цветов. Выведите информацию о всех книгах с размером экрана
не менее  7 дюймов.

\item GPS определяется названием, диагональю экрана. Для переносных GPS указывается наличие велосипедного
крепления; для автомобильных -- поддержка отображения пробок и наличие радар-детектора.
Выведите информацию обо всех GPS с размером экрана менее 7 дюймов.

\item Пылесос определяется названием модели. Для обычного пылесоса указывается мощность, для пылесоса-робота --
размер убираемого помещения и количество виртуальных стен. Выведите информацию обо всех пылесосах,
содержащих в названии слово Indesit.


\item  Туры определяются названием. Для пляжного тура указывается тип пляжа (галечный, песок); для экскурсионного --
количеством экскурсий. Выведите информацию обо всех турах, содержащих слово Египет.
\item Язык программирования определяется названием. Алгоритмические языки определяются поддержкой ООП
(отсутствует, на классах, прототипная), остальные языки -- типом (функциональный, логический, стиль ). 
Выведите информацию обо всех языках, название которых начинается с буквы <<А>>.
\item Контрагенты определяются названием. Индивидуальные предприниматели дополнительно определяются
наличием счета в банка, а юридические лица -- формой организации (ООО, ОАО. ЗАО).
Выведите информацию обо всех контрагентах, название которых начинается с буквы <<А>>.
\item Счет в банке определяется номером. Для текущего счета указывается плата за обслуживание, 
для сберегательного счета -- проценты годовых и наличие капитализации.
Выведите информацию обо всех счетах, номер которого начинается с 408178...
\item Автомобильная дорога определяется названием и километражом. Бесплатная дорога определяется статусом автомагистрали
(автомагистраль или нет), а платная -- стоимостью за километр для обычных пользователей. Выведите информацию о дорогах,
длина которых менее 100 км.
\item Офисное здание определяется адресом. В случае наличия стоянки указывается количество машиномест и стоимость
аренды за месяц. Выведите информацию о зданиях, в адресе которых присуствует слово Тверская.
\item Товары Интернет-магазина -- GPS-навигаторы и карты. 
Все товары определяются ценой и названием, GPS-навигаторы имеют назначение (ручной, автомобильный) и признак 
возможности загрузки карт;
карты -- размером (в Мб).  Выведите информацию о всех товарах со стоимостью менее 4000 рублей.
\item Товары Интернет-магазина --  чаи и кофе.
Все товары определяются ценой, названием и весом, кофе -- типом (растворимый, молотый, в зернах), чаи -- типом
(черный, зеленый). Выведите информацию о всех товарах с весом менее 150 г.
\item Объекты продаваемые в коттеджном поселке: участки (определяются площадью, стоимостью, наличием подряда), дома 
(определяются этажностью, площадью и стоимостью). Выведите все объекты со стоимостью меньше 1000000 рублей.
\item Вопросам теста соответствует формулировка и количество баллов за правильный ответ. 
Вопросам с вариантами правильных ответов соответствует 4 варианта ответа и номер правильного ответа;
остальным вопросам -- формулировка правильного ответа. Выведите все вопросы, оцениваемые в 10 баллов и выше.
\item Слова определяются собственно словом. Для существительных указывается род, для глаголов --
спряжение. Выведите информацию обо всех словах, начинающихся на букву <<А>>.
\item Операционная система определяется названием. Для операционной системы на базе Linux указывается
название менеджера пакетов; для остальных -- стоимость лицензии. Вывести все операционные системы, у которых название
начинается на букву <<A>>.
\item Рюкзаки определяются названием модели и емкостью. Для городских рюкзаков указывается наличие <<вентилируемой
спины>> для походных -- количество отделений и наличие крепления для трекинговых палок. Вывести информацию обо
всех рюкзаках, в названии которых присутствует слово <<Trek>>. 
\item Автостоянка определяется названием, количеством машиномест. Для крытой автостоянки указывается 
количество этажей. Для открытой стоянки -- наличие охраны. Вывести информацию обо всех автостоянках с количеством мест больше 20.
\item Партия определяется названием. Для тех партий, что финансируются из бюджета указывается размер
ассигнований, а для остальных -- количество депутатов в каких-либо представительных органах власти. Вывести
информацию обо всех партиях с названиями, начинающимися на буквы от <<А>> до <<К>>.
\end{enumerate}

