\subsection{Семинар <<Массивы в Kotlin>> (2 часа)}

В третьей работе предполагается освоение массивов языка Kotlin.

При выполнении заданий обращайте внимание на использование специфических особенностей языка везде, где это возможно: 
if и when могут быть как операторами, так и частью выражений; фигурные скобки во многих случаях можно опускать;
точки с запятой почти никогда не используются, корректно выбирайте, как помечать переменные: ключевым словом var или val.

При выполнении работ обеспечивайте \textbf{оптимальность} предлагаемой программы как по скорости, так и по памяти. В
случае противоречия между двумя критериями, выбирайте алгоритм, который обеспечивает лучшее быстродействие.

В частности, это обозначает, что нельзя использовать дополнительные строки (в заданиях, кроме как для ввода, строки не нужны),
следует избегать сложных структур (списков, множеств).

\textbf{Во всех заданиях предполагается использование массивов (не списков, не множеств и других подобных структур), которые оптимальны по скорости для больших по объему исходных данных}

\begin{enumerate}
\item С клавиатуры вводится описание массива из $10$ элементов в виде:

\textit{номер}:\textit{значение}

однако, порядок указания элементов может быть любой. Выведите все элементы
массива в порядке возрастания номеров.

\item В строке указано несколько неотрицательных целых чисел, разделенных пробелами (по одному
пробелу между числами). Какие цифры присутствуют в каждом числе?

\item В строке указано несколько неотрицательных целых чисел, разделенных пробелами (по одному
пробелу между числами). Какие цифры присутствуют хотя бы в двух числах?

\item В строке указано несколько неотрицательных целых чисел, разделенных пробелами (по одному
пробелу между числами). Какие цифры присутствуют ровно в одном числе?

\item В строке указано несколько неотрицательных целых чисел, разделенных пробелами (по одному
пробелу между числами). Какие цифры присутствуют ровно в двух числах?

\item В строке указано несколько неотрицательных целых чисел, разделенных пробелами (по одному
пробелу между числами). Какие цифры отсутствуют ровно в двух числах?

\item В строке указано несколько неотрицательных целых чисел, разделенных пробелами (по одному
пробелу между числами). Какие цифры отсутствуют ровно в одном числе?



\item В строке указано несколько слов, разделенных пробелами (по одному
пробелу между словами). Какие символы присутствуют в каждом слове? 
Предполагается, что все символы в строке имеют код, не больший, чем 127.

\item В строке указано несколько слов, разделенных пробелами (по одному
пробелу между словами). Какие символы присутствуют хотя бы в двух словах?
Предполагается, что все символы в строке имеют код, не больший, чем 127.


\item В строке указано несколько слов, разделенных пробелами (по одному
пробелу между словами). Какие символы присутствуют ровно в одном слове?
Предполагается, что все символы в строке имеют код, не больший, чем 127.


\item В строке указано несколько слов, разделенных пробелами (по одному
пробелу между словами). Какие символы отсутствуют ровно в одном слове?
Предполагается, что все символы в строке имеют код, не больший, чем 127.

\item В строке указано несколько слов, разделенных пробелами (по одному
пробелу между словами). Какие символы отсутствуют ровно в двух словах?
Предполагается, что все символы в строке имеют код, не больший, чем 127.



\item В строке указано несколько неотрицательных целых чисел, разделенных пробелами (по одному
пробелу между числами). Какое количество чисел удовлетворяет условию отсутствия
повторяющихся цифр? 

\item В строке указано несколько неотрицательных целых чисел, разделенных пробелами (по одному
пробелу между числами). Какое количество чисел удовлетворяет условию наличия
повторяющихся цифр? 

\item В строке указано несколько слов, разделенных пробелами (по одному
пробелу между словами). Какое количество слов удовлетворяет условию отсутствия
повторяющихся символов? 
Предполагается, что все символы в строке имеют код, не больший, чем 127.


\item В строке указано несколько слов, разделенных пробелами (по одному
пробелу между словами). Какое количество слов удовлетворяет условию наличия
повторяющихся символов? 
Предполагается, что все символы в строке имеют код, не больший, чем 127.


\item В строке указано несколько неотрицательных целых чисел, разделенных пробелами (по одному
пробелу между числами). В каком количестве чисел присутствуют все цифры от
0 до 9?

\item Имеется некоторая последовательность цифр от $0$ до $9$. 
С клавиатуры вводится $9$ строк следующего вида:

\textit{цифра}->\textit{цифра}

Каждая строка обозначает, что после цифры, стоящей до стрелки, в последовательности стоит цифра, стоящая после стрелки. 

Выведите исходную последовательность.

\item В строке указано несколько неотрицательных целых чисел, разделенных пробелами (по одному
пробелу между числами). Какие цифры присутствуют в каждом числе дважды?

\item В строке указано несколько неотрицательных целых чисел, разделенных пробелами (по одному
пробелу между числами). Какие цифры присутствуют хотя бы в одном числе дважды?

\item В строке указано несколько неотрицательных целых чисел, разделенных пробелами (по одному
пробелу между числами). Какие цифры присутствуют ровно в одном числе дважды?

\item С клавиатуры вводится несколько строк, последняя строка -- пустая (пустая строка -- признак окончания ввода и дальше игнорируется). Выведите символы, что присутствуют в каждой строке. Предполагается, что коды всех символов в строке не превышают 127.

\item С клавиатуры вводится несколько строк, последняя строка -- пустая (пустая строка -- признак окончания ввода и дальше игнорируется). Выведите символы, что присутствуют ровно в одной строке. 
Предполагается, что коды всех символов в строке не превышают 127.

\item С клавиатуры вводится несколько строк, последняя строка -- пустая (пустая строка -- признак окончания ввода и дальше игнорируется). Выведите символы, что присутствуют ровно в двух строках.
Предполагается, что коды всех символов в строке не превышают 127.


\item С клавиатуры вводится несколько строк, последняя строка -- пустая (пустая строка -- признак окончания ввода и дальше игнорируется). Выведите символы, что отсутствуют ровно в двух строках.
Предполагается, что коды всех символов в строке не превышают 127.



\end{enumerate}
