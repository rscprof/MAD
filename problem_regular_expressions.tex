\subsection{Практическая работа <<Регулярные выражения>> (2 часа)}

Используя функции, реализующие работу с регулярными выражениями, разработайте программу проверки синтаксической
корректности фрагмента программы на языке Паскаль (классический диалект) в 
соответствии с вашим вариантом (номер варианта в данной работе отличается от предыдущих). 

Указание: проверьте соответствие вашего представления об языке \href{https://www.cs.utexas.edu/users/novak/iso7185.pdf}{стандарту языка}.

В заданиях нигде не предполагается, что можно использовать необязательные круглые скобки и составные операторы без явного указания на это.

\begin{enumerate}
\item[1] Программа проверки правильности оператора case с переменной-селектором типа char (без else), в который
вложены операторы присваивания переменной строкового литерала.
\item[2] Программа проверки правильности оператора присваивания, в правой части которого допустимы
операции сложения, вычитания, умножения, деления, переменные, целые и вещественные числа (включая показательную форму).
\item[3] Программа проверки правильности описания массива, у которого индексы могут иметь тип-название
(например, boolean), ограниченный тип (для integer), а тип-элемента -- название типа (идентификатор).
\item[4] Программа проверки правильности описания массива, у которого индексы могут иметь тип-название
(например, boolean), перечислимый тип, а тип-элемента -- ограниченный тип для integer.
\item[5] Программа проверки правильности описания массива, у которого индексы могут иметь тип-название
(например, boolean), ограниченный тип (для char), а тип-элемента -- перечислимый тип.
\item[6] Программа проверки описания перечислимого и ограниченного типа (для integer, char). Строка должна начинаться
со слова type.

\item[7] Программа проверки правильности оператора write(ln), у которого в качестве аргументов могут участвовать
строковый литералы, целые и вещественные числа, выражения над числами, соединенные операциями сложения, вычитания, умножения
и деления. Может также быть указан формат вывода.

\item[8] Проверка правильности последовательности операторов read(ln), присваивания и write(ln).
У writeln возможные аргументы --
строковые литералы; у read(ln) -- названия переменных, элементы массивов. У оператора присваивания
слева название переменной или элемент массива, справа число (целое или вещественное).

\item[9] Проверка правильности оператора for, у которого начальным и конечным значениями могут быть
как целые числа, так и символы, а тело цикла -- оператор write(ln), у которого аргументы --
целые и вещественные числа и переменные.

\item[10] Проверка правильности оператора if, у которого условие имеет вид: \verb| <переменная><знак><число> |,
при этом знак -- это знак больше, меньше, больше или равно, меньше или равно, равно, не равно; число -- целое или
вещественное. Как в части then, так и в else (может быть опущен) указывается оператор write(ln) с аргументами -- переменными
и строками.


\item[11] Проверка правильности последовательности операторов присваивания, правая часть которых -- выражения,
в которых используются литералы типа <<множество>> со значениями типа integer, переменные и операции +,-,*.

\item[12] Проверка правильности последовательности операторов присваивания, правая часть которых -- выражения,
в которых используются литералы типа <<множество>> со значениями типа char, переменные и операции +,-,*.

\item[13] Проверка правильности оператора while, у которого условие имеет вид: \verb| <переменная><знак><число> |,
при этом знак -- это знак больше, меньше, больше или равно, меньше или равно, равно, не равно; число -- целое или
вещественное. Телом цикла является либо снова оператор while, либо оператор присваивания, в правой части которого могут
быть представлены целые числа, либо переменные.

\item[14]
Проверка правильности оператора if, у которого условие имеет вид: \verb| <переменная><знак><строка> |,
при этом знак -- это знак больше, меньше, больше или равно, меньше или равно, равно, не равно. Как в части then, так и в
else (может быть опущен) указывается оператор присваивания, правая часть которого -- выражение, содержащее переменные, целые
и вещественные числа и знаки операций +,-,*,/.

\item[15] Проверка правильности оператора repeat..until,
у которого условие имеет вид: \verb| <переменная><знак><число> |,
при этом знак -- это знак больше, меньше, больше или равно, меньше или равно, равно, не равно; число -- целое или
вещественное. Телом цикла является либо снова оператор repeat..until, либо оператор присваивания, в правой части которого
могут быть представлены целые числа, либо переменные.

\item[16] Программа проверки правильности оператора case с переменной-селектором перечислимого типа (без else), в который
вложены операторы присваивания переменной константы типа integer.

\item[17] Проверка правильности последовательности операторов присваивания, правая часть которых -- выражения,
в которых используются литералы типа <<множество>> со значениями перечислимого типа, переменные и операции +, -, *.

\item[18] Проверка правильности оператора for, у которого начальным и конечным значениями могут быть
значения перечислимого типа, а тело цикла -- оператор присваивания, в правой части которого записана
числовая константа.

\item[19] Проверка правильности оператора if, у которого условие -- это логическое выражение над
переменными логического типа. Как в части then, так и в else (может быть опущен) указывается
оператор write(ln) с аргументами -- переменными и числами.

\item[20] Проверка правильности оператора while, у которого условие -- это логическое выражение над
переменными логического типа. Телом цикла является оператор присваивания, правая часть которого -- выражение, содержащее
целые константы и операции сложения, вычитания, умножения и деления.


\item[21] Проверка правильности оператора repeat..until, у которого условие -- это логическое выражение над
переменными логического типа. Телом цикла является оператор присваивания, правая часть которого -- выражение, содержащее
целые константы и операции сложения, вычитания, mod и div.

\item[22] Проверка правильности оператора repeat..until,
у которого условие имеет вид: \verb| <переменная><знак><строка> |,
при этом знак -- это знак больше, меньше, больше или равно, меньше или равно, равно, не равно; строка -- литерал типа
string. Телом цикла является оператор writeln, аргументами которого являются
целые и вещественные числа, возможно, с указанием формата.

\item[23] Проверка правильности оператора while, у которого условие имеет вид: \verb| <переменная><знак><строка> |,
при этом знак -- это знак больше, меньше, больше или равно, меньше или равно, равно, не равно; строка -- литерал
типа string. Телом цикла является либо снова оператор while, либо оператор writeln, аргументами которого являются
выражения, содержащие целые числа, имена переменных и операции +, -, *. /.

\item[24] Программа проверки правильности оператора write(ln), у которого в качестве аргументов могут участвовать
строковый литералы и переменные, соединенные операцией сложения; целые числа и переменные, соединенные операциями сложения, вычитания,
mod  и div. Может также быть указан формат вывода.

\item[25] Проверка правильности последовательности операторов присваивания, правая часть которых -- выражения,
в которых используются строковые литералы и переменные, соединенные знаком \verb|+|; а также выражения, в
которых используются литералы типа integer, переменные и операции сложения, вычитания, \verb|div| и \verb|mod|.
\end{enumerate}

