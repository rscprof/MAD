\subsection{Семинар <<Разработка первой программы на Kotlin>> (2 часа)}

Цель первых четырех семинаров -- обеспечить подготовку к выполнению заданий, 
нацеленных на освоение особенностей языка Kotlin. В первой работе
необходимо написать небольшую программу, чтобы убедиться в понимании
базовых конструкций языка (функция main, ветвления, циклы) и целый тип, а также обеспечить успешную настройку среды разработки.

При выполнении заданий обращайте внимание на использование специфических особенностей языка везде, где это возможно: 
if и when могут быть как операторами, так и частью выражений; фигурные скобки во многих случаях можно опускать;
точки с запятой почти никогда не используются, корректно выбирайте, как помечать переменные: ключевым словом var или val.

При выполнении работ обеспечивайте \textbf{оптимальность} предлагаемой программы как по скорости, так и по памяти. В
случае противоречия между двумя критериями, выбирайте алгоритм, который обеспечивает лучшее быстродействие.

В частности, это обозначает, что нельзя использовать дополнительные строки (в заданиях, кроме как для ввода, строки не нужны),
следует избегать сложных структур (списков, множеств).

\textbf{Задание} %for+if+ввод+вывод+строки: 15 минут


Для данного неотрицательного целого числа (в пределах Int)
найдите указанный результат. Осуществите проверку 
корректности ввода. Оформите программу и устраните все warning.

\begin{enumerate}
\item сумма четных цифр
\item сумма нечетных цифр
\item произведение четных цифр
\item произведение нечетных цифр
\item максимальную четную цифру
\item минимальную четную цифру
\item максимальную нечетную цифру
\item минимальную нечетную цифру

\item сумма цифр, кратных трем
\item сумма цифр, некратных трем
\item произведение цифр, кратных трем
\item произведение цифр, некратных трем
\item максимальную цифру, кратную трем
\item минимальную цифру, кратную трем
\item максимальную цифру, некратную трем
\item минимальную цифру, некратную трем

\item сумма цифр, стоящих на четных позициях в числе (если нумеровать цифры с конца): для числа 1234 ответ 4
\item произведение цифр, стоящих на четных позициях в числе (если нумеровать цифры с конца): для числа 1234 ответ 3
\item максимальная цифра среди стоящих на четных позициях в числе (если нумеровать цифры с конца): для числа 1234 ответ 3
\item минимальная цифра среди стоящих на четных позициях в числе (если нумеровать цифры с конца): для числа 1234 ответ 1

\item сумма цифр, стоящих на позициях в числе, номера которых кратны трем (если нумеровать цифры с конца): для числа 1234 ответ 2
\item произведение цифр, стоящих на позициях в числе, номера которых кратны трем (если нумеровать цифры с конца): для числа 1234 ответ 2
\item максимальная цифра среди стоящих на позициях в числе, номера которых кратны трем (если нумеровать цифры с конца): для числа 1234 ответ 2
\item минимальная цифра среди стоящих на позициях в числе, номера которых кратны трем (если нумеровать цифры с конца): для числа 1234 ответ 2

\item максимальная цифра среди стоящих на позициях в числе, номера которых кратны четырем (если нумеровать цифры с конца): для числа 1234 ответ 1 

\end{enumerate}

