\subsection{Практическая работа <<Основы языка Kotlin>> (2 часа)}

Цель первой работы -- обеспечить подготовку к выполнению заданий, 
нацеленных на освоение особенностей языка Kotlin. В первой работе
необходимо написать несколько небольших программ, чтобы убедиться в понимании
базовых конструкций языка (функция main, ветвления, циклы) и базовые типы (целые, вещественные, логический, символьный, массивы, строки), а также обеспечить успешную настройку среды разработки.

При выполнении заданий обращайте внимание на использование специфических особенностей языка везде, где это возможно: 
if и when могут быть как операторами, так и частью выражений; фигурные скобки во многих случаях можно опускать;
точки с запятой почти никогда не используются, корректно выбирайте, как помечать переменные: ключевым словом var или val.

При выполнении работ обеспечивайте \textbf{оптимальность} предлагаемой программы как по скорости, так и по памяти. В
случае противоречия между двумя критериями, выбирайте алгоритм, который обеспечивает лучшее быстродействие.

В частности, это обозначает, что нельзя использовать дополнительные строки (в заданиях, кроме как для ввода, строки не нужны),
следует избегать сложных структур (списков, множеств). В целом необходимо выполнить задание теми способами, которые использовались в 
курсе <<Основы алгоритмизации и программирования>>, но на новом языке.

\textbf{Задание №1} %for+if+ввод+вывод+строки: 15 минут


Для данного неотрицательного целого числа (в пределах Int)
найдите указанный результат. Осуществите проверку 
корректности ввода. Оформите программу и устраните все warning.

\begin{enumerate}
\item сумма четных цифр
\item сумма нечетных цифр
\item произведение четных цифр
\item произведение нечетных цифр
\item максимальную четную цифру
\item минимальную четную цифру
\item максимальную нечетную цифру
\item минимальную нечетную цифру

\item сумма цифр, кратных трем
\item сумма цифр, некратных трем
\item произведение цифр, кратных трем
\item произведение цифр, некратных трем
\item максимальную цифру, кратную трем
\item минимальную цифру, кратную трем
\item максимальную цифру, некратную трем
\item минимальную цифру, некратную трем

\item сумма цифр, стоящих на четных позициях в числе (если нумеровать цифры с конца): для числа 1234 ответ 4
\item произведение цифр, стоящих на четных позициях в числе (если нумеровать цифры с конца): для числа 1234 ответ 3
\item максимальная цифра среди стоящих на четных позициях в числе (если нумеровать цифры с конца): для числа 1234 ответ 3
\item минимальная цифра среди стоящих на четных позициях в числе (если нумеровать цифры с конца): для числа 1234 ответ 1

\item сумма цифр, стоящих на позициях в числе, номера которых кратны трем (если нумеровать цифры с конца): для числа 1234 ответ 2
\item произведение цифр, стоящих на позициях в числе, номера которых кратны трем (если нумеровать цифры с конца): для числа 1234 ответ 2
\item максимальная цифра среди стоящих на позициях в числе, номера которых кратны трем (если нумеровать цифры с конца): для числа 1234 ответ 2
\item минимальная цифра среди стоящих на позициях в числе, номера которых кратны трем (если нумеровать цифры с конца): для числа 1234 ответ 2

\item максимальная цифра среди стоящих на позициях в числе, номера которых кратны четырем (если нумеровать цифры с конца): для числа 1234 ответ 1 

\end{enumerate}

\textbf{Задание №2} % массивы и строки, 30 минут

\begin{enumerate}
\item С клавиатуры вводится описание массива из $10$ элементов в виде:

\textit{номер}:\textit{значение}

однако, порядок указания элементов может быть любой. Выведите все элементы
массива в порядке возрастания номеров.

\item В строке указано несколько неотрицательных целых чисел, разделенных пробелами (по одному
пробелу между числами). Какие цифры присутствуют в каждом числе?

\item В строке указано несколько неотрицательных целых чисел, разделенных пробелами (по одному
пробелу между числами). Какие цифры присутствуют хотя бы в двух числах?

\item В строке указано несколько неотрицательных целых чисел, разделенных пробелами (по одному
пробелу между числами). Какие цифры присутствуют ровно в одном числе?

\item В строке указано несколько неотрицательных целых чисел, разделенных пробелами (по одному
пробелу между числами). Какие цифры присутствуют ровно в двух числах?

\item В строке указано несколько неотрицательных целых чисел, разделенных пробелами (по одному
пробелу между числами). Какие цифры отсутствуют ровно в двух числах?

\item В строке указано несколько неотрицательных целых чисел, разделенных пробелами (по одному
пробелу между числами). Какие цифры отсутствуют ровно в одном числе?



\item В строке указано несколько слов, разделенных пробелами (по одному
пробелу между словами). Какие символы присутствуют в каждом слове? 
Предполагается, что все символы в строке имеют код, не больший, чем 127.

\item В строке указано несколько слов, разделенных пробелами (по одному
пробелу между словами). Какие символы присутствуют хотя бы в двух словах?
Предполагается, что все символы в строке имеют код, не больший, чем 127.


\item В строке указано несколько слов, разделенных пробелами (по одному
пробелу между словами). Какие символы присутствуют ровно в одном слове?
Предполагается, что все символы в строке имеют код, не больший, чем 127.


\item В строке указано несколько слов, разделенных пробелами (по одному
пробелу между словами). Какие символы отсутствуют ровно в одном слове?
Предполагается, что все символы в строке имеют код, не больший, чем 127.

\item В строке указано несколько слов, разделенных пробелами (по одному
пробелу между словами). Какие символы отсутствуют ровно в двух словах?
Предполагается, что все символы в строке имеют код, не больший, чем 127.



\item В строке указано несколько неотрицательных целых чисел, разделенных пробелами (по одному
пробелу между числами). Какое количество чисел удовлетворяет условию отсутствия
повторяющихся цифр? 

\item В строке указано несколько неотрицательных целых чисел, разделенных пробелами (по одному
пробелу между числами). Какое количество чисел удовлетворяет условию наличия
повторяющихся цифр? 

\item В строке указано несколько слов, разделенных пробелами (по одному
пробелу между словами). Какое количество слов удовлетворяет условию отсутствия
повторяющихся символов? 
Предполагается, что все символы в строке имеют код, не больший, чем 127.


\item В строке указано несколько слов, разделенных пробелами (по одному
пробелу между словами). Какое количество слов удовлетворяет условию наличия
повторяющихся символов? 
Предполагается, что все символы в строке имеют код, не больший, чем 127.


\item В строке указано несколько неотрицательных целых чисел, разделенных пробелами (по одному
пробелу между числами). В каком количестве чисел присутствуют все цифры от
0 до 9?

\item Имеется некоторая последовательность цифр от $0$ до $9$. 
С клавиатуры вводится $9$ строк следующего вида:

\textit{цифра}->\textit{цифра}

Каждая строка обозначает, что после цифры, стоящей до стрелки, в последовательности стоит цифра, стоящая после стрелки. 

Выведите исходную последовательность.

\item В строке указано несколько неотрицательных целых чисел, разделенных пробелами (по одному
пробелу между числами). Какие цифры присутствуют в каждом числе дважды?

\item В строке указано несколько неотрицательных целых чисел, разделенных пробелами (по одному
пробелу между числами). Какие цифры присутствуют хотя бы в одном числе дважды?

\item В строке указано несколько неотрицательных целых чисел, разделенных пробелами (по одному
пробелу между числами). Какие цифры присутствуют ровно в одном числе дважды?

\item С клавиатуры вводится несколько строк, последняя строка -- пустая (пустая строка -- признак окончания ввода и дальше игнорируется). Выведите символы, что присутствуют в каждой строке. Предполагается, что коды всех символов в строке не превышают 127.

\item С клавиатуры вводится несколько строк, последняя строка -- пустая (пустая строка -- признак окончания ввода и дальше игнорируется). Выведите символы, что присутствуют ровно в одной строке. 
Предполагается, что коды всех символов в строке не превышают 127.

\item С клавиатуры вводится несколько строк, последняя строка -- пустая (пустая строка -- признак окончания ввода и дальше игнорируется). Выведите символы, что присутствуют ровно в двух строках.
Предполагается, что коды всех символов в строке не превышают 127.


\item С клавиатуры вводится несколько строк, последняя строка -- пустая (пустая строка -- признак окончания ввода и дальше игнорируется). Выведите символы, что отсутствуют ровно в двух строках.
Предполагается, что коды всех символов в строке не превышают 127.



\end{enumerate}

\textbf{Задание №3}

Обратите внимание, что в этом задании требуется оптимальное решение 
(продвинутые функции дадут менее оптимальные решения из-за копирования). Можно предполагать,
что все символам соответствует одно значение типа char (однако,
если вы сделаете корректное решение (разумеется, вне пары), то это будет плюсом).

\begin{enumerate}
\item Найдите первый символ в первом максимально длинном слове с нечетным числом символов в строке (в строке указываются только слова, разделенные одним или несколькими пробелами). 
\item Найдите последний символ в первом максимально длинном слове с нечетным числом символов в строке (в строке указываются только слова, разделенные одним или несколькими пробелами). 
\item Найдите первый символ в последнем максимально длинном слове с нечетным числом символов в строке (в строке указываются только слова, разделенные одним или несколькими пробелами). 
\item Найдите последний символ в последнем максимально длинном слове с нечетным числом симвлов в строке (в строке указываются только слова, разделенные одним или несколькими пробелами). 
\item Найдите первый символ в первом самом коротком слове в строке с нечетным числом символов (в строке указываются только слова, разделенные одним или несколькими пробелами). 
\item Найдите последний символ в первом самом коротком слове в строке с нечетным числом символов (в строке указываются только слова, разделенные одним или несколькими пробелами). 
\item Найдите первый символ в последнем самом коротком слове в строке с нечетным числом символов (в строке указываются только слова, разделенные одним или несколькими пробелами). 
\item Найдите последний символ в последнем самом коротком слове в строке с нечетным числом символов (в строке указываются только слова, разделенные одним или несколькими пробелами). 

\item Найдите первый символ в первом максимально длинном слове с четным числом символов в строке (в строке указываются только слова, разделенные одним или несколькими пробелами). 
\item Найдите последний символ в первом максимально длинном слове с четным числом символов в строке (в строке указываются только слова, разделенные одним или несколькими пробелами). 
\item Найдите первый символ в последнем максимально длинном слове с четным числом символов в строке (в строке указываются только слова, разделенные одним или несколькими пробелами). 
\item Найдите последний символ в последнем максимально длинном слове с четным числом симвлов в строке (в строке указываются только слова, разделенные одним или несколькими пробелами). 
\item Найдите первый символ в первом самом коротком слове в строке с четным числом символов (в строке указываются только слова, разделенные одним или несколькими пробелами). 
\item Найдите последний символ в первом самом коротком слове в строке с четным числом символов (в строке указываются только слова, разделенные одним или несколькими пробелами). 
\item Найдите первый символ в последнем самом коротком слове в строке с четным числом символов (в строке указываются только слова, разделенные одним или несколькими пробелами). 
\item Найдите последний символ в последнем самом коротком слове в строке с четным числом символов (в строке указываются только слова, разделенные одним или несколькими пробелами). 

\item Найдите второй символ в первом максимально длинном слове с четным числом символов в строке (в строке указываются только слова, разделенные одним или несколькими пробелами). 
\item Найдите предпоследний символ в первом максимально длинном слове с четным числом символов в строке (в строке указываются только слова, разделенные одним или несколькими пробелами). 
\item Найдите второй символ в последнем максимально длинном слове с четным числом символов в строке (в строке указываются только слова, разделенные одним или несколькими пробелами). 
\item Найдите предпоследний символ в последнем максимально длинном слове с четным числом симвлов в строке (в строке указываются только слова, разделенные одним или несколькими пробелами). 
\item Найдите второй символ в первом самом коротком слове в строке с четным числом символов (в строке указываются только слова, разделенные одним или несколькими пробелами). 
\item Найдите предпоследний символ в первом самом коротком слове в строке с четным числом символов (в строке указываются только слова, разделенные одним или несколькими пробелами). 
\item Найдите второй символ в последнем самом коротком слове в строке с четным числом символов (в строке указываются только слова, разделенные одним или несколькими пробелами). 
\item Найдите предпоследний символ в последнем самом коротком слове в строке с четным числом символов (в строке указываются только слова, разделенные одним или несколькими пробелами). 

\item Найдите первый символ в первом максимально длинном слове с числом символов,кратным трем, в строке (в строке указываются только слова, разделенные одним или несколькими пробелами). 



\end{enumerate}

